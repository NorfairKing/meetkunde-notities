\theoremstyle{plain}
\newtheorem{thm}{Theorem}[chapter] %Reset counter elk hoofdstuk
\theoremstyle{definition}
\newmdtheoremenv{de}[thm]{Definitie} % Definitie met frame
\newtheorem{ei}[thm]{Eigenschap}
\newtheorem{st}[thm]{Stelling}
\newtheorem{gev}[thm]{Gevolg}
\newtheorem{pr}[thm]{Propositie}
\newtheorem{opm}[thm]{Opmerking}
\newtheorem{vb}[thm]{Voorbeeld}
\newtheorem{tvb}[thm]{Tegenvoorbeeld}
\newtheorem{lem}[thm]{Lemma}
\newmdtheoremenv{al}[thm]{Algoritme}


\newcommand{\deref}[1]{\footnote{Zie definitie \ref{#1} op pagina \pageref{#1}.}}
\newcommand{\stref}[1]{\footnote{Zie stelling \ref{#1} op pagina \pageref{#1}.}}
\newcommand{\eiref}[1]{\footnote{Zie eigenschap \ref{#1} op pagina \pageref{#1}.}}
\newcommand{\gevref}[1]{\footnote{Zie gevolg \ref{#1} op pagina \pageref{#1}.}}
\newcommand{\prref}[1]{\footnote{Zie propositie \ref{#1} op pagina \pageref{#1}.}}
\newcommand{\opmref}[1]{\footnote{Zie opmerking \ref{#1} op pagina \pageref{#1}.}}
\newcommand{\vbref}[1]{\footnote{Zie voorbeeld \ref{#1} op pagina \pageref{#1}.}}
\newcommand{\tvbref}[1]{\footnote{Zie tegenvoorbeeld \ref{#1} op pagina \pageref{#1}.}}
\newcommand{\lemref}[1]{\footnote{Zie lemma \ref{#1} op pagina \pageref{#1}.}}
\newcommand{\alref}[1]{\footnote{Zie algoritme \ref{#1} op pagina \pageref{#1}.}}
\newcommand{\secref}[1]{\footnote{Zie sectie \ref{#1} op pagina \pageref{#1}.}}

\newcommand{\hint}[1]{Hint: #1}

% Mooiere TODO's
\newcommand{\TODO}[1]{\todo[color=red,inline,size=\small]{TODO: #1}}
\newcommand{\extra}[1]{\todo[color=orange,inline,size=\small]{EXTRA: #1}}
\newcommand{\clarify}[1]{\todo[color=yellow,inline,size=\small]{CLARIFY: #1}}
\newcommand{\question}[1]{\todo[color=green,inline,size=\small]{QUESTION: #1}}

\newcommand{\waarom}[0]{\clarify{waarom?}}
\newcommand{\needed}[0]{\clarify{referentie ?}}

\newcommand{\term}[1]{\index{#1}\textbf{#1}}
\newcommand{\zb}[0]{{\footnotesize {\it Zonder bewijs\/}}}

\newcommand{\bx}[1]{ \framebox[1.1\width]{#1} }
\newcommand{\examen}[0]{\bx{{\bf Examenvraag!}}\\}