\documentclass[main.tex]{subfiles}
\begin{document}

\chapter{Euclidische meetkunde}
\label{cha:euclidische-meetkunde}

\begin{de}
  Zij $p\in\mathbb{A}^{n}$, $v_{p}$ en $w_{p}$ rakende vectoren in $T_{p}\mathbb{A}^{n}$.
  Het \term{Euclidisch scalair product} $\cdot$ van $v_{p}$ en $w_{p}$ definieren we als volgt:
  \[ v_{p} \cdot w_{p} = \sum_{i=1}^{n}v_{i}w_{i} \]
\end{de}

\begin{ei}
  Distributiviteit ten opzichte van de optelling van vectoren.\\
  Zij $u$, $v$ en $w$ raakvectoren in $T_{p}\mathbb{A}^{n}$ en $a$ en $b$ scalars in $\mathbb{R}$.
  \[ (au + bv) \cdot w = au\cdot w + bv \cdot w \]

  \begin{proof}
    \[
    \begin{array}{rll}
      (au + bv) \cdot w &= \sum_{i=1}^{n}(au + bv)_{i}w_{i} &\\
                        &= \sum_{i=1}^{n}(au_{i} + bv_{i})w_{i} &\\
                        &= \sum_{i=1}^{n}(au_{i}w_{i} + bv_{i}w_{i}) &\\
                        &= \sum_{i=1}^{n}au_{i}w_{i} + \sum_{i=1}^{n}bv_{i}w_{i} &= au\cdot w + bv \cdot w
    \end{array}
    \]
  \end{proof}
\end{ei}

\begin{ei}
  Commutativiteit.\\
  Zij $v$ en $w$ raakvectoren in $T_{p}\mathbb{A}$.
  \[ v\cdot w = w\cdot v\]
  \begin{proof}
    \[ 
    \begin{array}{rll}
      v\cdot w &= \sum_{i=1}^{n}v_{i}w_{i} &\\
               &= \sum_{i=1}^{n}w_{i}v_{i} &= w \cdot v
    \end{array}
    \]
  \end{proof}
\end{ei}

\begin{ei}
  Positiviteit.\\
  Zij $v$ een raakvector in $T_{p}\mathbb{A}$.
  \[ v \cdot v \ge 0 \]

  \begin{proof}
    \[
    v \cdot v =  \sum_{i=1}^{n}v_{i}v_{i} = \sum_{i=1}^{n}v_{i}^{2} \ge 0
    \]
  \end{proof}
\end{ei}

\begin{ei}
  Zij $v$ een raakvector in $T_{p}\mathbb{A}$.
  \[ v \cdot v = 0 \Leftrightarrow v = 0 \]

  \begin{proof}
    \[
    v \cdot v =  \sum_{i=1}^{n}v_{i}v_{i} = \sum_{i=1}^{n}v_{i}^{2} = 0 \Leftrightarrow v = 0
    \]
  \end{proof}
\end{ei}

\begin{de}
  Een \term{Euclidische ruimte} $\mathbb{E}^{n}$ is een affiene ruimte $\mathbb{A}^{n}$, uitgerust met het scalair product $\cdot$.
\end{de}

\begin{de}
  Zij $p\in\mathbb{A}^{n}$, $v_{p}$ een rakende vector in $T_{p}\mathbb{A}^{n}$.
  De \term{lengte} of \term{norm} van $v_{p}$ definieren we als volgt:
  \[ \Vert v_{p} \Vert = \sqrt{v_{p} \cdot v_{p}} = \sqrt{\sum_{i=1}^{n}v_{i}^{2}} \]
\end{de}

\begin{ei}
  Driehoeksongelijkheid.\\
  Zij $v$ en $w$ raakvectoren in $T_{p}\mathbb{A}$.
  \[ \Vert v + w \Vert \le \Vert v \Vert + \Vert w \Vert \]

  \[
  \Vert v + w \Vert
  = \sqrt{\sum_{i=1}^{n}(v+w)_{i}^{2}}
  = \sqrt{\sum_{i=1}^{n}(v_{i}+w_{i})^{2}}
  = \sqrt{\sum_{i=1}^{n}v_{i}^{2}+w_{i}^{2}+ 2v_{i}w_{i}} 
  \le \sqrt{\sum_{i=1}^{n}v_{i}^{2}+w_{i}^{2}} 
  = \Vert v \Vert + \Vert w \Vert
  \]
\end{ei}

\begin{ei}
  \label{ei:ongelijkheid-van-cauchy-schwartz}
  Ongelijkheid van Cauchy-Schwartz\\
  Zij $v$ en $w$ raakvectoren in $T_{p}\mathbb{A}$.
  \[ |v \cdot w| \le \Vert v \Vert \Vert w \Vert \]

  \begin{proof}
    Kies twee vectoren $v$ en $w$ en een scalar $lambda\in \mathbb{R}$.
    Beschouw nu $(v+\lambda w) \cdot (v + \lambda w)$.
    Dit is zeker positief.
    \[ (v+\lambda w) \cdot (v + \lambda w) = (v \cdot v) + 2 \lambda (v \cdot w) + \lambda^{2}(w \cdot w) \]
    Beschouw bovenstaande gelijkheid als een tweedegraadsveelterm $P$ in $\lambda$ die steeds positief is.
    De discriminant van $P$ is dus negatief.
    \[ 4(v \cdot w)^{2} - 4(v \cdot v)(w \cdot w) \le 0 \]
    \[ \Rightarrow  (v \cdot w)^{2} - (v \cdot v)(w \cdot w) \le 0 \]
    \[ \Leftrightarrow |v \cdot w| - \Vert v \Vert \Vert w \Vert \le 0 \]
    $|v \cdot w|$ is dus kleiner of gelijk aan $\Vert v \Vert \Vert w \Vert$.
  \end{proof}
\end{ei}

\begin{ei}
  Zij $v$ en $w$ raakvectoren in $T_{p}\mathbb{A}$.
  \[ |v \cdot w| = \Vert v \Vert \Vert w \Vert \Leftrightarrow v = \lambda w \]

  \begin{proof}
    Bekijk eerst opnieuw het bewijs van de ongelijkheid van Cauchy-Schwartz.\eiref{ei:ongelijkheid-van-cauchy-schwartz}
    De ongelijkheid wordt een gelijkheid als er een unieke $\lambda$ bestaat zodat $(v+\lambda w) \cdot (v + \lambda w) = 0$ geldt.
    Dit is precies wanneer er een lambda bestaat zodat $v=\lambda w$ geldt.
  \end{proof}
\end{ei}

\begin{de}
  De cosinus van de \term{hoek} $\theta$ tussen twee vectoren $v$ en $w$ met $v$ en $w$ beide niet nul definieren we als volgt:
  \[ v \cdot w = \Vert v \Vert \Vert w \Vert \cos(\theta) \]
\end{de}

\begin{de}
  We noemen twee vectoren $v$ en $w$ \term{orthogonaal} als hun scalair product nul is.
  \[ v \bot w \Leftrightarrow v \cdot w = 0 \]
\end{de}

\begin{de}
  We noemen een basis $v = \{ v_{1}, \dotsc, v_{n} \}$ van een rakende ruimte $T_{p}\mathbb{E}^{n}$ \term{orthogonaal} als de vectoren van $v$ onderling orthogonaal zijn.
\end{de}

\begin{de}
  We noemen een basis $v = \{ v_{1}, \dotsc, v_{n} \}$ van een rakende ruimte $T_{p}\mathbb{E}^{n}$ \term{orthonormaal} als de vectoren van $v$ onderling orthogonaal zijn en lengte $1$ hebben.
\end{de}

\begin{st}
  Zij $v$ een raakvector van $T_{p}\mathbb{E}^{n}$ en $e = \{ e_{1}, \dotsc, e_{n} \}$ een basis van $T_{p}\mathbb{E}^{n}$.
  We noemen het rechterlid de \term{orthonormale expansie} van $v$ ten opzichte van $e$.
  \[ v = \sum_{i=1}^{n} (v \cdot e_{i})e_{i} \]
\TODO{bewijs p 79}
\end{st}

\begin{gev}
  \label{gev:identiteit-van-parseval}
  De \term{identiteit van Parseval}\\
  Zij $e = \{e_{1},\dotsc,e_{n}\}$ een orthonormale basis van $T_{p}\mathbb{E}^{n}$. 
  \[ v \cdot w = \sum_{i=1}^{n}(v \cdot e_{i})(w \cdot e_{i})\]
\extra{bewijs}
\end{gev}

\TODO{procedure van graham schmidt, bekijk lineairealgebra en p 79.}

\begin{de}
  Zij $p$ en $q$ twee punten van $\mathbb{E}^{n}$, dan defini\"eren we de \term{afstand} tussen $p$ en $q$ als $d(p,q)$.
  \[ d(p,q) = \Vert \overline{pq} \Vert = \Vert p-q \Vert \]
\end{de}

\begin{ei}
  \label{ei:symmetrie-afstand}
  Symmetrie van de afstand\\
  Zij $p$ en $q$ twee punten van $\mathbb{E}^{n}$.
  \[ d(p,q) = d(q,p) \]
  \extra{bewijs}
\end{ei}

\begin{ei}
  \label{ei:afstand-positief}
  Positiviteit van de afstand.
  Zij $p$ en $q$ twee punten van $\mathbb{E}^{n}$
  \[ d(p,q) \ge 0 \]
  \extra{bewijs}
\end{ei}

\begin{ei}
  \label{ei:afstand-nul-gelijk}
  Zij $p$ en $q$ twee punten van $\mathbb{E}^{n}$
  \[ d(p,q) = 0 \Leftrightarrow p = q \]
  \extra{bewijs}
\end{ei}

\begin{ei}
  \label{ei:driehoeksongelijkheid-in-En}
  Driehoeksongelijkheid in $\mathbb{E}^{n}$.
  \[ d(p,r) \le d(p,q) + d(q,r) \]
  \extra{bewijs}
\end{ei}

\section{Euclidische transformaties}
\label{sec:eucl-transf}

\begin{de}
  \label{de:euclidische-transformatie}
  Een affiene transformatie $F$ met lineair deel $A$ en translatiedeel $b$ noemen we een \term{Euclidische transformatie} of een \term{isometrie van de Euclidische ruimte} als en slechts als $A$ een orthogonale transformatie is.
  \[ A \in O(n) = \{ X \in GL(n,\mathbb{R}^{n}) \ |\ X^{T}X = I \} \]
\end{de}

\subsection{Orthogonale transformaties}
\label{sec:orth-transf}



\end{document}
