\documentclass[main.tex]{subfiles}
\begin{document}

\chapter{Affiene meetkunde}
\label{cha:affiene-meetkunde}

\section{Affiene ruimte}
\label{sec:affiene-ruimte}

\begin{de}
  Een $n$-dimensionaal punt is een veeltal met $n$ co\"ordinaten.
  \[ p = (p_{1}, p_{2}, \ldots, p_{n}) \]
\end{de}

\begin{de}
  De optelling van twee punten is niet gedefini\"eerd, maar \emph{de optelling van een punt en een vector} is gedefini\"eerd door ze beide als re\"ele $n$-tallen te beschouwen en coordinaatsgewijs op te tellen.
\end{de}

\begin{de}
  Een $n$-dimensionale \emph{affiene ruimte} $\mathbb{A}^{n}$ bestaat uit $n$-dimensionale punten.
  \[ \mathbb{A}^{n} = \left\{\{ (p_{1}, p_{2}, \ldots, p_{n}) \ |\ p_{i} \in \mathbb{R} \right\} \]
\end{de}

\begin{de}
  $\mathbb{A}^{2}$ noemen we \emph{het affiene vlak}.
\end{de}

\begin{de}
  Zij $p \in \mathbb{A}^{n}$ een punt van de $n$-dimensionale affiene ruimte $\mathbb{A}^{n}$ en zij $v \in \mathbb{R}^{n}$ een $n$-dimensionale re\"ele vector.
  Een koppel $(p,v)$ noemen we een \emph{raakevector} met aangrijpingspunt $p$ en vectordeel $v$.
  \[ v_{p} = (p,v) \]
\end{de}

\begin{de}
  Twee raakvectoren $v_{p} = (p,v)$ en $w_{q} = (q,w)$ zijn gelijk als elk zowel hun aangrijpingspunten en vectordelen gelijk zijn.
  \[ v_{p} = w_{q} \Leftrightarrow p = q \wedge v = w \]
\end{de}

\begin{de}
  De \emph{rakende ruimte} $T_{p}\mathbb{A}^{n}$ \emph{in een punt} $p$ \emph{aan een affiene ruimte} $\mathbb{A}^{n}$ is de verzameling van raakvectoren met $p$ als aangrijpingspunt in $\mathbb{A}^{n}$.
  \[ \left\{\{v_{p} = (p,v)\ |\ v \in \mathbb{R}^{n} \right\} \]
  Het symbool $T$ in de notatie van de rakende ruimte staat voor 'tangent'.
\end{de}

\begin{de}
  De verzameling van alle raakvectoren $v_{p}$ aan punten $p$ in een affiene ruimte $\mathbb{A}^{n}$ noemen we de \emph{rakende bundel} m$T\mathbb{A}^{n}$ van die affiene ruimte $\mathbb{A}^{n}$.
  \[
  \begin{array}{rl}
  T\mathbb{A}^{n} &= \left\{v_{p}\ |\ p \in \mathbb{A}^{n}, v \in \mathbb{R}^{n} \right\}\\
                 &= \left\{(p,v)\ |\ p \in\mathbb{A}^{n}, v \in \mathbb{R}^{n} \right\}\\
                 &= \mathbb{A}^{n} \times \mathbb{R}^{n}
  \end{array}
  \]
\end{de}

\begin{de}
  Zij $v_{p}$ en $w_{p}$ twee raakvectoren in de rakende ruimte $T_{p}\mathbb{A}^{n}$ van hetzelfde punt $p$, dan defini\"eren we \emph{de som} $v_{p}+w_{p}$ als volgt.
  \[ v_{p} \boldsymbol{ + } w_{p} = (v + w)_{p}\]
  Merk op dat de $\boldsymbol{ + }$ verschilt van de $+$. $\boldsymbol{ + }$ is de optelling voor raakvertoren en $+$ is de optelling voor vrije vectoren. Verder zullen we deze beide als $+$ schrijven.
  De som van twee raakvectoren met een verschillend aangrijpingspunt is niet gedefinieerd.
\end{de}

\begin{de}
  Zij $v_{p}$ een raakvector aan een punt $p$ van de affiene ruimte $\mathbb{A}^{n}$ en $\lambda \in \mathbb{R}$ een re\"eel getal, dan definieren we \emph{het scalair product} $\lambda v_{p}$ als volgt. 
  \[ \lambda v_{p} = (\lambda v)_{p}\]
\end{de}

\begin{st}
  \label{st:rakende-ruimte-is-vectorruimte}
  Elke rakende ruimte $T_{p}\mathbb{A}^{n}$ in een punt $p$ aan $\mathbb{A}^{n}$ vormt een re\"ele vectorruimte.
  \begin{proof}
    We bewijzen de axioma's van een vectorruimte niet opnieuw.
    Ze gelden in $\mathbb{R}^{n}$, dus ze gelden in $T_{p}\mathbb{A}^{n}$.
  \end{proof}
\end{st}

\begin{st}
  \label{st:phi-isomorphisme}
  Voor elk willekeurig punt $p$ van de affiene ruimte $\mathbb{A}^{n}$ is de afbeelding van de raakvector op de vector een isomorfisme van de rakende ruimte in dat punt en de re\"ele vectorruimte $\mathbb{R}^{n}$.
  $phi_{p}$ is dus een isomorphisme.
  \[ \phi_{p}: T_{p}\mathbb{A}^{n} \rightarrow \mathbb{R}^{n}: v_{p} \mapsto v \]

  \begin{proof}
    Een isomorphisme is een bijectieve lineaire afbeelding.
    \begin{itemize}
    \item $\phi_{p}$ is een bijectie.
      \begin{itemize}
      \item $\phi_{p}$ is een injectie.
      \[ \forall v_{p},w_{p} \in T_{p}\mathbb{A}^{n}: \phi_{p}(v_{p}) = \phi_{p}(w_{p}) \Rightarrow v_{p} = w_{p}\]
      \item $\phi_{p}$ is een surjectie.
      \[ \forall v \in \mathbb{R}^{n},\ \exists v_{p} \in T_{p}\mathbb{A}^{n}: \phi_{p}(v_{p}) = v \]
      \end{itemize}
    \item $\phi_{p}$ bewaart de lineariteit:
    \[ \phi_{p}(v_{p}+w_{p}) = \phi_{p}((v+w)_{p}) = v + w = \phi_{p}(v_{p}) + \phi_{p}(w_{p}) \] 
    \[ \phi_{p}(\lambda v_{p}) = \phi_{p}(\lambda v)_{p}) = \lambda v = \lambda\phi_{p}(v_{p})\]
    \end{itemize}
  \end{proof}
\end{st}

\begin{st}
  Voor elke twee willekeurige punten $p$ en $q$ van de affine ruimte $\mathbb{A}^{n}$ zijn de rakende ruimten isomorf.
  $\psi$ is dus een isomorphisme.
  \[ \psi_{pq}: T_{p}\mathbb{A}^{n} \rightarrow T_{q}\mathbb{A}^{n}: v_{p} \mapsto v_{q}\]

  \begin{proof}
    $\psi_{pq}$ is een samenstelling van isomorphismen\footnote{Zie het isomorphisme $\phi_{p}$ (stelling \ref{st:phi-isomorphisme})}, en bijgevolg ook een isomorphisme.
    \[ \psi_{pq} = \phi_{p}^{-1} \circ\phi_{p}\]
  \end{proof}
\end{st}

\section{Affiene deelruimten}
\label{sec:affiene-deelruimten}

\begin{de}
  \label{de:affiene-deelruimte}
  Zij $p$ een punt in de affiene ruimte $\mathbb{A}^{n}$ in $V$ een $k$-dimensionale deelruimte van $\mathbb{R}^{n}$ met $0 \le k \le n$.
  We noemen $p + V$ de \emph{affiene deelruimte} van $\mathbb{A}^{n}$ met \emph{richting} $V$ en \emph{aangrijpingspunt} $p$.
  \[ p + V = \left\{ p + v \in \mathbb{A}^{n}\ |\ v \in V \right\} \]
\end{de}

\begin{de}
  Affiene ruimten met \'e\'en dimensie noemen we \emph{affiene rechten}.
\end{de}

\begin{de}
  Affiene ruimten met twee dimensies noemen we \emph{affiene vlakken}.
\end{de}

\begin{de}
  Wanneer we spreken over $n$-dimensionale affiene ruimten noemen we $n-1$-dimensionale affiene ruimten \emph{affiene hypervlakken}.
\end{de}

\begin{lem}
\label{lem:equivalenties-affiene-deelruimte}
  Zij $p$ en $q$ twee punten uit de affiene ruimte $\mathbb{A}^{n}$ en $V$ een lineaire deelruimte van $\mathbb{R}^{n}$. De volgende uitspraken zijn equivalent:
  \begin{enumerate}
  \item $q \in p+V$.
  \item $\overrightarrow{pq} \in V$
  \item $p + V = q + V$ (Het aangrijpingspunt van een affiene deelruimte is niet uniek.)
  \end{enumerate}

  \begin{proof}
    Bewijs door circulaire implicaties.
    \begin{itemize}
    \item (1) $\Rightarrow$ (2)\\
      Als $q \in p+V$ geldt, dan bestaat er een $v \in V$ zodat $q = p + v$ geldt.\footnote{Zie de definitie van affiene deelruimten. (Definitie \ref{de:affiene-deelruimte})}.
      $\overrightarrow{pq} = q - p$ is danprecies $v$, en we toonden net dat $v \in V$ geldt.
    \item (2) $\Rightarrow$ (3)\\
      Om de gelijkheid van deze twee verzamelingen aan te tonen bewijzen we de twee inclusies:
      \begin{itemize}
      \item $\forall x \in p + V:\ x \in q + V$\\
        Kies een willekeurige $x\in p + V$. Er bestaat nu een $v \in V$ zodat $x = p + v$ geldt.\footnote{Zie de definitie van affiene deelruimten. (Definitie \ref{de:affiene-deelruimte})}.
        \[ p + v = p + v + q - q = q - \overrightarrow{pq} + v\]
        Omdat $\overrightarrow{pq} \in V$ geldt, geldt ook $- \overrightarrow{pq} + v \in V$\footnote{De optelling is intern in een vectorruimte.}.
        Noem nu $- \overrightarrow{pq} + v = w$, dan bestaat er dus een $w \in V$ zodat $x = q + w$ en zit $x$ bijgevolg ook in $q + V$.
      \item $\forall x \in q + V:\ x \in p + V$\\
        Hernoem $p$ naar $q$ en omgekeerd en kijk naar het vorige puntje.
      \end{itemize}
    \item (3) $\Rightarrow$ (1)\\
      $q$ zit steeds in $q + V$ (tel bij $q$ de nulvector op).
      Omdat $p + V$ en $q + V$ gelijk zijn zit $q$ dus ook in $p + V$.
    \end{itemize}
  \end{proof}
\end{lem}

\begin{st}
  Twee affiene deelruimten $p + V$ en $q + W$ zijn gelijk als en slechts als de deelrruimten $V$ en $W$ gelijk zijn, en het verschil tussen $p$ en $q$ als vector in $V = W$ zit.
  \[ p + V = q + W \Leftrightarrow V = W \wedge \overrightarrow{pq} \in V\]

  \begin{proof}
    Bewijs van een equivalentie.
    Kies willekeurige deelruimten $V$ en $W$ van $\mathbb{R}^{n}$ en punten $p$ en $q$ uit de affiene ruimte $\mathbb{A}^{n}$.
    \begin{itemize}
    \item $\Rightarrow$\\
      Als $p + V = q + W$ geldt, dan zit $q$ in $p + V$ en geldt $\overrightarrow{pq} \in V$.\footnote{Zie lemma \ref{lem:equivalenties-affiene-deelruimte}.}
      We bewijzen nu beide inclusies om aan te tonen dat $V$ en $W$ gelijk zijn.
      \begin{itemize}
      \item $\forall x \in V:\ x \in W$\\
      Kies een willekeurige $v \in V$.
      $p + v \in p + V$ geldt en $p + V = q + W$, dus er bestaat een $w \in W$ zodat $q + w = p + v$ geldt.
      \[ w = (p-q) + v \]
      Vermits $\overrightarrow{pq} \in V$ geldt, zit $w$ in $V$.\footnote{De optelling is intern in een vectorruimte.}.
      \item $\forall x \in W:\ x \in V$\\
        Hernoem $V$ naar $W$ en omgekeerd en kijk naar het vorige puntje
      \end{itemize}
    \item $\Leftarrow$\\
      Dit is al bewezen in deel 3 van het vorige lemma.\footnote{Zie lemma \ref{lem:equivalenties-affiene-deelruimte}.}
    \end{itemize}
  \end{proof}
\end{st}

\begin{st}
  \label{st:affiene-deelruimten-niet-lege-doorsnede-gelijk}
  Twee affiene deelruimten $p + V$ en $q + W$ van $\mathbb{R}^{n}$ zijn gelijk als en slechts als de deelrruimten $V$ en $W$ gelijk zijn, en hun doorsnede niet leeg is.
  \[ p+V = q+W \Leftrightarrow V = W \wedge p+V \cap q+W \neq \emptyset \]
  
  \begin{proof}
    \begin{itemize}
    \item $\Rightarrow$\\
      Twee gelijke deelruimten zijn uiteraard gelijk. Hun doorsnede is dan ook niet leeg, want die is gelijk aan de volledige deelruimte.
    \item $\Leftarrow$\\
      Stel dat de doorsnede van $p + V$ en $q + W$ niet leeg is, dan bestaat er een punt $a$ in $p + V \cap q + W$.
      \[ \exists a \in p + V \cap q + W \]
      Dat betekent dat er twee vectoren $v \in V$ en $w \in W$ bestaan zodat het volgende geldt:
      \[ a = p + v \text{ en} a = q + W \]
      \[ p+v = q+w \Leftrightarrow q-p = v-w\]
      Nu geldt dat $p + V$ en $q + W$ gelijk zijn.\footnote{Zie lemma \ref{lem:equivalenties-affiene-deelruimte}.}
    \end{itemize}
  \end{proof}
\end{st}

\begin{de}
\label{de:parallel}
  Twee affiene deelruimten $S = p + V$ en $T = q + W$ van $\mathbb{A}^{n}$ zijn \emph{parrallel} als hun richtingen $V$ en $W$ gelijk zijn.
  \[ V = W \Leftrightarrow S \parallel T \]
\end{de}

\begin{de}
\label{de:zwak-parallel}
  Twee affiene deelruimten $S = p + V$ en $T = q + W$ van $\mathbb{A}^{n}$ zijn \emph{zwak parrallel} als de richting van de ene deelruimte een deel is van de richting van de andere.
  \[ V \subseteq W \Leftrightarrow S \vartriangleleft T \]
\end{de}

\begin{de}
  Als twee affiene deelruimten $S$ en $T$ van $\mathbb{A}^{n}$ niet parallel, noch zwak parallel zijn, dan zijn ze \emph{snijdend} als ze niet disjunct zijn.
  \[ S \cap T \neq \emptyset \]
\end{de}

\begin{de}
  Als twee affiene deelruimten $S$ en $T$ van $\mathbb{A}^{n}$ niet parallel, noch zwak parallel zijn, dan zijn ze \emph{kruisend} als ze disjunct zijn.
  \[ S \cap T = \emptyset \]
\end{de}

\begin{st}
  Zij $S$ en $T$ parallelle affiene deelruimten van $\mathbb{A}^{n}$, dan zijn ze ofwel gelijk, ofwel disjunct.
  \[ S \parallel T \Rightarrow S=T \wedge S \cap T = \emptyset \]

  \begin{proof}
    Kies twee willekeurige parallelle affiene deelruimten $S = p + V$ en $T = q + W$ van $\mathbb{A}^{n}$.
    Stel dat de doorsnede $S \cap T$ niet leeg is, dan bestaat er een punt $r$ in zowel $S$ als in $T$.
    Er bestaat dan ook vrije vectoren $v$ en $w$ zodat $p + v = r = q + w$.\footnote{Zie de definitie van affiene deelruimten. (Definitie \ref{de:affiene-deelruimte})}.
    $r$ is dus een element van $p + V$ en bijgevolg zijn $S$ en $T$ gelijk.\footnote{Zie lemma \ref{lem:equivalenties-affiene-deelruimte}.}
  \end{proof}
\end{st}

\begin{st}
  Zij $S$ en $T$ zwak parallelle affiene deelruimten van $\mathbb{A}^{n}$ ($S \vartriangleleft  T$), dan is $S$ ofwel een deel van $T$, ofwel is de doorsnede van $S$ en $T$ leeg.

  \begin{proof}
    Kies twee willekeurige zwak parallelle affiene deelruimten $S = p + V$ en $T = q + W$ van $\mathbb{A}^{n}$ ($S \vartriangleleft  T$)
    Stel dat de doorsnede $S \cap T$ niet leeg is, dan bestaat er een punt $r$ in zowel $S$ als in $T$.
    Er geldt nu zowel $S = r + V$ als $T = r + W$.
    Omdat $S$ zwak parallel is met $T$ geldt $V \subseteq W$\footnote{Zie de definitie van `zwak parallel' (Definitie \ref{de:zwak-parallel}).} en bijgevolg nu ook $S \subseteq T$
  \end{proof}
\end{st}

\begin{st}
  Zij $S$ een affiene deelruimte van $\mathbb{A}^{n}$.
  Als $q \in \mathbb{A}^{n}$ geldt, dan bestaat er een unieke deelruimte $T$ van $\mathbb{A}^{n}$ door $q$, parallel met $S$.

  \begin{proof}
    Kies een willekeurige affiene deelruimte $S = p + V$ van $\mathbb{A}^{n}$.
    \begin{itemize}
    \item Er bestaat een deelruimte $T$ die aan bovenstaande beschrijving voldoet: $T = q + V$
    \item Deze deelruimte is uniek omdat de richting van deze deelruimte gelijk moet zijn aan $V$ opdat ze parallel zou zijn met $S$ en het aangrijpingspunt gelijk moet zijn aan $q$ opdat ze door $q$ zou gaan.
    \end{itemize}
  \end{proof}
\end{st}

\begin{st}
\label{st:affiene-doorsnede-ruimte}
  Zij $S = p + V$ en $T = q + W$ twee affiene deelruimten van $\mathbb{A}^{n}$.
  De doorsnede $U = S \cup T$ van $S$ en $T$ is ofwel leeg, ofwel ook een affiene deelruimte van $\mathbb{A}^{n}$ met richting $V \cap W$.

  \begin{proof}
    Zij $S = p + V$ en $T = q + W$ twee willekeurige affiene deelruimten van $\mathbb{A}^{n}$.
    Zij $U = S \cup T$ de doorsnede van $S$ en $T$.
    Stel dat de doorsnede $U$ niet leeg is, dan bestaat er een punt $r \in U$. Nu geldt zowel $S = r + V$ als $T = r + W$. Kies nu nog een punt $x \in U$ in de doorsnede van $S$ en $T$.
    \[
    \begin{array}{cll}
      x \in U &\Leftrightarrow \exists v \in V, \exists w \in W:& x = r + v \wedge x = r + w\\
              &\Leftrightarrow \exists v \in V, \exists w \in W:& x = r + v \wedge v = w\\
              &\Leftrightarrow \exists v \in V \cap W:& x = r + v\\
    \end{array}
    \]
  \end{proof}
\end{st}


\begin{st}
  Zij $S = p + V$ en $T = q + W$ twee affiene deelruimten van $\mathbb{A}^{n}$.
  \[ S \cap T \neq \emptyset \Leftrightarrow \overrightarrow{pq} \in V + W \]

  \begin{proof}
    Kies $S = p + V$ en $T = q + W$ twee willekeurige affiene deelruimten van een affiene ruimte $\mathbb{A}^{n}$.
    \[
    \begin{array}{cll}
      S \cap T \neq \emptyset &\Leftrightarrow \exists v \in V, \exists w \in W:& p + v = q + w\\
                              &\Leftrightarrow \exists v \in V, \exists w \in W:& p - q = w - v\\
                              &\Leftrightarrow q - p \in V + W\\
    \end{array}
    \]
  \end{proof}
\end{st}

\begin{gev}
  \label{gev:deelruimten-niet-leeg-dimensie-n}
  Zij $S = p + V$ en $T = q + W$ twee affiene deelruimten van $\mathbb{A}^{n}$ zodat $V + W = \mathbb{R}^{n}$ geldt, dan is hun doorsnede niet leeg.
\TODO{bewijs}
\end{gev}

\begin{de}
\label{de:som-van-affiene-deelruimten}
  Zij $S = p + V$ en $T = p + W$ twee affiene deelruimten van $\mathbb{A}^{n}$ door een gemeenschappelijk punt $p$, dan definieren we de \emph{som van twee affiene deelruimten} als volgt:
  \[ S + T = p + (V + W) \]
  Merk op dat er in deze gelijkheid drie verschillende optellingen gebruikt worden die met hetzelfde symbool genoteerd worden. De eerste `$+$' is de som van twee affiene deelruimten, de tweede is de som van een punt met een vectorruimte en de derde is de som van twee vectorruimten.
\end{de}

\begin{st}
  \label{st:dimensiestelling}
  De \emph{dimensiestelling}.\footnote{\'E\'en van de favoriete vragen om te bewijzen bij lineaire algebra.}\\
  Zij $S = p + V$ en $T = p + W$ twee affiene deelruimten van $\mathbb{A}^{n}$ door een gemeenschappelijk punt $p$, dan geldt het volgende over de dimensies ervan:
  \[ dim S + dim T = dim(S \cap T) + dim (S + T) \]

  \begin{proof}
    Kies $S = p + V$ en $T = p + W$ twee willekeurige affiene deelruimten van een affiene ruimte $\mathbb{A}^{n}$ door een gemeenschappelijk punt $p$.
    Blik terug op de dimensiestelling uit de lineaire algebra:
    \[ dim V + dim W = dim(V \cap W) + dim (V + W) \]
    Omdat $S$ en $T$ tenminste \'e\'en punt gemeenschappelijk hebben geldt $S \cap T = p + (V \cap W)$.\footnote{Zie stelling \ref{st:affiene-doorsnede-ruimte}.} Bovendien geldt $S + T = p + (V + W)$.\footnote{Zie de definitie van de som van twee deelruimten (Definitie \ref{de:som-van-affiene-deelruimten}).} 
    \[
    \begin{array}{rll}
       dim(S \cap T) + dim (S + T) &= dim(p + (V \cap W)) + dim(p + (V + W)) &\\
                    &= dim(V \cap W) + dim (V + W)&\\
                    &= dim V + dim W&\\
                    &= dim(p + V) + dim(p + W) &= dim S + dim T\\
    \end{array}
    \]

  \end{proof}
\end{st}

\begin{st}
  Zij $S$ en $T$ affiene deelruimten van $\mathbb{A}^{n}$.
  Er bestaat er een deelruimte $T'$ van $T$ zodat $S$ parallel is met $T'$ als en slechts als $S$ zwak parallel is met $T$.

  \begin{proof}
    Kies twee willekeurige affiene deelruimten $S = p + V$ en $T = q + W$ van $\mathbb{A}^{n}$.
    \begin{itemize}
    \item $\Rightarrow$\\
      Als er een deelruimte $T'$ bestaat van $T$ zodat $S$ parallel is met $T'$, dan heeft $T'$ als aangrijpingspunt $q$ en richting $V$.
      Omdat $T'$ een deelruimte is van $T$, is $V$ een deelruimte van $W$.
      Dat betekent precies dat $S$ zwak parallel is met $T$.
      \[ S \vartriangleleft T \]
    \item $\Leftarrow$\\
      Als $S$ zwak parallel is met $T$, geldt per definite dat $V$ een deelruimte is van $W$.
      Beschouw nu de affiene ruimte $T'$ met aangrijpingspunt $q$ en richting $V$.
      Deze affiene deelruimte is een affiene deelruimte van $T$ en $S$ is parallel met $T'$.
    \end{itemize}
  \end{proof}
\end{st}

\section{Parametervergelijkingen}
\label{sec:parametervergelijkingen}
\begin{de}
  Zij $p+V \subseteq \mathbb{A}^{n}$ een affiene deelruimte en $\{v_{1},\dotsc,v_{k}\}$ een basis van $V$.
  Een willekeurig punt $x\in \mathbb{A}^{n}$ behoort tot $p+V$ als en slechts als er $\lambda_{1},\dotsc,\lambda_{k} \in \mathbb{B}$ bestaan zodat het volgende geldt:\footnote{Zie lemma \ref{lem:equivalenties-affiene-deelruimte}}
  \[
  x = p + \lambda_1v_1 + \dotsb + \lambda_{k}v_{k}  
  \]
  We noemen deze vergelijking de \emph{parametervergelijkingen} van $p+V$.
  \[
  p+V \leftrightarrow
  \left\{
  \begin{array}{rcl}
  x_{1} &=& p_{1} + \lambda_{1}v_{11} + \dotsb + \lambda_{k}v_{k1}\\
  \vdots && \vdots\\
  x_{n} &=& p_{n} + \lambda_{1}v_{1n} + \dotsb + \lambda_{k}v_{kn}
  \end{array}
  \right.
  \]
\end{de}

\begin{de}
  Zij $p+V \subseteq \mathbb{A}^{n}$ een affiene deelruimte en $\{v_{1},\dotsc,v_{k}\}$ een basis van $V$.
  We kunnen nu de oplossingen van de parametervergelijkingen $\tilde{q} = (\lambda_{1},\dotsc,\lambda_{k})$ beschouwen as een een punt van een $k$-dimensionale affiene ruimte.
  We noemen deze ruimte de coordinaatruimte van $p+V$.
  \clarify{ Naam is zelf verzonnen. }
\end{de}

\begin{st}
  Elke affiene deelruimte $p+V \subseteq \mathbb{A}^{n}$ is de oplossingsruimte van een lineair stelsel.
  
  \begin{proof}
    Zijn $p\in\mathbb{A}^{n}$, $p+V\subseteq \mathbb{A}^{n}$ een affiene deelruimte door $p$ en $\{v_{1},\dotsc,v_{k}\}$ een basis van $V$.
    De parametervergelijkingen van $p+V$ zijn nu gegeven door de volgende gelijkheid:
    \[
    x = p + \lambda_1v_1 + \dotsb + \lambda_{k}v_{k}  
    \]
    Dit komt neer op volgend stelsel:
    \[
      p+V \leftrightarrow
      \left\{
      \begin{array}{rcl}
      x_{1} -p_{1} &=& \lambda_{1}v_{11} + \dotsb + \lambda_{k}v_{k1}\\
      \vdots && \vdots\\
      x_{n} -p_{n} &=& \lambda_{1}v_{1n} + \dotsb + \lambda_{k}v_{kn}
      \end{array}
      \right.
    \] 
    Dit is een niet-homogeen lineair stelsel in de onbekenden $\lambda_{1},\dotsc,\lambda_{k}$.
    Het punt $x$ behoort tot $p+V$ als en slechts als er getallen $\lambda_{1},\dotsc,\lambda_{k}$ bestaan zodat het stelsel geldt, of meer specifiek, als het stelsel oplosbaar is.
    De matrix van het stelsel ziet er als volgt uit:
    \[
    \begin{pmatrix}
    v_{11} & \hdots & v_{k1}\\
    \vdots & \ddots & \vdots\\
    v_{1n} & \hdots & v_{kn}\\
    \end{pmatrix}
    \] 
    De uitgebreide matrix ziet er dan als volgt uit.
    \[
    \left(
    \begin{array}{ccc|c}
    v_{11} & \hdots & v_{k1} & x_{1} - p_{1}\\
    \vdots & \ddots & \vdots & \vdots\\
    v_{1n} & \hdots & v_{kn} & x_{n} - p_{n}\\
    \end{array}
    \right)
    \]
    Het stelsel heeft nu oplossingen als en slechts als de rang van de matrix dezelfde is als de rang van de uitgebreide matrix.
    De matrix heeft rang $k$, want de kolommen (een basis) zijn onderling lineair onafhankelijk.
    Merk op dat de uitgebreide matrix $((k+1)\times n)$ als dimensies heeft, met $k \le n$.
    Opdat de rang van de uitgebreide matrix $k$ zou zijn, moet de determinant van elke $((k+1) \times (k+1))$-deelmatrix nul zijn. (anders had de uitgebreide matrix immers rang $k+1$.)
    Er zijn nu $n-k$ voorwaarden opdat alle $((k+1) \times (k+1))$-deelmatrices determinant nul hebben:
    \[
    \text{ voor }
    j = 1,\dotsc,n-k:\ 
    x \in p+V \Leftrightarrow
    \begin{vmatrix}
    x_{1}-p_{1} & \hdots & x_{k}-p_{k} & x_{k+j}-p_{k+j}\\
    v_{11}      & \hdots & v_{1k}      & v_{1 k+j}\\
    \vdots      & \ddots & \vdots      & \vdots\\
    v_{k1}      & \hdots & v_{kk}      & v_{k k+j}\\
    \end{vmatrix}
    = 0
    \]
    \clarify{ waarom? }
    We kunnen deze determinanten allemaal ontwikkelen naar de eerste rij.
    We krijgen dan de $n-k$ \emph{carthesische vergelijkingen} van $p+V$.
    \[
    p+V \leftrightarrow 
	\left\{    
    \begin{array}{cccccccccc}
    a_{1,1}(x_{1}-p_{1}) &+& \dotsb &+& a_{1,k}(x_{k}-p_{k}) &+& a_{1,k+1}(x_{k+1}-p_{k+1}) &=& 0\\
    a_{2,1}(x_{1}-p_{1}) &+& \dotsb &+& a_{2,k}(x_{k}-p_{k}) &+& a_{2,k+2}(x_{k+2}-p_{k+2}) &=& 0\\
    \vdots && \vdots && \vdots && \vdots && \vdots\\
    a_{n+k,1}(x_{1}-p_{1}) &+& \dotsb &+& a_{n-k,k}(x_{k}-p_{k}) &+& a_{n-k,n}(x_{n}-p_{n}) &=& 0
    \end{array}
    \right.     
    \]
    Deze vergelijkingen zijn bovendien lineair onafhankelijk.
    \clarify{ waarom? }
	Tenslotte kennen we dan ook een stelsel voor $V$.     
    \[
    V \leftrightarrow 
	\left\{    
    \begin{array}{cccccccccc}
    a_{1,1}x_{1} &+& \dotsb &+& a_{1,k}x_{k} &+& a_{1,k+1}x_{k+1} &=& 0\\
    a_{2,1}x_{1} &+& \dotsb &+& a_{2,k}x_{k} &+& a_{2,k+2}x_{k+2} &=& 0\\
    \vdots && \vdots && \vdots && \vdots && \vdots\\
    a_{n+k,1}x_{1} &+& \dotsb &+& a_{n-k,k}x_{k} &+& a_{n-k,n}x_{n} &=& 0
    \end{array}
    \right.     
    \]
    Elke $k$-dimensionale affiene deelruimte is dus de oplossingsverzameling van een lineair stelsel van $n-k$ lineaire vergelijkingen.
  \end{proof}
\end{st}

\section{Affiene hypervlakken}
\label{sec:affiene-hypervlakken}
\begin{st}
  \label{st:vergelijking-affien-hypervlak}
  We hebben slechts \'e\'en vergelijking nodig op een hypervlak te beschrijven.
  
  \begin{proof}
    Inderdaad, er is maar \'e\'en $((k+1)\times(k+1))$-dimensionale deelmatrix van de uitgebreide matrix die $p+V$ bepaalt:
    \[
    H \leftrightarrow
    \begin{vmatrix}
    x_{1}-p_{1}   & x_{2}-p_{2}  & \hdots & x_{n}-p_{n}\\
    v_{1,1}       & v_{1,2}      & \hdots & v_{1,n}\\
    \vdots        & \vdots       & \ddots & \vdots\\
    v_{n-1,1}     & v_{n-1,2}    & \hdots & v_{n-1,n} 
    \end{vmatrix}
    = 0
    \]
    De vergelijking van $H$ is dan van de volgende vorm:
    \[
    a_{1}(x_{1}-p_{1}) + a_{2}(x_{2}-p_{2}) + \dotsb + a_{n}(x_{n}-p_{n}) = 0
    \]
    Het $n$-tal $(a_{1},\dotsc,a_{n})$ is bovendien uniek bepaald, op de evenredigheidsfactor na, en wordt het \emph{richtingsgetal} van $H$ genoemd.
    \clarify{ waarom? }
    \clarify{ wat is de evenredigheidsfactor? }
  \end{proof}
\end{st}

\begin{st}
  \label{st:parallelle-hypervlakken-gelijke-richtingsgetallen}
  Twee hypervlakken zijn parallel als en slechts als hun richtingsgetallen evenredig zijn.
  \begin{proof}
  Zij $H$ een hypervlak met de volgende vergelijking:
  \[
    a_{1}(x_{1}-p_{1}) + a_{2}(x_{2}-p_{2}) + \dotsb + a_{n}(x_{n}-p_{n}) = 0
  \]
  De richting van het hypervlak is de oplossingsverzameling van de volgende vergelijking:
  \[
    a_{1}x_{1} + a_{2}x_{2} + \dotsb + a_{n}x_{n} = 0
  \]
  Parallelle hypervlakken hebben dezelfde richting\footnote{Zie definitie \ref{de:parallel}.}, dus hun richtingsgetallen moeten evenredig zijn.
  \end{proof}
\end{st}

\section{Vergelijkingen van affiene deelruimten en hun onderlinge stand}
\label{sec:vergelijkingen-van-affiene-deelruimten}

\section{Rechten in $\mathbb{A}^{n}$}
\begin{de}
  \label{de:rechte}
  Zijn $p$ een punt van $\mathbb{A}^{n}$, en $v\in \mathbb{R}^{n}$ een niet-nulvector, dan is $L$ de \emph{rechte} door $p$ in de richting van $<v>$.
  \[
  L = p + <v>
  \]
  Deze rechte heeft de volgende vergelijking als parametervergelijking.
  \[
  L \leftrightarrow x = p + \lambda v
  \]
  In co\"ordinaten:
  \[
  L \leftrightarrow \forall i: x_{i} = p_{i} + \lambda v_{i}
  \]
  Stel nu dat $v_{1}$ niet nul is, dan geldt $\lambda = \frac{x_{1}-p_{1}}{v_{1}}$ en zijn dit dus de carthesische vergelijkingen van $L$:
  \[
  v_{1}(x_{i}-p_{i}) = v_{i}(x_{1}-p_{1})
  \]
\end{de}

\begin{st}
  De rechte $L=pq$ door twee punten $p$ en $q$ is precies $p + <\vec{pq}>$.
  De carthesische vergelijkingen van $pq$ zijn dan de volgenden:
  \[
  (q_{1}-p_{1})(x_{i}-p_{i}) = (q_{j}-p_{j})(x_{1}-p_{1})
  \]
\question{Wat doen die $j$s hier?}

\TODO{bewijs}
\end{st}


\begin{st}
  \label{st:unieke-rechte-door-twee-punten}
  Door twee punten $p$ en $q$ van een affiene ruimte $\mathbb{A}^{n}$ gaat een unieke rechte.

  \begin{proof}
    Er gaat een rechte door $p$ en $q$ met $p$ als aangrijpingspunt en de vectorruimte met $\overleftarrow{pq}$ als basis als richting.
    Deze rechte is bovendien uniek.
\clarify{ waarom is deze rechte uniek? Intu\"itief is het duidelijk, maar hoe tonen we het aan? }
  \end{proof}
\end{st}

\begin{de}
  De \emph{barycentrische co\"ordinaten} van een punt $x$ ten opzichte van twee punten $p$ en $q$ zijn de $\lambda_{1}$ en $\lambda_{2}$ die aan volgende gelijkheid voldoen.
  \[ x = \lambda_{1}p + \lambda_{2}q \text{ met } \lambda_{1} + \lambda_{2} = 1 \]
\end{de}

\begin{de}
  Het \emph{barycentrum} of het midden van twee punten is het punt met als barycentrische co\"ordinaten $(\frac{1}{2},\frac{1}{2})$.
\end{de}

\begin{st}
  Zij $L$ de rechte bepaald door twee punten $p$ en $q$ in $\mathbb{A}^{n}$.
  Voor elke punt $x$ van de rechte $L$ geldt dat $x$ kan geschreven worden als een affiene combinatie is van $p$ en $q$.
  \[ x \in L \Leftrightarrow \exists \lambda_{1},\lambda_{2} \in \mathbb{R}: x = \lambda_{1}p + \lambda_{2}q \text{ met } \lambda_{1} + \lambda_{2} = 1 \]
  
  \begin{proof}
    \begin{itemize}
    \item $\Rightarrow$\\
      Als $x$ op de rechte $L$ ligt, dan bestaat er een $\lambda$ zodat het volgende geldt:\footnote{Zie definitie \ref{de:rechte}.}
      \[ x = p + \lambda(p - q) = (1 - \lambda)p + \lambda q \]
      Kies nu $\lambda_{1} = 1-\lambda$ en $\lambda_{2} = \lambda$.
      \[ \lambda_{1} + \lambda_{2} = (1-\lambda) + \lambda = 1 \]
    \item $\Leftarrow$\\
      Als $x = \lambda_{1}p + \lambda_{2}q$ geldt met $\lambda_{1} + \lambda_{2} = 1$, kunnen we $\lambda_{1}$ in functie van $\lambda_{2}$ schrijven.
      \[ x = (1-\lambda_{2})p + \lambda_{2}q = p + \lambda(q-p)\]
      Dit komt overeen met de parametervergelijking van $L$, dus $x$ ligt op $L$.
    \end{itemize}
  \end{proof}
\end{st}

\begin{de}
  Drie unten $p$, $q$ en $r$ van $\mathbb{A}^{n}$ zijn colineair als ze op dezelfde rechte liggen.
\end{de}

\begin{st}
  Er bestaan drie niet-colineaire punten $p$, $q$ en $r$ in een affien vlak $\mathcal{V}$.

\TODO{bewijs}
\end{st}

\begin{de}
  De \emph{deelverhouding} van drie colineaire punten $p$, $q$ en $r$ is een getal genoteerd als $(p,q,r)$ dat voldoet aan volgende gelijkheid.
  \[ \vec{pr} = (p,q,r)\vec{pq} \]
  We gebruiken ook wel de volgende notatie:
  \[ (p,q,r) = \frac{\vec{pr}}{\vec{pq}} = \frac{r-p}{q-p} = \frac{p-r}{p-q} \]
  Dit zogenaamde ``delen door vectoren'' is enkel gedefinieerd voor colineaire punten.
\end{de}


\begin{st}
  \label{st:stelling-van-thales}
  \emph{Stelling van Thales}\\
  Zij $H_{1}$, $H_{2}$ en $H_{3}$ drie parallelle hypervlakken van $\mathbb{A}_{n}$
  \[
  \begin{array}{rl}
    H_{1} = p_{1} + V\\
    H_{2} = p_{2} + V\\
    H_{3} = p_{3} + V\\
  \end{array}
  \]
  Zij $L$ een rechte, niet zwak parallel met $H_{1}$.
  \[ L = p + W \text{ met } W \not\subsetneq V \]
  Zij $d_{i}$ de doorsnede van $p$ met $H_{i}$.
  De deelverhouding $(d_1,d_2,d_3)$ hangt niet af van $L$, enkel van de $H_{i}$.
 
  \begin{proof}
    De doorsneden van $L$ met de $H_{i}$ zijn niet leeg.\footnote{Zie gevolg \ref{gev:deelruimten-niet-leeg-dimensie-n}.}
    \[ \L \cap H_{i} \neq \emptyset \]
    Uit de dimensiestelling\footnote{Zie stelling \ref{st:dimensiestelling}.} volgt dat de doorsneden $L\cap H_{i}$ punten zijn, dus het is zinvol om van $d_{i}$ te spreken.
    \[
    \begin{array}{cccccccc}
    dim L &+& dim H_{i} &=& dim(L \cap H_{i}) &+& dim(L + H_{i})\\
    2     &+& (n-1)    &=& dim(L \cap H_{i}) &+& n\\
          & & 1        &=& dim(L \cap H_{i})
    \end{array}
    \] 
    We beschouwen nu twee willekeurige rechten $L$ en $L'$ en bewijzen dat $(d_1,d_2,d_3)$ gelijk is aan $(d_1',d_2',d_3')$.
    \[ d_{i}' = L' \cap H_{i} \]
    De $H_{i}$ zijn parallel dus hun richtingsgetallen zijn gelijk.\footnote{Zie stelling \ref{st:parallelle-hypervlakken-gelijke-richtingsgetallen}.}
    De $H_{i}$ hebben dus de volgende carthesische vergelijkingen.\footnote{Zie stelling \ref{st:vergelijking-affien-hypervlak}.}
    \[ H_{i} \leftrightarrow \sum_{j=1}^{n}a_{j}(x_{j}-(d_{i})_{j}) = 0 \]
    De punten $d_{i}'$ zitten respectievelijk in $H_{i}$, dus die voldoen ook aan bovenstaande vergelijkingen voor $x$.
    \[ \sum_{j=1}^{n}a_{j}((d_{i}')_{j}-(d_{i})_{j}) = 0 \]
    \[ \Rightarrow \sum_{j=1}^{n}a_{j}(d_{i}')_{j} = \sum_{j=1}^{n}a_{j}(d_{i})_{j}\]
    Noem nu de deelverhoudingen $(d_1,d_2,d_3)$ en $(d_1',d_2',d_3')$ respectievelijk $\lambda$ en $\lambda'$, dan gelden volgende gelijkheden.;
    \[ \lambda = \frac{d_{3}-d_{1}}{d_{2}-d_{1}} \text{ en } \lambda' = \frac{d_{3}'-d_{1}'}{d_{2}'-d_{1}'} \]
    \[ \Rightarrow d_{3}-d_{1} = \lambda {d_{2}-d_{1}} \text{ en } d_{3}'-d_{1}' = \lambda' d_{2}'-d_{1}' \]
    We gebruiken dit nu om $\lambda_{1} = \lambda_{2}$ te bewijzen.
    \[
    \begin{array}{rl}
      0 &= 0 - 0\\
        &= \sum_{j=1}^{n}a_{j}((d_{3}')_{j}-(d_{3})_{j}) - \sum_{j=1}^{n}a_{j}((d_{1}')_{j}-(d_{1})_{j}) \\
        &= \sum_{j=1}^{n}a_{j}(((d_{3}')_{j}-(d_{3})_{j}) - ((d_{1}')_{j}-(d_{1})_{j})) \\
        &= \sum_{j=1}^{n}a_{j}((d_{3}' - d_{1}')_{j} - (d_{3} -d_{1})_{j}) \\
        &= \sum_{j=1}^{n}a_{j}(\lambda'(d_{2}' - d_{1}')_{j} - \lambda(d_{2} -d_{1})_{j}) \\
        &= \lambda'\sum_{j=1}^{n}a_{j}(d_{2}' - d_{1}')_{j} - \lambda\sum_{j=1}^{n}a_{j}(d_{2} -d_{1})_{j} \\
        &= \lambda'\sum_{j=1}^{n}a_{j}(d_{2} - d_{1})_{j} - \lambda\sum_{j=1}^{n}a_{j}(d_{2} -d_{1})_{j} \\
        &= (\lambda' -\lambda)\sum_{j=1}^{n}a_{j}(d_{2} -d_{1})_{j} \\
    \end{array}
    \]
    $L$ is niet zwak parallel met $H_{i}$ en snijdt ($L \cap H_{i} \neq \emptyset$) ligt $d_{2}$ niet in $H_{1}$.
    Bijgevolg geldt de volgende gelijkheid.
    \[ \sum_{j=1}^{n}a_{j}(d_{2} -d_{1})_{j} \neq 0 \]
    $\lambda$ moet dus gelijk zijn aan $\lambda'$.
  \end{proof}
\end{st}

\section{Barycentrische co\"ordinaten}
\label{sec:baryc-coordinaten}

\begin{de}
  \label{de:affien-afhankelijk}
  We noemen $k+1$ punten van een affiene ruimte $\mathbb{A}^{n}$ affien afhankelijk als en slechts als ze in een $l$-dimensionale affiene deelruimte van $\mathbb{A}^{n}$ liggen met een dimensie strikt kleiner dan $k$.
\end{de}

\begin{st}
  \label{st:critirium-affien-afhankelijk}
  \emph{Criterium voor affiene afhankelijkheid}\\
  De $k+1$ punten $p_{0},\dotsc,p_{k}$ van een $n$-dimensionale affiene ruimte $\mathbb{A}^{n}$ zijn affien afhankelijk als en slechts als de vrije vectoren $\vec{p_{0}p_{i}}$ met $i \in \{1,\dotsc,k\}$ lineair afhankelijk zijn.

  \begin{proof}
    Zij $V$ de vectorruimte opgespannen door de vrije vectoren $\overrightarrow{p_{0}p_{i}}$, dan liggen de $k+1$ punten $p_{0},\dotsc,p_{k}$ allemaal in $p_{0} + V$.
    \[ V = <\overrightarrow{p_{0}p_{1}},\dotsc,\overrightarrow{p_{0}p_{k}}> \]
    \[ \forall i:\ p_{i} \in p_{0}+V \]
    Nu geldt dat de dimensie van $p_{0} + V$ kleiner of gelijk is aan $k$ als er slechts als de vrije vectoren lineair afhankelijk zijn.
    \begin{itemize}
    \item $\Rightarrow$\\
      Als de punten $p_{0},\dotsc,p_{k}$ affien onafhankelijk zijn dan is de dimensie van de affiene deelruimte waarin ze zich bevinden van dimensie groter of gelijk aan $k$.\footnote{Zie definitie \ref{de:affien-afhankelijk}.}
      De vrije vectoren $\overrightarrow{p_{0}p_{1}},\dotsc,\overrightarrow{p_{0}p_{k}}$ zijn dan zeker lineair onafhankelijk.
      \clarify{waarom?}
    \item $\Leftarrow$\\
      Stel dat de vrije vectoren $\overrightarrow{p_{0}p_{1}},\dotsc,\overrightarrow{p_{0}p_{k}}$ lineair onafhankelijk zijn, dan is de dimensie van $V$ gelijk aan $k$.
      Als dan $S = p_{0}+W$ een affiene deelruimte is van dimensie $l$ zodat alle punter $p_{i}$ erin liggen, dan liggen alle vrije vectoren $\overrightarrow{p_{0}p_{1}},\dotsc,\overrightarrow{p_{0}p_{k}}$ ook is $W$. $V$ moet dan een deelverzameling zijn van $W$. Bijgevolg is de dimensie van $W$ groter of gelijk aan die van $V$.
      \[ dim W \ge dim V = k \]
      De punten kunnen dus nooit in een affiene deelruimte liggen met een dimensie die strikt kleiner is dan $k$.
      Ze zijn dus affien onafhankelijk.
    \end{itemize}
  \end{proof}
\end{st}

\begin{gev}
  $k+1$ affien onafhankelijke punten $p_{0},\dotsc,p_{k}$ van $\mathbb{A}^{n}$ bepalen een $k$-dimensionale affiene deelruimte $D$ van $\mathbb{A}^{n}$.
  \[ D = p_{0} + \overrightarrow{p_{0}p_{1}},\dotsc,\overrightarrow{p_{0}p_{k}}> \]
  We noemen dit de affiene deelruimte bepaald door de punten $p_{0},\dotsc,p_{k}$.

\TODO{bewijs}
\end{gev}

\begin{de}
  De \emph{barycentrische co\"ordinaten} van een punt $x$ ten opzichte van $k+1$ punten $p_{0},\dotsc,p_{k}$ zijn de $\lambda_{0},\dotsc,\lambda_{k} \in \mathbb{R}$ die aan volgende gelijkheid voldoen.
  \[ x = \sum_{i=0}^{k}\lambda k_{i}p_{i} \text{ met } \sum_{i=0}^{n}\lambda_{i} = 1 \]
\end{de}

\begin{de}
  Het \emph{barycentrum} of het midden van $k+1$ punten is het punt met als barycentrische co\"ordinaten $(\frac{1}{k+1},\dotsc,\frac{1}{k+1})$.
\end{de}


\begin{st}
  Zij $S$ de affiene deelruimte van $\mathbb{A}^{n}$ bepaald door de affien onafhankelijk. punten $p_{0},\dotsc,p_{k}$ en zij $x$ nog een punt in $\mathbb{A}^{n}$.
  $x$ zit in $S$ als en slechts als er $\lambda_{0},\dotsc,\lambda_{k} \in \mathbb{R}$ bestaan zodat volgende gelijkheid geldt:
  \[ x = \sum_{i=0}^{k}\lambda k_{i} \text{ met } \sum_{i=0}^{n}\lambda_{i} = 1\]

  \begin{proof}
    Een punt $x$ ligt in $S$ als en slechts als er $\mu_{i}$ bestaan zodat de volgende gelijkheid geldt.
    \[ x = p_{0} + \sum_{i=1}^{n}\mu_{i}(p_{i}-p_{0}) = \left(1 - \sum_{i=1}^{n}\mu_{i}\right)p_{0} + \sum_{i=1}^{n}p_{i} \]
    Kies tenslotte de $\lambda_{i}$ als volgt:
    \[ \lambda_{0} = \left(1 - \sum_{i=1}^{n}\mu_{i}\right) \]
    \[ \lambda_{i} = \mu_{i} \text{ als } i \neq 0 \]
    De som van de $\lambda_{i}$ is nu wel degelijk $1$.
  \end{proof}
\end{st}


\begin{de}
  De parametervergelijkingen van de affiene deelruimte $S$ van $\mathbb{A}^{n}$ bepaald door $k+1$ affien onafhankelijke punten $p_{0},\dotsc,p_{k}$ wordt gegeven door volgende gelijkheid.
  \[ S \leftrightarrow x = p_{0} + \sum_{i=1}^{n}\mu_{i}(p_{i}-p_{0}) \]
\end{de}

\end{document}
