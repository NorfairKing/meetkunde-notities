\documentclass[main.tex]{subfiles}
\begin{document}

\chapter{Affiene meetkunde}
\label{cha:affiene-meetkunde}

\section{Affiene ruimte}
\label{sec:affiene-ruimte}

\begin{de}
  Een $n$-dimensionaal punt is een veeltal met $n$ co\"ordinaten.
  \[ p = (p_{1}, p_{2}, \ldots, p_{n}) \]
\end{de}

\begin{de}
  Een $n$-dimensionale \emph{affiene ruimte} $\mathbb{A}^{n}$ bestaat uit $n$-dimensionale punten.
  \[ \mathbb{A}^{n} = \left\{\{ (p_{1}, p_{2}, \ldots, p_{n}) \ |\ p_{i} \in \mathbb{R} \right\} \]
\end{de}

\begin{de}
  $\mathbb{A}^{2}$ noemen we \emph{het affiene vlak}.
\end{de}

\begin{de}
  Zij $p \in \mathbb{A}^{n}$ een punt van de $n$-dimensionale affiene ruimte $\mathbb{A}^{n}$ en zij $v \in \mathbb{R}^{n}$ een $n$-dimensionale re\"ele vector.
  Een koppel $(p,v)$ noemen we een \emph{raakevector} met aangrijpingspunt $p$ en vectordeel $v$.
  \[ v_{p} = (p,v) \]
\end{de}

\begin{de}
  Twee raakvectoren $v_{p} = (p,v)$ en $w_{q} = (q,w)$ zijn gelijk als elk zowel hun aangrijpingspunten en vectordelen gelijk zijn.
  \[ v_{p} = w_{q} \Leftrightarrow p = q \wedge v = w \]
\end{de}

\begin{de}
  De \emph{rakende ruimte} $T_{p}\mathbb{A}^{n}$ \emph{in een punt} $p$ \emph{aan een affiene ruimte} $\mathbb{A}^{n}$ is de verzameling van raakvectoren met $p$ als aangrijpingspunt in $\mathbb{A}^{n}$.
  \[ \left\{\{v_{p} = (p,v)\ |\ v \in \mathbb{R}^{n} \right\} \]
\end{de}

\begin{de}
  De verzameling van alle raakvectoren $v_{p}$ aan punten $p$ in een affiene ruimte $\mathbb{A}^{n}$ noemen we de \emph{rakende bundel} m$T\mathbb{A}^{n}$ van die affiene ruimte $\mathbb{A}^{n}$.
  \[
  \begin{array}{rl}
  T\mathbb{A}^{n} &= \left\{v_{p}\ |\ p \in \mathbb{A}^{n}, v \in \mathbb{R}^{n} \right\}\\
                 &= \left\{(p,v)\ |\ p \in\mathbb{A}^{n}, v \in \mathbb{R}^{n} \right\}\\
                 &= \mathbb{A}^{n} \times \mathbb{R}^{n}
  \end{array}
  \]
\end{de}

\begin{de}
  Zij $v_{p}$ en $w_{p}$ twee raakvectoren in de rakende ruimte $T_{p}\mathbb{A}^{n}$ van hetzelfde punt $p$, dan defini\"eren we de som $v_{p}+w_{p}$ als volgt.
  \[ v_{p}+w_{p} = (v+w)_{p}\]
\end{de}

\begin{de}
  Zij $v_{p}$ een raakvector aan een punt $p$ van de affiene ruimte $\mathbb{A}^{n}$ en $\lambda \in \mathbb{R}$ een re\"eel getal, dan defini\"eren we het scalair product $\lambda v_{p}$ als volgt. 
  \[ \lambda v_{p} = (\lambda v)_{p}\]
\end{de}

\begin{st}
  Elke rakende ruimte in een punt $p$ aan $\mathbb{A}^{n}$ vormt een vectorruimte over $\mathbb{R}$.
  \TODO{bewijs zie p 6 voor opgave.}
  \begin{proof} We bewijzen de axioma's van een vectorruimte
    \begin{itemize}
    \item associativiteit van de optelling
    \item neutraal element voor de optelling
    \item symmetrisch element voor de optelling
    \item commutativiteit van de optelling
    \item distributiviteit van het scalair product ten opzichte van de optelling
    \item distributiviteit van de optelling ten opzichte van het scalair product
    \item gemengde associativiteit
    \item neutraal element van het scalair product
    \end{itemize}
  \end{proof}
\end{st}

\begin{st}
  \label{st:phi-isomorphisme}
  Voor elk willekeurig punt $p$ van de affiene ruimte $\mathbb{A}^{n}$ is de afbeelding van de raakvector op de vector een isomorfisme van de rakende ruimte in dat punt en de re\"ele vectorruimte $\mathbb{R}^{n}$.
  $phi_{p}$ is dus een isomorphisme.
  \[ \phi_{p}: T_{p}\mathbb{A}^{n} \rightarrow \mathbb{R}^{n}: v_{p} \mapsto v \]
  \begin{proof}
    Een isomorphisme is een bijectieve lineaire afbeelding.
    \begin{itemize}
    \item $\phi_{p}$ is een bijectie.
      \begin{itemize}
      \item $\phi_{p}$ is een injectie.
      \[ \forall v_{p},w_{p} \in T_{p}\mathbb{A}^{n}: \phi_{p}(v_{p}) = \phi_{p}(w_{p}) \Rightarrow v_{p} = w_{p}\]
      \item $\phi_{p}$ is een surjectie.
      \[ \forall v \in \mathbb{R}^{n},\ \exists v_{p} \in T_{p}\mathbb{A}^{n}: \phi_{p}(v_{p}) = v \]
      \end{itemize}
    \item $\phi_{p}$ bewaart de lineariteit:
    \[ \phi_{p}(v_{p}+w_{p}) = \phi_{p}((v+w)_{p}) = v + w = \phi_{p}(v_{p}) + \phi_{p}(w_{p}) \] 
    \[ \phi_{p}(\lambda v_{p}) = \phi_{p}(\lambda v)_{p}) = \lambda v = \lambda\phi_{p}(v_{p})\]
    \end{itemize}
  \end{proof}
\end{st}

\begin{st}
  Voor elke twee willekeurige punten $p$ en $q$ van de affine ruimte $\mathbb{A}^{n}$ zijn de rakende ruimten isomorf.
  $\psi$ is dus een isomorphisme.
  \[ \psi_{pq}: T_{p}\mathbb{A}^{n} \rightarrow T_{q}\mathbb{A}^{n}: v_{p} \mapsto v_{q}\]
  \begin{proof}
    $\psi_{pq}$ is een samenstelling van isomorphismen\footnote{Zie het isomorphisme $\phi_{p}$ (stelling \ref{st:phi-isomorphisme})}, en bijgevolg ook een isomorphisme.
    \[ \psi_{pq} = \phi_{p}^{-1} \circ\phi_{p}\]
  \end{proof}
\end{st}

\end{document}
