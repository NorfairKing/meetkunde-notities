\documentclass[main.tex]{subfiles}
\begin{document}

\chapter{Examens}
\label{cha:examens}
In dit hoofstuk worden oude examens opgelost.
Er wordt telkens eerst de opgave gegeven en daarna de oplossing.
Merk op dat de opgaven gratis online beschikbaar zijn.
Let ook op, de examenvragen komen hoogstwaarschijnlijk niet terug, maar het zijn goede oefeningen.


\includepdf[pages=-]{opgaven/januari_2012_1.pdf}

\section{Januari 2012 I}

\subsection*{Mondeling gedeelte}
\begin{enumerate}
\item
  \begin{enumerate}[(a)]
  \item \TODO{}
  \item \TODO{}
  \item \TODO{}
  \end{enumerate}
\item \TODO{}
\end{enumerate}
\subsection*{Schriftelijk gedeelte}
\begin{enumerate}
\item
  \begin{enumerate}[(a)]
  \item 
    \[
    \begin{pmatrix}
      2\\0\\1\\1
    \end{pmatrix}
    + \lambda
    \begin{pmatrix}
      1\\2\\0\\2
    \end{pmatrix}
    =
    \begin{pmatrix}
      -1\\0\\4\\-5
    \end{pmatrix}
    + \mu
    \begin{pmatrix}
      2\\0\\1\\2
    \end{pmatrix}
    \longleftrightarrow
    \begin{pmatrix}
      3\\0\\-3\\-4
    \end{pmatrix}
    =
    \lambda
    \begin{pmatrix}
      -1\\-2\\-0\\-2
    \end{pmatrix}
    + \mu
    \begin{pmatrix}
      2\\0\\1\\2
    \end{pmatrix}
    \]
    \[
    \left(
      \begin{array}{cc|c}
        -1 & 2 & 3\\
        2 & 0 & 0\\
        0 & 1 & -3\\
        -2 & 2 & -4
      \end{array}
    \right)
    \]
    Dit stelsel is strijdig, dus de doorsnede van $L_{1}$ en $L_{2}$ is leeg.
    $(1,2,0,2)$ is bovendien lineair onafhankelijk van $(2,0,1,2)$, dus de rechten kruisen.
  \item 
    We kennen de richting van de gemeenschappelijke loodlijn kennen we nu al, we berekenen dus nog het snijpunt ervan met $L_{1}$.
    \[ 
    \begin{pmatrix}
      2\\0\\1\\1
    \end{pmatrix}
    + a
    \begin{pmatrix}
      1\\2\\0\\2
    \end{pmatrix}
    =
    \begin{pmatrix}
      -1\\0\\4\\-5
    \end{pmatrix}
    + b
    \begin{pmatrix}
      2\\0\\1\\2
    \end{pmatrix}
    + c
    \begin{pmatrix}
      0\\1\\2\\-1
    \end{pmatrix}
    \longleftrightarrow
    \begin{pmatrix}
      2 & 0 & 1 & 3\\
      0 & 1 & 2 & 0\\
      1 & 2 & 0 & -3\\
      2 & -1 & 2 & 6
    \end{pmatrix}
    \]
    De oplossing hiervan is $(a,b,c)=(1,-2,1)$.
    Het snijpunt van $L_{1}$ met de gemeenschappelijke loodlijn is dus $p$:
    \[
    \begin{pmatrix}
      2\\0\\1\\1
    \end{pmatrix}
    +
    \begin{pmatrix}
      1\\2\\0\\2
    \end{pmatrix}
    =
    \begin{pmatrix}
      3\\2\\1\\3
    \end{pmatrix}
    \]
    De gemeenschappelijke loodlijn is dan $L_{3}$:
    \[
    L_{3} =
    \begin{pmatrix}
      3\\2\\1\\3
    \end{pmatrix}
    +
    \lambda
    \begin{pmatrix}
      0\\1\\2\\-1
    \end{pmatrix}
    \]
  \item Nee, de richtingsvector van de gemeenschappelijke loodlijn moet van een zeer specifieke vorm zijn.
    % Elke vector op de gemeenschappelijke loodlijn van $S$ en $T$ staat loodrecht op zowel $(1,2,0,2)$ als $(2,0,1,2)$:
    % \[
    % \left\{
    %   \begin{array}{c}
    %     (1,2,0,2) \cdot (u,x,y,z) = 0\\
    %     (2,0,1,2) \cdot (u,x,y,z) = 0
    %   \end{array}
    % \right.
    % \longleftrightarrow
    % \left\{
    %   \begin{array}{c}
    %     u+2x+2z = 0\\
    %     2u+y+z = 0
    %   \end{array}
    % \right.
    % \longleftrightarrow
    % \begin{pmatrix}
    %   1 & 0 & \frac{1}{2} & 1\\
    %   0 & 1 & -\frac{1}{2} & 1
    % \end{pmatrix}
    % \]
  \end{enumerate}

\item 
  \begin{figure}[H]
    \centering
    \begin{tikzpicture}[scale=1,extended line/.style={shorten >=-#1,shorten <=-#1},extended line/.default=1cm] 
      \coordinate [label=left:$l$] (l1) at (-1,0);
      \coordinate (l2) at (1,3);
      \coordinate [label=right:$m$] (m1) at (2,1);
      \coordinate (m2) at (0,4);
      \draw [extended line=1cm] (l1) -- (l2);
      \draw [extended line=1cm] (m1) -- (m2);

      \coordinate [label=above:$M$] (m) at (1,1);
      \fill (m) circle [radius=1pt];

    \end{tikzpicture}
  \end{figure}
  \begin{itemize}
  \item Syntetisch
    We zoeken een punt op $l$ en een punt op $m$ zodat $|AM|$ gelijk is aan $|MB|$.
  \item \TODO{}
  \item Analytisch
  \item \TODO{}
  \end{itemize}
\item \TODO{}
\item \TODO{}
\end{enumerate}



\section{Januari 2012 II}

\includepdf[pages=-]{opgaven/januari_2012_2.pdf}






\includepdf[pages=-]{opgaven/januari_2011_1.pdf}

\includepdf[pages=-]{opgaven/januari_2011_2.pdf}

\includepdf[pages=-]{opgaven/augustus_2011.pdf}




\includepdf[pages=-]{opgaven/januari_2009_2.pdf}

\end{document}
