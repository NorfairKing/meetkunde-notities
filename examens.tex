\documentclass[main.tex]{subfiles}
\begin{document}

\chapter{Examens}
\label{cha:examens}
In dit hoofstuk worden oude examens opgelost.
Er wordt telkens eerst de opgave gegeven en daarna de oplossing.
Merk op dat de opgaven gratis online beschikbaar zijn.
Let ook op, de examenvragen komen hoogstwaarschijnlijk niet terug, maar het zijn goede oefeningen.


\includepdf[pages=-]{opgaven/januari_2012_1.pdf}

\section{Januari 2012 I}

\subsection{Mondeling gedeelte}
\begin{enumerate}
\item
  \begin{enumerate}[(a)]
  \item \TODO{}
  \item \TODO{}
  \item \TODO{}
  \end{enumerate}
\item \TODO{}
\end{enumerate}
\subsection{Schriftelijk gedeelte}
\begin{enumerate}
\item
  \begin{enumerate}[(a)]
  \item 
    \[
    \begin{pmatrix}
      2\\0\\1\\1
    \end{pmatrix}
    + \lambda
    \begin{pmatrix}
      1\\2\\0\\2
    \end{pmatrix}
    =
    \begin{pmatrix}
      -1\\0\\4\\5
    \end{pmatrix}
    + \mu
    \begin{pmatrix}
      2\\0\\1\\2
    \end{pmatrix}
    \longleftrightarrow
    \begin{pmatrix}
      3\\0\\-3\\6
    \end{pmatrix}
    =
    \lambda
    \begin{pmatrix}
      -1\\-2\\-0\\-2
    \end{pmatrix}
    + \mu
    \begin{pmatrix}
      2\\0\\1\\2
    \end{pmatrix}
    \]
    \[
    \left(
      \begin{array}{cc|c}
        -1 & 2 & 3\\
        2 & 0 & 0\\
        0 & 1 & -3\\
        -2 & 2 & 6
      \end{array}
    \right)
    \]
    Hiereen is eenvoudig te zijn dat $\lambda$ enkel $0$ kan zijn en $\mu$ enkel $-\frac{1}{3}$.
    De doorsnede van de twee rechten is dus als volgt:
    \[
    S \cap T =
    \left\{
      \begin{pmatrix}
        2 \\0\\1\\1
      \end{pmatrix}
    \right\}
    \]
  \end{enumerate}

\item 
  \begin{figure}[H]
    \centering
    \begin{tikzpicture}[scale=1,extended line/.style={shorten >=-#1,shorten <=-#1},extended line/.default=1cm] 
      \coordinate [label=left:$l$] (l1) at (-1,0);
      \coordinate (l2) at (1,3);
      \coordinate [label=right:$m$] (m1) at (2,1);
      \coordinate (m2) at (0,4);
      \draw [extended line=1cm] (l1) -- (l2);
      \draw [extended line=1cm] (m1) -- (m2);

      \coordinate [label=above:$M$] (m) at (1,1);
      \fill (m) circle [radius=1pt];

    \end{tikzpicture}
    \caption{oefening 5, een illustratie}
  \end{figure}
  \begin{itemize}
  \item Syntetisch
    We zoeken een punt op $l$ en een punt op $m$ zodat $|AM|$ gelijk is aan $|MB|$.
  \item \TODO{}
  \item Analytisch
  \item \TODO{}
  \end{itemize}
\item \TODO{}
\item \TODO{}
\end{enumerate}



\section{Januari 2012 II}

\includepdf[pages=-]{opgaven/januari_2012_2.pdf}






\includepdf[pages=-]{opgaven/januari_2011_1.pdf}

\includepdf[pages=-]{opgaven/januari_2011_2.pdf}

\includepdf[pages=-]{opgaven/augustus_2011.pdf}




\includepdf[pages=-]{opgaven/januari_2009_2.pdf}

\end{document}
