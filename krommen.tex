\documentclass[main.tex]{subfiles}
\begin{document}

\chapter{Algemene theorie van krommen in $\mathbb{E}^{n}$}
\label{cha:algemene-theorie-van}

\section{Krommen en vectorvelden langs krommen}
\label{sec:kromm-en-vect}

\begin{de}
  Een \term{kromme} in de Euclidische ruimte $\mathbb{E}^{n}$ is een afbeelding van de vorm $\alpha$...
  \[ \alpha:\ I \subseteq \mathbb{R} \rightarrow \mathbb{E}^{n}:\ t \mapsto \alpha(t) = (\alpha_{1}(t), \alpha_{2}(t), \dotsc, \alpha_{n}(t)) \]
  ... waarbij $I=(a,b)$ een open interval is en de $a_{i}:I \rightarrow \mathbb{R}$ re\"ele functies zijn.
  De functies $\alpha_{i}$ noemen we de \term{co\"ordinaatsfunctes} van $\alpha$ en $t$ de \term{parameter} op de kromme.
  $\alpha$ wordt ook wel eens de \term{parametrisatie} van een kromme genoemd.
\end{de}

\begin{de}
  We noemmen een kromme $\alpha$ \term{differentieerbaar} als elke $\alpha_{i}$ oneindig vaak differentieerbaar is.
  \[ a_{i} \in C^{\infty}(I,\mathbb{R}) \]
\end{de}

\begin{de}
  Een \term{rechte} in $\mathbb{E}^{n}$ is een kromme.
  Vvoor $p\in \mathbb{E}^{n}$ en $v\in \mathbb{R}^{n}$ is de afbeelding $\alpha$ een parametrisatie voor de rechte door $p$ in de richting van $v$.
    \[ \alpha:\ \mathbb{R} \rightarrow \mathbb{E}^{n}:\ t \mapsto p+tv \]
\end{de}

\begin{de}
  Een \term{cirkel} in $\mathbb{E}^{n}$ is een kromme.
  \[ \alpha: \mathbb{R} \rightarrow \mathbb{E}^{2}:\ t \mapsto (m_{1}+ R\cos(t),m_{2}+ R\sin(t)) \]
\end{de}

\begin{de}
  \examen
  Een \term{helix} of \term{cirkelschroeflijn} in $\mathbb{E}^{3}$ is de baan van een punt dat een schroefbeweging uitvoert.
  \[ \alpha: \mathbb{R} \rightarrow \mathbb{E}^{3}\ t \mapsto (a\cos t, a \sin t, bt) \]
\end{de}

\begin{de}
  Zij $\alpha: I \subseteq \mathbb{R} \rightarrow \mathbb{E}^{n}$ een kromme.
  Een \term{vectorveld langs $\alpha$} is een afbeelding van de vorm $Y$:
  \[ Y:\ I \rightarrow T\mathbb{E}^{n}:\ t\mapsto Y(t) = (Y_{1}(t),\dotsc,Y_{n}(t))_{\alpha(t)} \in T_{\alpha(t)}\mathbb{E}^{n} \]
  We noemen de $Y_{i}: I \rightarrow \mathbb{R}$ de componentsfuncties van $Y$.
\end{de}

\begin{de}
  $Y$ is \term{differentieerbaar} als en slechts als elke $Y_{i}$ oneindig vaak differentieerbaar is.
  \[ Y_{i} \in C^{\infty}(I,\mathbb{R}) \]
\end{de}

\begin{opm}
  Vanaf dit punt in de cursus zullen we steeds veronderstellen dat krommen, vectorvelden en functies onbeperkt differentieerbaar zijn.
\end{opm}

\begin{de}
  Het \term{snelheidsvectorveld} $\alpha'$ associeert met een kromme $\alpha$ en vectorveld als volgt:
  \[ \alpha:\ I \subseteq \mathbb{R} \rightarrow \mathbb{E}^{n}:\ t \mapsto \alpha(t) = (\alpha_{1}(t), \alpha_{2}(t), \dotsc, \alpha_{n}(t)) \]
  \[ \alpha':\ I \rightarrow T\mathbb{E}^{n}:\ t\mapsto \alpha'(t) = (\alpha'_{1}(t),\dotsc,\alpha'_{n}(t))_{\alpha(t)} \in T_{\alpha(t)}\mathbb{E}^{n} \]
  Hierin is $\alpha'_{i}$ de afgeleide van $\alpha$.
  We defini\"eren bovendien de \term{snelheid} van $\alpha$ als $v$:
  \[ v: I \rightarrow \mathbb{R}:\ t \mapsto v(t) = \Vert \alpha'(t) \Vert \]
  Terslotte defini\"eren we de \term{lengte} van het segment $\alpha$ tussen $\alpha(a)$ en $\alpha(b)$. als $l$:
  \[ l = \int_{a}^{b}v(t)dt = \int_{a}^{b}\Vert \alpha'(t) \Vert dt \]
\end{de}

\begin{de}
  Zij $\alpha$ een kromme en $a\in I$ een element (van wat?)
  We defini\"eren de \term{booglengte} vanaf $a$ als $s$:
  \[ s:\ I \rightarrow \mathbb{R}:\ t \mapsto s(t) = \int_{a}^{t}v(u)du = \int_{a}^{t}\Vert \alpha'(u) \Vert du  \]
\end{de}

\begin{de}
  De \term{raaklijn} aan $\alpha$ in $t_{0}\in I$ (of in $\alpha(t_{0})$) is de rechte door $\alpha(t_{0})$ in de richting van $\alpha'(t_{0})$.
  (De raaklijn is enkel gedefini\"eerd als $\alpha'(t_{0})$ niet nul is.)
\end{de}

\begin{de}
  Zij $Y = (Y_{1},\dotsc,Y_{n})$ een vectorveld langs een kromme $\alpha$ met componentsfuncties $Y_{i}$.
  De \term{afgeleide} $Y'$ van het vectorveld $Y$ is het vectorveld met componentsfuncties $Y_{i}'$.
  \[ Y' = (Y'_{1},\dotsc,Y'_{n}) \]
\end{de}

\begin{de}
  Het \term{versnellingsvectorveld} is de afgeleide van het snelheidsvectorveld.
\end{de}

\begin{de}
  Zij $Y$ en $Z$ twee vectorvelden langs een kromme $\alpha:\ I \subseteq \mathbb{R}\rightarrow \mathbb{E}^{n}$.
  We defini\"eren $Y+Z$ als volgt:
  \[ (Y + Z)(t) = Y(t) + Z(t)\]
\end{de}

\begin{de}
  Zij $Y$ en $Z$ twee vectorvelden langs een kromme $\alpha:\ I \subseteq \mathbb{R}\rightarrow \mathbb{E}^{n}$.
  We defini\"eren $Y\cdot Z$ als volgt:
  \[ (Y \cdot Z)(t) = Y(t) \cdot Z(t)\]
\end{de}

\begin{de}
  Zij $Y$ een vectorveld langs een kromme $\alpha:\ I \subseteq \mathbb{R}\rightarrow \mathbb{E}^{n}$ en $f: I \rightarrow \mathbb{R}$ een functie.
  We defini\"eren $fY$ als volgt:
  \[ (fY)(t) = f(t)Y(t)\]
\end{de}

\begin{st}
  De \term{rekenregels van de afgeleide van een vectorveld}\\
  \begin{itemize}
  \item $(Y+Z)' = Y' + Z'$
  \item $(Y \cdot Z)' = Y'\cdot Z + Y \cdot Z'$
  \item $(fY)' = f'Y + fY'$
  \end{itemize}
\end{st}

\begin{de}
  Een \term{orthonormaal referentieveld} langs een kromme $\alpha$ defini\"eren we als een verzameling van $n$ vectorvelden $Y_{i}$ zodat $Y_{i}\cdot Y_{j} = \delta_{ij}$ geldt.
\end{de}

\section{Herparametrisaties en de booglengteparametrisatie}
\label{sec:herp-en-de}

\begin{de}
  Een \term{diffeomorfisme} is een bijectief en differentieerbare functie $h$ zodat ook $h^{-1}$ differentieerbaar is.
\end{de}

\begin{de}
  Zij $I,\bar{I} \subseteq \mathbb{R}$ open intervallen en $\alpha:\ I \rightarrow \mathbb{E}^{n}$ een kromme.
  Als $h: \bar{I} \rightarrow I$ een difeomorfisme is, dan is $\beta$ opnieuw een kromme die we een \term{Herparametrisatie} van $\alpha$ noemen.
  \[ \beta = \alpha \circ h: \bar{I} \rightarrow \mathbb{R} \]
\end{de}

\begin{st}
  Zij $\beta = \alpha \circ h$ een herparametrisatie van een kromme $\alpha$ dan geldt het volgende ...
  \[ v_{\beta} = |h'|(v_{\alpha} \circ h) \]
  ... en als $h$ bovendien stijgend is ook het volgende:
  \[ s_{\beta} = s_{\alpha}(h(t)) \]
  Als $h$ dalend is geldt analoog het volgende:
  \[ s_{\beta} = -s_{\alpha}(h(t)) \]
  \begin{proof}
    We bewijzen de delen appart
    \begin{itemize}
    \item 
      Beschouw eerst $\beta'(t)$.
      \[ \beta'(t) = (\alpha \circ h)'(t) = \alpha'(h(t)) h'(t) \] Gaan
      we dan verder met $v_{\beta}(t)$:
      \[ v_{\beta}=(t) = \Vert \beta'(t) \Vert = \Vert
      \alpha'(h(t))h'(t) \Vert = |h'(t)|\Vert \alpha'(h(t)) \Vert =
      |h'(t)|v_{\alpha}(h(t)) \] Of kortweg:
      \[ v_{\beta} = |h'|(v_{\alpha} \circ h) \]
    \item 
      De booglengte van $\beta$, gemeten vanaf $t_{0}\in \bar{I}$ en de booglengte van $\alpha$, gemeten vanaf $h_{0} = h(t_{0})$ verhouden zich als volgt:
      \[ s_{\beta} = \int_{t_{0}}^{t}v_{\beta}(u)\ du = \int_{t_{0}}^{t}v_{\alpha}(h(u))h'(u)\ du = \int_{h(t_{0})}^{h(t)}v_{\alpha}v_{\alpha}(u)\ dv = s_{\alpha}(h(t)) \]
    \end{itemize}
  \end{proof}
\end{st}

\begin{opm}
  Het beeld van een kromme en een herparametrisatie van die kromme valt samen.
\end{opm}

\begin{de}
  We noemen een kromme $\alpha$ \term{regulier} als $v(t)$ strikt positief is voor alle $t$.
\end{de}

\begin{st}
  Zij $\alpha$ een reguliere kromme.
  \[ s_{\alpha}'(t) = v_{\alpha}(t) \]
\end{st}

\begin{st}
  Elke herparametrisatie van een reguliere kromme is regulier.
  \zb
\end{st}

\begin{st}
  Zij $\alpha$ een reguliere kromme.
  \begin{itemize}
  \item $\alpha$ heeft een herparametrisatie met snelheid $1$.
  \item Als $\beta$ en $\beta'$ beide zulke
    herparametrisaties zijn, dan geldt het volgende voor $c\in \mathbb{R}$ een constante.
    \[ \beta(t) = \beta'(\pm t + c) \]
  \end{itemize}

  \begin{proof}
    We bewijzen elk deel appart.
    \begin{itemize}
    \item 
      Als $\alpha$ regulier is, heeft $\alpha$ overal een positieve snelheid.
      $s_{\alpha}$ is dan een diffeomorfisme van $I$ naar $\bar{I} = s_{\alpha}(I)$.
      Noem de inverse van $s_{\alpha}$ nu $h$:
      \[ h:\ \bar{I}\rightarrow I \] Er geldt dan vanuit die definitie
      het volgende:
      \[ \forall t:\ s_{\alpha}(h(t)) = t \] We leiden deze vergelijking
      nu af om het volgende te bekomen:
      \[ \forall t:\ v_{\alpha}(h(t))h'(t) = 1 \] De herparametrisatie
      $\beta = \alpha \circ h$ voldoet aan de voorwaarden:
      \[ v_{\beta} = |h'|(v_{\alpha} \circ h) = h'(v_{\alpha} \circ h) =
      1\]
    \item Stel nu dat $\beta_{1}$ en $\beta_{2}$ beide herparametrisaties van $\alpha$ zijn met snelheid $1$, dan bestaat er een diffeomorfisme $q$ als volgt:
      \[ \beta_{1}(t) = \beta_{2}(q(t))\]
      $q'(t)$ moet dan $1$ of $-1$ zijn, dus $q(t)$ moet $\pm t + c$ zijn voor een bepaalde constante $c\in \mathbb{R}$.
    \end{itemize}
  \end{proof}
\end{st}

\begin{de}
  Een kromme met snelheid $1$ noemen we \term{booglengtegeparametriseerd}.
\end{de}
\begin{de}
  Een herparametrisatie met snelheid $1$ van een reguliere kromme noemen we een \term{booglengteherparametrisatie}.
\end{de}

\begin{st}
  \label{st:constante-snelheid-loodrecht-op-versnelling}
  Een reguliere kromme $\alpha$ heeft een constante snelheid als en slechts als $\alpha'' \bot \alpha'$ geldt.

  \begin{proof}
    Als $\alpha$ een constante sneldheid $c$ heeft, dan geldt $\alpha'\cdot \alpha' = c^{2}$ en bijgevolg het volgende:
    \[ 0 = (\alpha'\cdot \alpha')' = 2\alpha''\cdot \alpha' \]
  \end{proof}
\end{st}

\begin{st}
  Een reguliere kromme $\alpha$ is een rechte als en slechts als $\alpha''$ en $\alpha'$ colineair zijn.

  \begin{proof}
    Zij $\beta$ de booglengteherparametrisatie van $\alpha$
    Dan is $\beta''$ nul als $\beta$ (en dus $\alpha$) een rechte is:
    \[ \beta(s) = p+sv \text{ met } \Vert v \Vert = 1 \]
    Omgekeerd, als $f\beta' = \beta''$ geldt, dan ook het volgende:
    \[ f = f \cdot \beta'\cdot \beta' = \beta''\cdot \beta' = 0 \]
    Merk op dat de laatste gelijkheid enkel geldt omdat $\beta$ een constante snelheid heeft.\stref{st:constante-snelheid-loodrecht-op-versnelling}
    $\beta''$ moet nu dus nul zijn, en $\beta$ (en dus ook $\alpha$) daarom een rechte.
  \end{proof}
\end{st}

\section{Congruente krommen}
\label{sec:congruente-krommen}

\begin{opm}
   Zij $\alpha$ een kromme $\alpha:\ I\rightarrow \mathbb{E}^{n}$ en $F$ een isometrie van $\mathbb{E}^{n}$, dan is $F \circ \alpha$ opnieuw een kromme in $\mathbb{E}^{n}$.
\end{opm}

\begin{de}
  Zij $Y$ een vectorveld langs een kromme $\alpha:\ I\rightarrow \mathbb{E}^{n}$ en $F$ een isometrie van $\mathbb{E}^{n}$.
  We definieren een vectorveld $F_{*}Y$ langs $F(\alpha)$ als volgt:
  \[ (F_{*}Y)(t) = F_{*}(Y(t)) \in F_{*}(T_{\alpha(t)}\mathbb{E}^{n}) = T_{F(\alpha)(t)}\mathbb{E}^{n} \]
\end{de}

\begin{st}
  \label{st:rekenregel-afgeleide-afbeelding-na-isometrie}
  Zij $F = b \circ A$ een isometrie van $\mathbb{E}$ en $\alpha$ een kromme $\alpha:\ I\rightarrow \mathbb{E}^{n}$.
  \[ (F(\alpha))' = F_{*}(\alpha') \]
  \begin{proof}
    \[ (F(\alpha))'(t) = (A\alpha + b)'(t) = (A\alpha')(t) = A(\alpha'(t)) = F_{*}(\alpha'(t)) \]
  \end{proof}
\end{st}

\begin{st}
  Zij $F = b \circ A$ een isometrie van $\mathbb{E}$ en $\alpha$ een kromme $\alpha:\ I\rightarrow \mathbb{E}^{n}$.
  \[ (F_{*}(Y))' = F_{*}(Y') \]
  \begin{proof}
    \[ (F_{*}(Y))'(t) = (AY)'(t) = A(Y'(t)) = F_{*}(Y'(t)) = (F_{*}Y')(t) \]
  \end{proof}
\end{st}

\begin{st}
  Zij $F = b \circ A$ een isometrie van $\mathbb{E}$ en $\alpha$ een kromme $\alpha:\ I\rightarrow \mathbb{E}^{n}$.
  \[ v_{F(\alpha)} = v_{\alpha} \]
  \begin{proof}
    \[ v_{F(\alpha)} = \Vert F(\alpha)' \Vert = \Vert F_{*}(\alpha') = \Vert \alpha' \Vert = v_{\alpha} \]
    Merk op dat dit enkel geldt omdat $F$ afstanden bewaart.
  \end{proof}
\end{st}

\begin{gev}
  De afbeelding van een kromme onder een isometrie is booglengtegeparametriseerd als en slechts als de kromme booglengtegeparametriseerd is.
\end{gev}

\begin{st}
  Zij $F = b \circ A$ een isometrie van $\mathbb{E}$ en $\alpha$ een kromme $\alpha:\ I\rightarrow \mathbb{E}^{n}$.
  \[ s_{F(\alpha)} = s_{\alpha} \]
  \begin{proof}
    \[ v_{F(\alpha)} = \int_{a}^{t}v_{F(\alpha)}(u)\ du = \int_{a}^{t}v_{\alpha}(u)\ du= v_{\alpha} \]
    Merk op dat dit enkel geldt omdat $F$ afstanden bewaart.
  \end{proof}
\end{st}


\begin{de}
  We noemen twee krommen $\alpha$ en $\beta$ in $\mathbb{E}^{n}$ \term{congruent} als er een isometrie $F$ van $\mathbb{E}^{n}$ bestaat zodat $\beta = F \circ \alpha$ geldt.
\end{de}



\section{Krommen en de afgeleide afbeelding}
\label{sec:krommen-en-de}


\begin{st}
  Zij $U$ een open deel van $\mathbb{E}^{n}$ en $F: U \subseteq \mathbb{E}^{n} \rightarrow \mathbb{E}^{n}$ een differentieerbare afbeelding.
  Kies $p$ een element van $U$ en $v_{p}\in T_{p}\mathbb{E}^{n}$ een raakvector aan $p$.
  \[ (F_{*})_{p}v_{p} = (P \circ \alpha)'(t_{0}) \]
  Hierin is $\alpha:\ I \subseteq \mathbb{R} \rightarrow \mathbb{E}^{n}$ een willekeurige kromme zodat $\alpha(t_{0}) = p$ en $\alpha'(t_{0}) = v_{p}$ gelden en waarvan het beeld een deel is van $U$.
\TODO{bewijs p 129}
\end{st}


\chapter{Krommen in het Euclidisch vlak $\mathbb{E}^{2}$}
\label{cha:kromm-het-eucl}

\section{De complexe structuur van $\mathbb{E}^{2}$}
\label{sec:de-compl-struct}

\begin{de}
  De \term{complexe structuur} $J$ van $\mathbb{E}^{2}$ is de volgende afbeelding:
  \[
  J: T\mathbb{E}^{2} \rightarrow T\mathbb{E}^{2}: 
  \begin{pmatrix}
    v_{1}\\v_{2}
  \end{pmatrix}_{p}
  \mapsto
  \begin{pmatrix}
    -v_{2}\\v_{1}
  \end{pmatrix}_{p}
  =
  \left(
    \begin{pmatrix}
      0 & -1\\
      1 & 0
    \end{pmatrix}
    \begin{pmatrix}
      v_{1}\\v_{2}
    \end{pmatrix}
  \right)_{p}
  \]
\end{de}

\begin{opm}
  De complexe structuur kan je je voorstellen door een vector $90$ graden tegen de klok in te draaien.
\end{opm}

\begin{st}
  $J$, beperkt tot een rakende ruimte $T_{p}\mathbb{E}^{2}$  is een orthogonale lineaire transformatie.
\extra{bewijs}
\end{st}

\begin{st}
  $J^{2} = -\mathbb{I}_{2}$
\extra{bewijs}
\end{st}

\begin{st}
  $J$ is antisymmetrisch.
  \[ \forall v,w \in T_{p}\mathbb{E}^{2}:\ Jv\cdot w = -v \cdot Jw \]
\extra{bewijs}
\end{st}

\begin{st}
  \[ \forall v,w \in T_{p}\mathbb{E}^{2}:\ Jv\cdot v = 0 \]
\extra{bewijs}
\end{st}

\begin{st}
  Zij $v\in T_{p}\mathbb{E}^{2}$ een eenheidsvector, dan is $(v,Jv)$ een positief geori\"enteerde orthonormale basis voor $T_{p}\mathbb{E}^{2}$.
\extra{bewijs}
\end{st}

\section{Het Frenet-apparaat voor booglengtegeparametriseerde vlakke krommen}
\label{sec:het-frenet-apparaat}

\begin{de}
  Zij $\beta$ een booglengtegeparametriseerde krommen in $\mathbb{E}^{2}$.
  Het vectorveld $T(s) = \beta'(s)$ is het \term{eenheidsrakend vectorveld} van $\beta$.
\end{de}

\begin{de}
  Zij $\beta$ een booglengtegeparametriseerde krommen in $\mathbb{E}^{2}$.
  Het vectorveld $N = JT$ noemen we de \term{geori\"enteerde normaal} op $\beta$.
\end{de}

\begin{de}
  Zij $\beta$ een booglengtegeparametriseerde krommen in $\mathbb{E}^{2}$.
  $(T,N)$ is een positief geori\"enteerd orthonormaal referentiestelsel langs $\beta$ dat we het \term{Frenet-apparaat} noemen.
\end{de}

\begin{st}
  Zij $\beta: I \subseteq \mathbb{R} \rightarrow \mathbb{E}^{2}$ een booglengtegeparametriseerde kromme en zij $(T,N)$ het Frenet-referentiestelsel, dan bestaat er een functie $\kappa: I \rightarrow \mathbb{R}$ zodat het volgende geldt:
  \[ 
  \left\{
    \begin{array}{cl}
      T' = \kappa N\\
      N' = -\kappa T
    \end{array}
  \right.
  \]
\TODO{bewijs p 136}
\end{st}

\begin{de}
  We noemen $\kappa= T'\cdot N$ de \term{georienteerde kromming} van $\beta$ en de formules uit de vorige stelling de \term{formules van Frenet}.
\end{de}

\begin{st}
  Stel dat $\beta: I \subseteq \mathbb{R} \rightarrow \mathbb{E}^{2}$ een booglengtegeparametriseerde kromme is met kromming nul, dan is $\beta$ een deel van een rechte.
\TODO{bewijs p 137}
\end{st}

\begin{st}
  Stel dat $\beta: I \subseteq \mathbb{R} \rightarrow \mathbb{E}^{2}$ een booglengtegeparametriseerde kromme is met constante kromming $k$, dan is $\beta$ een cirkel met straal $\frac{1}{|k|}$.
\TODO{bewijs p 138}
\end{st}
\TODO{osculerende parabool}

\section{Osculerende cirkel en evoluut}
\label{sec:oscul-cirk-en}

\begin{de}
  Zij $\beta:\ I \subseteq \mathbb{R} \rightarrow \mathbb{E}^{2}$ een booglengtegeparametriseerde kromme, $s_{0}$ een element van $I$ zodat $\kappa(s_{0})$ niet nul is.
  Het punt $m$ wordt het \term{krommingsmiddelpunt} van $\beta$ in $s_{0}$ genoemd.
  \[ m = \beta(s_{0}) + \frac{1}{\kappa(s_{0})}N(s_{0})\]
\end{de}

\begin{de}
  Zij $\beta:\ I \subseteq \mathbb{R} \rightarrow \mathbb{E}^{2}$ een booglengtegeparametriseerde kromme met krommingsmiddelpunt $s_{0}$ van $\beta$.
  De cirkel met middelpunct $M$ en straal $r= \frac{1}{|\kappa(s_{0})|}$ noemen we de \term{osculatiecirkel} aan $\beta$ in $\beta(s_{0})$ genoemd.
\end{de}

\begin{st}
  Zij $\beta:\ I \subseteq \mathbb{R} \rightarrow \mathbb{E}^{2}$ een booglengtegeparametriseerde kromme, $s_{0}$ een element van $I$ zodat $\kappa(s_{0})$ niet nul is.
  Er bestaat precies \'e\'en booglengtegeparametriseerde cirkel $c$ waarvoor het volgende geldt:
  \[ c(s_{0}) = \beta(s_{0}) \wedge c'(s_{0}) = \beta'(s_{0}) \wedge c''(s_{0}) = \beta''(s_{0})\]
\TODO{bewijs p 143}
\end{st}

\begin{de}
  Zij $\beta:\ I \subseteq \mathbb{R} \rightarrow \mathbb{E}^{2}$ een booglengtegeparametriseerde kromme waarbij $\kappa(s)$ voor geen enkele $s\in I$ nul is.
  De kromme $\gamma$ die bestaat uit alle krommingsmiddelpunten van $\beta$ wordt de \term{centrale kromme} of de \term{evoluut} van $\beta$ genoemd.
  \[ \gamma(s) = \beta(s) + \frac{1}{\kappa(s)}N(s) \]
\end{de}

\section{Spiraalbogen en het Lemma van Kneser}
\label{sec:spiraalbogen-en-het}

\begin{de}
  Zij $\beta:\ I \subseteq \mathbb{R} \rightarrow \mathbb{E}^{2}$ een booglengtegeparametriseerde kromme met kromming $\kappa$ en zij $a$ en $b$ elementen van $I$ als volgt, dan noemen we $\beta|_{[a,b]}$ een \term{spiraalboog}.
  \begin{itemize}
  \item $a \le b$
  \item $\kappa > 0$ op $[a,b]$
  \item $\kappa$ is strikt stijgend op $[a,b]$.
  \end{itemize}
\end{de}

\begin{de}
  Noteer $B_{s}$, respectievelijk $\overline{B_{s}}$ voor de open, respectievelijk gesloten, schijf met als rand de osculatiecirkel in $\beta(s)$.
\end{de}

\begin{lem}
  Het \term{Lemma van Kneser}\\
  Als $\beta|_{[a,b]}$ een spiraalboog is, dan geldt $\overline{B_{b}} \subseteq B_{a}$.
\TODO{bewijs p 146}
\end{lem}

\begin{st}
  Een spiraalboog heeft geen zelfdoorsnijdingen.
\extra{bewijs}
\end{st}

\begin{st}
  Als $\beta|_{[a,b]}$ een spiraalboog is, en $c$ een element van $(a,b)$, dan ligt $\beta|_{[a,c)}$ helemaal buiten $B_{c}$ en $\beta|_{(c,b]}$ helemaal binnen $B_{c}$.
\extra{bewijs}
\end{st}

\section{Kromming van congruente krommen}
\label{sec:kromm-van-congr}

\begin{st}
  Zij $F$ een isometrie van $\mathbb{E}^{2}$ en zij $\epsilon = \det F_{*} = \pm 1$
  Als $\alpha$ en $\beta = F \circ \alpha$ twee congruente booglengtegeparametriseerde krommen zijn, dan geldt het volgende:
  \[ F_{*}T_{\alpha} = T_{\beta} \quad\wedge\quad F_{*}N_{\alpha} = \epsilon N_{\beta} \wedge \kappa_{\alpha} = \kappa_{\beta} \]

  \begin{proof}
    \[ F_{*}T_{\alpha} = F_{*}(\alpha') = (F \circ \alpha)' = \beta' = T_{\beta} \]
    \stref{st:rekenregel-afgeleide-afbeelding-na-isometrie}
\TODO{bewijs p 148}
  \end{proof}
\end{st}

\begin{st}
  \examen
  De \term{congruentiestelling voor vlakke krommen}\\
  Zij $\alpha$ en $\beta$ twee booglengtegeparametriseerde krommen die op een zelfde interval $I$ gedefini\"eerd zijn.
  \[ \alpha,\beta:\ I \subseteq \mathbb{R} \rightarrow \mathbb{E}^{2} \]
  Als $\kappa_{\alpha}$ gelijk is aan $\kappa_{\beta}$, dan bestaat er een ori\"entatiebewarende isometrie $F$ zodat $\beta = F \circ \alpha$ geldt.
  Als $\kappa_{\alpha}$ tegengesteld is aan $\kappa_{\beta}$, dan bestaat er een ori\"entatieomkerende isometrie $F$ zodat $\beta = F \circ \alpha$ geldt\\
  Met andere woorden:
  \textbf{``Twee booglengtegeparametriseerde krommen zijn congruent als ze dezelfde (of een tegengestelde) kromming hebben.''}

  \begin{proof}
    Gevalsonderscheid.
    \begin{itemize}
    \item $\kappa_{\alpha} = \kappa_{\beta}$\\
      \begin{itemize}
      \item 
      Stel dat $\alpha$ en $\beta$ twee krommen zijn met constante snelheid $1$ en dezelfde kromming.
      Beschouw een willekeurig punt $s_{0}$ uit het interval $I$.
      Zij $F$ de (unieke!\stref{st:bestaan-isometrie}) isometrie die $\alpha(s_{0})$ op $\beta(s_{0})$ afbeeldt alsook het bijhornde Frenet apparaat.
      \[ F(\alpha(s_{0})) = \beta_{s_{0}} \quad\wedge\quad
      F_{*}(T_{\alpha}(s_{0})) = T_{\beta}(s_{0}) \quad\wedge\quad
      F_{*}(N_{\alpha}(s_{0})) = N_{\beta}(s_{0}) \] Deze isometrie is
      bovendie ori\"entatiebewarend.\gevref{gev:orientatie-isometrie}
    \item We defini\"eren nu een hulpkromme $\gamma$ als $F \circ \alpha$ en zullen aantonen dat $\gamma$ gelijk is aan $\beta$.
      Merk nu op dat $\gamma$ en $\alpha$ congruent zijn, dus ze hebben dezelfde kromming.
      $\beta$ heeft in het bijzonder dan ook dezelfde kromming als $\alpha$.
    \item Definieer nu de functie $f$: (merk op dat we de aangrijpingspunten verwaarlozen, het scalair product maakt toch enkel gebruik van het vectordeel).
      \[ f:\ I \rightarrow \mathbb{R}:\ s \mapsto f(s) = T_{\beta}(s) \cdot T_{\gamma}(s) \]
      We gebruiken nu de ongelijkheid van Cauchy-Schwarz:
      \[ T_{\beta}(s) \cdot T_{\gamma}(s) \le \left\| T_{\beta}(s) \right\| \cdot \left\| T_{\gamma}(s) \right\|  \]
      $f(s)$ is dus steeds kleiner of gelijk aan $1$ met gelijkheid wanneer $T_{\beta}(s)$ en $T_{\gamma}(s)$ gelijk zijn.

    \item We bewijzen nu dat $f(s)$ identiek $1$ is, daaruit volgt dan dat $\beta$ en $\gamma$ gelijk zijn in eerste afgeleide.
      \begin{itemize}
      \item $f(s_{0}) = T_{\beta}(s_{0}) \cdot T_{\gamma}(s_{0}) = T_{\beta}(s_{0}) \cdot F_{*}T_{\alpha}(s_{0}) = T_{\beta}(s_{0})^{2} = 1$.
      \item $f'$ is identiek nul en $f$ dus constant.
        \[
        \begin{array}{rll}
          f' &= T_{\beta}'\cdot T_{\gamma} + T_{\beta}\cdot T'_{\gamma}\\
             &= \kappa_{\beta}N_{\beta} \cdot T_{\gamma} + T_{\beta} \cdot \kappa_{\gamma}N_{\gamma}\\
             &= \kappa_{\beta}(JT_{\beta} \cdot T_{\gamma} + T_{\beta} \cdot JT_{\gamma})
             &= 0
        \end{array}
        \]
        In de eerste gelijkheid gebruiken we de productregel van afgeleiden op de definitie van $f$.
        In de tweede gelijkheid gebruiken we de definitie van de kromming: $T' = \kappa N$.
        Tenslotte gebruiken we nog dat $\kappa_{\gamma}$ en $\kappa_{\beta}$ gelijk zijn en de definitie van $N$: $N=JT$.
        Ook nog belangrijk voor de laatste gelijkheid is dat $J$ antisymmetrisch is.
      \item $f$ is dus constant en $1$ in $s_{0}$, dus identiek $1$.
      \end{itemize}

    \item Integratie van $\beta'=\gamma'$ geeft ons $\beta(s) =\gamma(s) + c$, maar omdat $\gamma$ en $\beta$ gelijk zijn in $s_{0}$ moet $c$ nul zijn.
    \end{itemize}

    \item $\kappa_{\alpha} = - \kappa_{\beta}$\\
    \end{itemize}
  \end{proof}
\end{st}

\section{Intrinsieke vergelijking}
\label{sec:intr-verg}

\begin{lem}
  Als $f$ en $g$ twee functies $I \subseteq \mathbb{R} \rightarrow \mathbb{R}$ zijn met $f^{2}+g^{2} = 1$, dan bestaat er een functie $\theta: I \rightarrow \mathbb{R}$ zodat het volgende geldt:
  \[ f = \cos(\theta) \text{ en } g = \sin(\theta) \]
\TODO{bewijs p 150}
\end{lem}
\TODO{hoekfunctie}

\begin{de}
  Een vergelijking van de vorm ``$\kappa = \dotsb$'' bepaalt een unieke booglengtgeparametriseerde vlakke kromme, op ori\"entatiebewarende isometrie\"en na.
  Zo\'n vergelijking heet de \term{intrinsieke vergelijking} van een kromme.
\end{de}

\section{Kromming van reguliere krommen}
\label{sec:kromm-van-regul}


\begin{st}
  De \term{formules van Frenet voor reguliere vlakke krommen}\\
  Beschouw een reguliere kromme $\alpha:\ I \subseteq \mathbb{R} \rightarrow \mathbb{E}^{2}$ met snelheid $v>0$, dan gelden volgende vergelijkingen.
  \[
  \left\{
    \begin{array}{cl}
      T' &= v\kappa N\\
      V' &= -v\kappa T
    \end{array}
  \right.
  \]
\TODO{bewijs p 153}
\end{st}

\section{Globale studie van vlakke krommen}
\label{sec:globale-studie-van}

\begin{de}
  Een \term{lokale eigenschap} van een kromme beschrijft de kromme in de buurt van een bepaald punt.
\end{de}

\begin{de}
  Een \term{globale eigenschap} van een kromme bescrijft de kromme in zijn geheel.
\end{de}

\subsection{Gesloten krommen}
\label{sec:gesloten-krommen}

\begin{de}
  We noemen een kromme $\alpha:\ \mathbb{R} \rightarrow \mathbb{E}^{2}$ \term{gesloten} als er een strikt positief getal $\omega\in \mathbb{R}^{+}_{0}$ bestaat zodat $\alpha(t + \omega) = \alpha(t)$ voor elke $t\in \mathbb{R}$ geldt.
  $\omega$ noemen we een \term{periode} van $\alpha$.
\end{de}

\begin{de}
  De kleinste periode van een kromme noemen we de \term{echte periode}.
\end{de}

\begin{de}
  We noemen een gesloten kromme met periode $\omega$ \term{enkelvoudig gesloten} als de beperking $\alpha|_{[0,\omega)}$ injectief is. 
\end{de}

\begin{ei}
  Een enkelvoudig gesloten kromme snijdt zichzelf niet.
\extra{bewijs}
\end{ei}

\subsection{Totale kroming en rotatieindex}
\label{sec:totale-kroming-en}

\begin{de}
  De \term{rotatieindex} van een kromme $\beta$ noemen we $i_{\beta}$:
  \[ i_{\beta} = \frac{1}{2\pi} (\theta(L) - \theta(0)) \]
  De rotatieindex geeft aan hoe dikwijls een kromme netto ronddraait in positieve zin tijdens \'e\'en periode.
\end{de}

\begin{st}
  De kromming $\kappa$ van een gesloten booglengtegeparametriseerde vlakke kromme $\beta$ met lengte $L$ is gerelateerd aan de rotatieindex:
  \[ \int_{0}^{L}\kappa(s)\ ds = 2\pi i_{\beta} \]
\TODO{bewij p 158}
\end{st}

\begin{de}
  We noemen het volgende de \term{totale kromming} van $\beta$:
  \[ \int_{0}^{L}\kappa(s)\ ds \]
\end{de}

\begin{st}
  De rotatieindex van een enkelvoidig gesloten vlakke kromme is $1$ of $-1$.
\zb
\end{st}

\begin{st}
  De \term{stelling van Jordan}\\
  Een enkelvoudig gesloten vlakke kromme deelt het vlak op in twee gebieden:: een begrensde binnenkant en een onbegrensde buitenkant.
\zb
\end{st}

\begin{st}
  De \term{isoperimetrische ongelijkheid}\\
  \[ L^{2} \ge 4\pi A \]
  De gelijkheid treedt op als en slechts als $\beta$ een cirkel is (die eenmaal doorlopen wordt).
\zb
\end{st}


\chapter{Krommen in de Euclidische ruimte$\mathbb{E}^{3}$}
\label{cha:kromm-het-eucl}

\section{Het Frenet-apparaat voor krommen in $\mathbb{E}^{3}$}
\label{sec:het-frenet-apparaat-voor-krommen-in-e3}


\begin{de}
  Een kromme in de driedimensionale Euclidische ruimte noemen we een ruimtekromme.
\end{de}

\begin{de}
  Zij $\beta: I \subseteq \mathbb{R} \rightarrow \mathbb{E}^{3}$ een booglengtegeparametriseerde ruimtekromme.
  $\beta'$ is dan een vectorveld langs $\beta$ met lengte $1$.
  We noemen $\beta'$ het \term{eenheidsrakend vectorveld} aan $\beta$ en noteren dit met $T$.
\end{de}

\begin{st}
  Het vectorveld $T'$ staat loodrecht op $T$.
\extra{bewijs}
\end{st}

\begin{de}
  We defini\"eren de \term{kromming} van een ruimtekromme als $\kappa$:
  \[ \kappa = \Vert T' \Vert \]
\end{de}

\begin{st}
  Een ruimtekromme met kromming $0$ is een rechte.
\extra{bewijs}
\end{st}

\begin{de}
  We defini\"eren een vectorveld $N$ langs $\beta$ als volgt.
  \[ N = \frac{T'}{\kappa} \]
  $N$ noemen we het \term{hoofdnormaalvectorveld} van $\beta$.
\end{de}

\begin{de}
  We defini\"eren een vectorveld $B$ langs $\beta$ als volgt.
  \[ B = T \times N \]
  $B$ noemen we het \term{binormaalvectorveld} van $\beta$.
\end{de}

\begin{de}
  We noemen $(T,N,B)$ het \term{Frenet-referentiestelsel} van $\beta$.
\end{de}

\begin{lem}
  $B'$ is evenredig met $N$.
\TODO{bewijs p 166}
\end{lem}
\begin{de}
  We defini\"eren een functie $\tau$ als volgt.
  \[ \tau = -B' \cdot N \]
  $\tau$ wordt de \term{torsie} van $\beta$ genoemd.
\end{de}

\begin{st}
  De \term{formules van Frenet voor ruimtekrommen}\\
  Zij $\beta$ een booglengtegeparametriseerde ruimtekromme met kromming $\kappa>0$.
  \[
  \left\{
    \begin{array}{r cccc}
      T' &=& & \kappa N &\\
      N' &=& -\kappa T & & +\tau B\\
      B' &=& & -\tau N &\\
    \end{array}
  \right.
  \]
\TODO{bewijsp 167}
\end{st}

\begin{de}
  De rechte door $\beta(s)$ in de richting van $T(s)$ noemen we de \term{raaklijn} in $\beta(s)$ aan $\beta$.
\end{de}

\begin{de}
  Het vlak, opgespannen door $T(s)$ en $N(s)$, noemen we het \term{osculatievlak} in $\beta(s)$ aan $\beta$.
\end{de}

\begin{st}
  Zij $\beta$ een booglengtegeparametriseerde ruimtekromme met kromming $\kappa> 0$, dan is $\beta$ gelegen in een vlak als en slechts als $\tau$ nul is.
\TODO{bewijs p 169}
\end{st}

\section{Kromming en torsie van reguliere ruimtekrommen}
\label{sec:kromming-en-torsie}

\begin{st}
  De \term{formules van Frenet voor reguliere ruimtekrommen}\\
  Zij $\alpha$ een reguliere kromme in $\mathbb{E}^{3}$ met snelheid $v>0$ en kromming $\kappa > 0$.
  \[
  \left\{
    \begin{array}{r cccc}
      T' &=& & v\kappa N &\\
      N' &=& -v\kappa T & & +v\tau B\\
      B' &=& & -v\tau N &\\
    \end{array}
  \right.
  \]
\extra{bewijs}
\end{st}

\section{Kromming en torsie van congruente krommen}
\label{sec:kromming-en-torsie-1}

\begin{st}
  Zij $F$ een isometrie van $\mathbb{E}^{3}$ en zij $\epsilon = \det F_{*} \pm 1$.
  Als $\alpha$ en $\beta = F \circ \alpha$ twee congruente booglengtegeparametriseerde krommen zijn met $\kappa_{\alpha} > 0$, dan geldt het volgende:
  \[ F_{*}T_{\alpha} = T_{\alpha}, F_{*}N_{\alpha} = N_{\beta} \text{ en } F_{*}B_{\alpha}= B_{\beta} \]
  \[ \kappa_{\alpha} = \kappa_{\beta} \text{ en } \tau_{alpha} = \epsilon \tau_{\beta} \]
\TODO{bewijs p 175}
\end{st}

\begin{st}
  De \term{congruentiestelling voor ruimtekrommen}
  Zij $\alpha, \beta:\ I \subseteq \mathbb{R} \rightarrow \mathbb{E}^{3}$ twee booglengtegeparametriseerde ruimtekrommen die op een zelfde interval gedefini\"eerd zijn.
  Als $\kappa_{\alpha} = \kappa_{\beta}>0$ en $\tau_{alpha} = \pm \tau_{\beta}$ gelden, dan zijn $\alpha$ en $\beta$ congruent.
  Als $\tau_{alpha}$ gelijk is aan $\tau_{\beta}$, is de isometrie die $\alpha$ en $\beta$ relateert ori\"entatiebewarend, anders is die ori\"entatieomkerend.
\TODO{bewijs p 176}
\end{st}

\begin{st}
  Zij $\alpha, \beta:\ I \subseteq \mathbb{R} \rightarrow \mathbb{E}^{3}$ twee reguliere ruimtkrommen die op een  zelfde interval gedefini\"eerd zijn.
  Als $v_{\alpha} = v_{\beta} > 0$, $\kappa_{\alpha} = \kappa_{\beta}>0$ en $\tau_{alpha} = \pm \tau_{\beta}$ gelden, dan zijn $\alpha$ en $\beta$ congruent.
  \TODO{bewijs}
\end{st}

\section{Cilinderschroeflijnen en cirkelschroeflijnen}
\label{sec:cilind-en-cirk}

\begin{de}
  Een reguliere kromme $\alpha: I \subseteq \mathbb{R} \rightarrow \mathbb{E}^{3}$ wordt een \term{Cilinderschroeflijn} of \term{cilindrische helix} genoemd als het eenheidsrakend vectorveld $T$ een constante hoek maake met een gegeven vector .
  \[ \forall t \in I:\ \exists u\in \mathbb{R}^{3}, \exists \theta \in \mathbb{R}:\ \Vert u \Vert = 1 \wedge T(t) \cdot u = \cos(\theta) \]
\end{de}

\begin{st}
  Een reguliere ruimtekromme $\beta$ met $\kappa>0$ is een Cilinderschroeflijn als en slechts als $\frac{\tau}{\kappa}$ een constante functie is.
\TODO{bewijs p 181}
\end{st}

\begin{st}
  Als $\alpha$ een reguliere ruimtekromme is met $\kappa>0$ en $\tau$ allebei constant, dan is $\alpha$ (na Herparametrisatie) congruent met een cirkelschroeflijn.
\TODO{bewijs p 181}
\end{st}

\end{document}
