\documentclass[main.tex]{subfiles}
\begin{document}

\chapter{Het vectorproduct in $\mathbb{E}^{3}$}
\label{cha:het-vect-mathbb}

\begin{lem}
  Zij $v$ en $w$ twee raakvectoren in $p\in \mathbb{E}^{3}$, dan bestaat er juist \'e\'en raakvector $u$ in $p$ zodat voor alle raakvectoren $x$ in $p$ het volgende geldt:
  \[ x \cdot u = Vol(x,v,w) \]
  \begin{proof}
    \begin{itemize}
      Bewijs van uniek bestaan.
    \item Uniciteit\\
      Zij $(e_{1},e_{2},e_{3})$ de natuurlijke basis van $T_{p}\mathbb{E}^{3}$.
      Stel dat er een $u$ aan de voorwaarde voldoet voor een bepaalde $v$ en $w$ in $T_{p}\mathbb{E}^{3}$.
      \[
      u_{1} = e_{1}\cdot u = Vol(e_{1},v,w) = 
      \begin{vmatrix}
        1 & 0 & 0\\
        v_{1} & v_{2} & v_{3}\\
        w_{1} & w_{2} & w_{3}
      \end{vmatrix}
      =
      \begin{vmatrix}
        v_{2} & v_{3}\\
        w_{2} & w_{3}
      \end{vmatrix}
      \]
      Analoog vinden we de andere coordinaten van $u$:
      \[ 
      u_{2} = -
      \begin{vmatrix}
        v_{1} & v_{3}\\
        w_{1} & w_{3}
      \end{vmatrix}
      ,\quad
      u_{3} = 
      \begin{vmatrix}
        v_{1} & v_{2}\\
        w_{1} & w_{2}
      \end{vmatrix}
      \]
      $u$ moet dus uniek zijn.
    \item Bestaan\\
      Bestaan is nu eenvoudig.
      Construeer $u$ als volgt:
      \[ u = \left( 
        \begin{vmatrix}
          v_{2} & v_{3}\\
          w_{2} & w_{3}
        \end{vmatrix},
        -
        \begin{vmatrix}
          v_{1} & v_{3}\\
          w_{1} & w_{3}
        \end{vmatrix},
        \begin{vmatrix}
          v_{1} & v_{2}\\
          w_{1} & w_{2}
        \end{vmatrix}
      \right) \]
    \end{itemize}
  \end{proof}
\extra{wat een lelijk bewijs!}
\end{lem}

\begin{de}
  Voor elke twee raakvectoren $v$ en $w$ in $p\in \mathbb{E}^{3}$ defini\"eren we het vectorproduct $v\times w$ als volgt:
  \[
  v \times w =
  \begin{vmatrix}
    e_{1} & e_{2} & e_{3}\\
    v_{1} & v_{2} & v_{3}\\
    w_{1} & w_{2} & w_{3}
  \end{vmatrix}
  \]
  Hierboven is $(e_{1},e_{2},e_{3})$ de natuurlijke basis.
\end{de}
\TODO{formele}

\begin{st}
  Het vectorproduct is bilineair.
\question{definieer bilineair}
\extra{bewijs}
\end{st}

\begin{st}
  \[ \forall v,w \in V:\ v \bot v\times w \wedge w \bot \times v\times w \]
\extra{bewijs}
\end{st}

\begin{st}
  \[ \forall v,w \in V:\ v\times w = 0 \Leftrightarrow v \text{ en } w \text{ zijn lineair afhankelijk.} \]
\extra{bewijs}
\end{st}

\begin{st}
  \[ \forall v,w \in V:\ v\times w = - w\times v \]
\extra{bewijs}
\end{st}

\begin{st}
  \[ \forall u,v,w \in V:\ u \times (v \times w) = (u \cdot w)v - (u \cdot v)w \]
\extra{bewijs}
\end{st}

\begin{st}
  De \term{identiteit van Jacobi}\\
  \[ \forall u,v,w \in V:\ u \times (v \times w) + v \times (w \times u) + w \times (u \times v) = 0 \]
\extra{bewijs}
\end{st}

\begin{st}
  De \term{identiteit van Lagrange}\\
  \[
  \forall a,b,c,d \in V:\ (a\times b) \cdot (c \times d) =
  \begin{vmatrix}
    a \cdot c & a \cdot d\\
    b \cdot c & b \cdot d
  \end{vmatrix}
  \]
\TODO{bewijjs p 109}
\end{st}

\begin{st}
  \[ \forall v,w \in V:\ \Vert v \times w\Vert = \Vert v \Vert \Vert w \Vert |\sin(\theta)| \]
\TODO{bewijs p 109}
\end{st}

\begin{st}
  Als $v$ en $w$ lineair onafhankelijk zijn, dan is $(v,w,v\times w)$ een positief geori\"enteerde basis van $T_{p}\mathbb{E}^{3}$.
\extra{bewijs}
\end{st}

\begin{st}
  Als $(u,v,w)$ een positief geori\"enteerde orthonormale basis van $T_{p}\mathbb{E}^{3}$ is, dan geldt het volgende:
  \[ w = u \times v \wedge v = w \times u \wedge u = v \times w \]
\extra{bewijs}
\end{st}


\end{document}
