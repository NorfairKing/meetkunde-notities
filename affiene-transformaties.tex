 \documentclass[main.tex]{subfiles}
\begin{document}

\chapter{Affiene Transformaties}
\label{cha:affiene-transformaties}

\begin{de}
  Een afbeelding $F$ noemen we een \emph{affiene transformatie} als er een reguliere $n\times n$ matrix $A$ en een punt $b$ bestaat zodat we $F$ als volgt definieren:
  \[ F:\ \mathbb{A}^{n} \rightarrow \mathbb{A}^{n}: p \mapsto F(p) = Ap + b \text{ met } det(A) \neq 0 \]
  $A$ noemen we het lineair deel van $F$ en $b$ het translatiedeel.
\end{de}

\begin{de}
  Zij $F$ een affiene transformatie met $A=\mathbb{I}^{n}$, dan noemen we $F$ een translatie in de richting van $b$.
  \[ F:\ \mathbb{A}^{n} \rightarrow \mathbb{A}^{n}: p \mapsto F(p) = p + b \]
\end{de}

\begin{de}
  Zij $F$ een affiene transformatie als volgt, dan noemen we $F$ een \emph{homothetie} $H_{p_{0},r}$ met centrum $p_{0}$ en factor $r$.
  \[ F:\ \mathbb{A}^{n} \rightarrow \mathbb{A}^{n}: p \mapsto F(p) = p_{0} + r(\overrightarrow{p_{0}p}) = rp + (1-r)p_{0} \]
\end{de}

\begin{st}
  Een affiene transformatie $F:\ \mathbb{A}^{n} \rightarrow \mathbb{A}^{n}: p \mapsto F(p) = Ap + b$ kan steeds ontbonden worden in een translatie $t_{b}$ en een lineaire transformatie $A$.
  \[ F = t_{b} \circ A \]
  De ontbinding is bovendien uniek.

  \begin{proof}
    Stel dat deze ontbinding niet uniek was, dan bestonden er $t_{b},A,t_{d},C$ zodat $F = t_{b}\circ A = t_{d} \circ C$ geldt.
    $t_{b}$ en $t_{d}$ zijn dan beide gelijk want $F(0) = b = d$ geldt.
    Er moet dan ook voor alle $p\in \mathbb{A}^{n}$ gelden dat $Ap+b$ gelijk is aan $Cp+b$, dus $A$ moet gelijk zijn aan $C$.
  \end{proof}
\end{st}

\begin{st}
  Elke affiene transformatie is een bijectie.

  \begin{proof}
    Elke translatie is bijectief en heeft als inverse $(t_{b})^{-1} = t_{-b}$.
    \clarify{waarom?}
    Elke affiene transformatie is de samenstelling van inverteerbare lineaire transformatie en een translatie:
    \[ F = t_{b} \circ A \]
    Zowel $t_{b}$ als $A$ zijn inverteerbaar, dus $F$ is ook inverteerbaar.
  \end{proof}
\end{st}

\begin{de}
  Zij $F = t_{b} \circ A$ een affiene transformatie en $p\in \mathbb{A}^{n}$ een punt, dan definieren we de \emph{afgeleide afbeelding} $(F_{*})_{p}$ in het punt $p$ als de volgende lineaire afbeelding:
  \[ (F_{*})_{p}:\ T_{p}\mathbb{A}^{n} \rightarrow T_{F(p)}\mathbb{A}^{n}:\ v_{p} \mapsto (F_{*})_{p}(v_{p}) = (Av)_{F(p)} \]
  We definieren ook de afgeleide afbeelding $F_{*}v$ op vrije vectoren:
  \[ F_{*}:\ V\rightarrow V:\ v\mapsto F_{*}(v) = Av \]
\end{de}

\begin{st}
  Zij $F = t_{b} \circ A$ een affiene transformatie en $p,q \in \mathbb{A}^{n}$ twee punten, dan geldt het volgende:
  \[ \overrightarrow{F(p)F(q)} = F_{*}(\overrightarrow{pq}) \]

  \begin{proof}
    \[
    \begin{array}{rll}
      \overrightarrow{F(p)F(q)} &= F(q) - F(p) &\\
                                &= (Aq+b) - (Ap+b) &\\
                                &= Aq - Ap &\\
                                &= A(q-p) &= F_{*}(\overrightarrow{pq})
    \end{array}
    \]
  \end{proof}
\end{st}

\begin{st}
  \label{st:samenstelling-affiene-transformatie-intern}
  De samenstelling $F\circ G$ van twee affiene transformaties $F$ en $G$ is opnieuw een affiene transformatie.
  Bovendien geldt $(F\circ G)_{*} = F_{*} \circ G_{*}$.

  \begin{proof}
    Zij $F = t_{b} \circ A$ en $G = t_{d} \circ C$ twee affiene transformaties.
    \[ (F\circ G)(p) = F(Cp + d) = A(Cp + d) + b = ACp + Ad + b \]
    $F\circ G$ is dus een affiene transformatie met lineair deel $AC$ en translatiedeel $Ad + b$.
    Tenslotte geldt dan ook het volgende:
    \[ (F\circ G)_{*} = AC = F_{*} \circ G_{*} \]
  \end{proof}
\end{st}

\begin{st}
  \label{st:samenstelling-affiene-transformatie-inverse}
  De inverse $F^{-1}$ van een affiene transformatie $F$ is opnieuw een affiene transformatie.
  Bovendien geldt $(F^{-1})_{*} = (F_{*})^{-1}$.

  \begin{proof}
    Zij $F = t_{b} \circ A$ een affiene transformatie.
    Zij $H$ de affiene transformatie met lineair deel $A^{-1}$ en translatiedeel $-A^{-1}b$.
    $H$ is nu de inverse van $F$.
    Kies namelijk een willekeurig punt $p\in \mathbb{A}^{n}$, dan geldt het volgende:
    \[
    \begin{array}{rll}
      (F \circ H)(p) &= F(A^{-1}p - A^{-1}b) &\\
                     &= (A^{-1}p - A^{-1}b)A + b &\\
                     &= A^{-1}pA - A^{-1}bA + b &\\
                     &= A^{-1}Ap - A^{-1}Ab + b &\\
                     &= p - b + b &= p
    \end{array}
    \]
    Bovendien geldt ook het volgende:
    \[ (F^{-1})_{*} = H_{*} = A^{-1} \]
  \end{proof}
\end{st} 

\begin{st}
  De verzameling $A(n,\mathbb{R}),\circ$ is een groep.

  \begin{proof}
    We bewijzen elke eigenschap van een groep.
    \begin{itemize}
    \item De bewerking $\circ$ is intern.
      Inderdaad, zie \ref{st:samenstelling-affiene-transformatie-intern}.
    \item De bewerking is associatief.
      Inderdaad, de samenstelling van affiene transformaties is associatief.
      \clarify{ waarom ? }
    \item Er bestaat een neutraal element.
      Beschouw de affiene transformatie $F = \mathbb{I}^{n} + \vec{0}$.
      Nu geldt voor elke affiene transformatie $G$ het volgende:
      \[ F \circ G = G = G \circ F \]
    \item Er bestaat voor elk element een invers element.
      Inderdaad, zie \ref{st:samenstelling-affiene-transformatie-inverse}.
    \end{itemize}
  \end{proof}
\end{st}

\begin{de}
  De groep $A(n,\mathbb{R}),\circ$ noemen we de \emph{affiene groep} in dimensie $n$.
\end{de}

\begin{st}
  \label{st:raakende-ruimtes-transformatie}
  Zij $p$ en $q$ twee punten van $\mathbb{A}^{n}$ en zijn $\{v_{1},\dotsc,v_{n}\}$ een basis van $T_{p}\mathbb{A}^{n}$ en $\{w_{1},\dotsc,w_{n}\}$ een basis van $T_{q}\mathbb{A}^{n}$, dan bestaat er een unieke affiene transformatie $F$ van $\mathbb{A}^{n}$ zodat het volgende geldt:
  \begin{itemize}
  \item $F(p) = q$
  \item $F_{*}v_{i} = w_{i}$ voor $i \in \{ 1,\dotsc,n \}$
  \end{itemize}

  \begin{proof}
    We bewijzen eerst de uniciteit en vervolgens het bestaan.
    \begin{itemize}
    \item Als er zo een unieke transformatie bestaat, is deze uniek.\\
      Stel dat er een transformatie $F = t_{b} \circ A$ bestaat die aan de voorwaarden voldoet.
      Nu is $Fv_{i}$ gelijk aan $Av_{i}$.
      Omdat elke lineaire transformatie uniek bepaal is door het beeld van een basis, is $A$ uniek bepaald.
      Vervolgens moet er een $b$ bestaan zodat de volgende gelijkheid geldt:
      \[ q = F(p) = Ap + b \]
      Die $b$ is dan gelijk aan $q-Ap$.
      Omdat $A$ uniek is is $b$ ook uniek.
    \item
      Het bestaan is nu eveneen aangetoond. 
      Kies namelijk $A$ de unieke lineaire transformatie zodat $Av_{i} = w_{i}$ geldt.
      Met andere woorden zodat de basis $V$ afgebeeldt wordt op $W$.
      Kies vervolgens $b$ zodat $b=q-Ap$ geldt, dan voldoet $F$ aan de voorwaarden.
      Nog explicieter construeren we $A$ als volgt:
      Zet de vectoren $v_{i}$ in de kolommen van een matrix $V$.
      Doe hetzelfde met de vectoren $w_{i}$ en de matrix $W$.
      Nu ziet $A$ er als volgt uit:
      \[ A = WV^{-1} \]
    \end{itemize}
  \end{proof}
\end{st}

\begin{gev}
  Zij $p$ en $q$ twee punten van $\mathbb{A}^{n}$ en zij $V = \{v_{1},\dotsc,v_{k}\}$ lineair onafhankelijke vectoren in $T_{p}\mathbb{A}^{n}$ en $W = \{w_{1},\dotsc,w_{k}\}$ lineair onafhankelijke vectoren in $T_{q}\mathbb{A}^{n}$ die aan de volgende voorwaarden voldoet.
  Er bestaat dan een (niet noodzakelijk unieke) affiene transformatie $F$ van $\mathbb{A}^{n}$.
  \begin{itemize}
  \item $F(p) = q$
  \item $F_{*}v_{i} = w_{i}$ voor $i \in \{ 1,\dotsc,n \}$
  \end{itemize}

  \begin{proof}
    Omdat $T_{p}\mathbb{A}^{n}$ en $T_{q}\mathbb{A}^{n}$ $n$ dimensies hebben, kan zowel $V$ als $W$ uitgebreid worden tot een basis van $T_{p}\mathbb{A}^{n}$, respectievelijk $T_{q}\mathbb{A}^{n}$.
    Zij $\alpha$ en $\beta$ die basissen.
    Nu bestaat er dus een unieke affiene transformatie voor de specifieke $\alpha$ en $\beta$ die aan de voorwaarden voldoet,\footnote{Zie stelling \ref{st:raakende-ruimtes-transformatie}.}
    Merk op dat de $\alpha$ en $\beta$ niet uniek zijn.
  \end{proof}
\end{gev}

\begin{gev}
  Zij $\{p_{0},\dotsc,p_{k}\}$ en $\{q_{0},\dotsc,q_{k}\}$ twee verzamelingen van $k+1$ affien onafhankelijke punten in $\mathbb{A}^{n}$, dan bestaat er een affiene transformatie $F$ van $\mathbb{A}^{n}$ zodat $F(p_{i}) = q_{i}$ voor $i \in \{ 1,\dotsc,k \}$. Bovendien is $F$ uniek als $k=n$.

  \begin{proof}
    Beschouw de verzamelingen $V = \{\overrightarrow{p_{0}p_{1}},\dotsc,\overrightarrow{p_{0}p_{k}}\}$ en $W = \{\overrightarrow{q_{0}q_{1}},\dotsc,\overrightarrow{q_{0}q_{k}}\}$.
    $V$ en $W$ bevatten nu elk $k$ lineair onafhankelijke vectoren.\footnote{Zie stelling \ref{st:criterium-affiene-afhankelijkheid}.}
    Er bestaat nu dus een affiene transformatie $F$ die voldoet aan deze voorwaanden:
    \begin{itemize}
    \item $F(p_{0}) = q_{0}$
    \item $F_{*}(\overrightarrow{p_{0}p_{i}}) = \overrightarrow{q_{0}q_{i}}$ voor $i \in \{ 1,\dotsc,n \}$
    \end{itemize}
    Bekijk nu $F(p_{i})$.
    \[
    \begin{array}{rll}
      F(p_{i}) &= F(p_{0}) + \overrightarrow{F(p_{0})F(p_{i})} &\\
               &= F(p_{0}) + F(\overrightarrow{p_{0}p_{i}}) &\\
               &= q_{0} + \overrightarrow{q_{0}q_{i}} &= q_{i}
    \end{array} 
    \]
    Diezelfde affiene transformatie is dus de affiene transformatie die $p_{i}$ op $q_{i}$ afbeeldt.
    Als $k$ gelijk is aan $n$ is deze transformatie bovendien uniek.\footnote{Zie stelling \ref{st:raakende-ruimtes-transformatie}.}
   \end{proof}
\end{gev}


\section{Affiene invarianten}
\label{sec:affiene-invarianten}

\begin{de}
  We noemen een begrip of eigenschap \emph{affien invariant} als ze bewaard blijven onder affiene transformaties.
\end{de}

\begin{st}
  \label{st:affiene-transformatie-deelruimte-invariant}
  Zij $F = t_{b} \circ A$ een affiene transformatie.
  Zij $S = p+V$ een affiene deelruimte van $\mathbb{A}^{n}$ met dimensie $k$, dan is $F(S)$ ook een affiene deelruimte van $\mathbb{A}^{n}$ met dimensie $k$.
  Sterker nog:
  \[ F(S) = F(p) + F_{*}(V) \]

  \begin{proof}
    Kies een willekeurig punt $q= p+v$ van $S$:
    \[
    \begin{array}{rll}
    F(s) &= A(p+v) + b &\\
         &= Ap + b + Av &= F(p) + F_{*}(v)      
    \end{array}
    \]
  \end{proof}
\end{st}

\begin{st}
  Zij $F = t_{b} \circ A$ een affiene transformatie.
  Zij $S= p+V$ en $T= q+W$ twee affiene deelruimten, dan is de onderlinge ligging van $F(S) = F(p) + F_{*}(V)$ en $F(T) = F(q) + F(W)$ gelijk aan de onderlinge ligging van $S$ en $T$.  

  \begin{proof}
    We maken een gevalsonderscheid voor de verschillende mogelijkheden van onderlinge ligging.
    \begin{itemize}
    \item Stel dat $S$ en $T$ parallel zijn:
      \[ S \parallel T \Leftrightarrow V = W \Leftrightarrow F_{*}(V)=F_{*}(W) \Leftrightarrow F(S) \parallel F(T) \]
    \item Stel dat $S$ en $T$ zwak parallel zijn:
      \[ S \triangleleft T \Leftrightarrow V \subseteq W \Leftrightarrow F_{*}(V)\subseteq F_{*}(W) \Leftrightarrow F(S) \triangleleft F(T) \]
    \item Stel dat $S$ en $T$ snijden, dan bestaat er een punt $x$ in de doorsnede van $S$ en $T$.
      Er bestaan dus een $v\in V$ en een $w \in W$ zodat het volgende geldt:
      \[
      \begin{array}{rrll}
                      & p + v  &= x   &= q + w\\
         \Rightarrow  & F(p+v) &=F(x) &= F(q+w)\\
         \Rightarrow  & F(p)+F_{*}(v) &= F(x) &= F(q)+F_{*}(w)\\
      \end{array}
      \]
    \item Stel dat $S$ en $T$ kruisen, dan bestaat er geen punt $x$ in de doorsnede, en zijn $S$ en $T$ niet zwak parallel.
      Omdat $S$ en $T$ niet parallel of zwak parallel zijn, zijn $F(S)$ en $F(T)$ niet parallel, noch zwak parallel.
      Stel nu dat er een punt $x$ in de doorsnede van $F(S)$ en $F(T)$ zit zodat $F(S)$ en $F(T)$ snijden, dan sneden $S$ en $T$.
      Contradictie.
    \end{itemize}
  \end{proof}
\end{st}

\begin{st}
  Zij $F = t_{b} \circ A$ een affiene transformatie.
  $p_{0},\dotsc,p_{k} \in \mathbb{A}^{n}$ zijn affien onfhankelijk als en slechs als $F(p_{0}),\dotsc,F(p_{k}) \in \mathbb{A}^{n}$ affien onafhankelijk zijn.

  \begin{proof}
     Deze stelling is equivalent met de volgende: ``$p_{0},\dotsc,p_{k} \in \mathbb{A}^{n}$ zijn affien afhankelijk als en slechs als $F(p_{0}),\dotsc,F(p_{k}) \in \mathbb{A}^{n}$ affien afhankelijk zijn.''
     Kies dus $k+1$ affien afhankelijke punten $p_{0},\dotsc,p_{k}$ van $\mathbb{A}^{n}$.
     Dit betekent dat de vectoren $\overrightarrow{p_{0}p_{i}}$ lineair onafhankelijk zijn. \footnote{Zie stelling \ref{st:critirium-affien-afhankelijk}.}
     De affiene transformatie $F$ gedraagt zich voor vectoren als een lineaire transformatie.
     Lineaire transformaties zijn lineair\footnote{duh!} en behouden bijgevolg lineaire afhankelijkheid.
     De vectoren $\overrightarrow{F(p_{0}p_{i})}$ zijn dus ook lineair afhankelijk.
     Dit houdt precies in dat de punten $F(p_{0}),\dotsc,F(p_{k})$ affien onafhankelijk zijn.
  \end{proof}
\end{st}


\begin{st}
  Zij $p$, $q$ en $r$ punten op een rechte $L$, dan zijn $F(p)$, $F(q)$ en $F(r)$ punten op de rechte $F(L)$ en is bovendien de deelverhouding $(p,q,r)$ gelijk aan $(F(p),F(q),F(r))$.
  \[ 
  \overrightarrow{pr} = (q,p,r) \overrightarrow{pq} \Rightarrow \overrightarrow{F(p)F(r)} = (q,p,r) \overrightarrow{F(p)F(q)}
  \]
  \begin{proof}
    Als $p$, $q$ en $r$ colineair zijn, dan bestaat er een $\lambda$ zodat we $r$ kunnen schrijven als volgt:
    \[ r = p + \lambda(\overrightarrow{pq}) \]
    Beelden we nu $r$ af onder $F$ dan krijgen we volgende gelijkheid:
    \[ F(r) = F(p +\lambda(\overrightarrow{pq})) = F(p) + \lambda(\overrightarrow{F(p)F(q)}) \]
\clarify{Waarom geldt die laatste gelijkheid?}
    Dit betekent precies dat $F(p)$, $F(q)$ en $F(r)$ colineair zijn.
    Bekijken we nu de deelverhouding $(q,p,r)$ en beelden we ook de vergelijking daarvan af onder $F$, dan krijgen we volgende gelijkheid:
    \[
    \begin{array}{rrl}
                  & F(\overrightarrow{pr}) &= F((q,p,r) \overrightarrow{pq})\\
      \Rightarrow & \overrightarrow{F(p)F(r)} &= (q,p,r) \overrightarrow{F(p)F(q)}
    \end{array}
    \]
\clarify{Waarom geldt die laatste overgang?}
  \end{proof}
\end{st}

\begin{st}
  Zij $F:\ \mathbb{A}^{n} \rightarrow \mathbb{A}^{n}$ met $n\ge 2$ een bijectie die collineaire punten afbeeldt ap colineaire punten, dan is $F$ een affiene transformatie.

Geen bewijs
\TODO{Toch eens een bewijs proberen?}
\end{st}

\section{Dilataties}
\label{sec:dilataties}

\begin{de}
  Een affiene transformatie $F$ van $\mathbb{A}^{n}$ wordt een \emph{dilatatie} genoemd als en slechts als er een $\lambda\in \mathbb{R}_{0}$ bestaat zodat het volgende geldt.
  \[ F_{*} = \lambda  \mathbb{I} \]
  \[ F:\ p \mapsto \lambda \mathbb{I} p + b \]
\end{de}

\begin{st}
  \label{st:samenstelling-dilataties-dilatatie}
  De samenstelling van twee dilataties $F$ en $G$ is een dilatatie.
  
  \begin{proof}
    Zij $F=\lambda I$ en $G= \mu I$ twee dilataties.
    \[ (G \circ F)_{*} = \mu \mathbb{I}  \cdot \lambda \mathbb{I} = (\mu\lambda) \mathbb{I} \]
    Noem $\mu\lambda$ nu $\nu$, dan zien we dat $G \circ F$ een dilatatie is.
  \end{proof}
\end{st}

\begin{de}
  De verzameling van dilataties van een affiene ruimte $\mathbb{A}^{n}$ wordt genoteerd als $Dil(\mathbb{A}^{n})$.
\end{de}

\begin{st}
  De verzameling van dilataties vormt een deelgroep van de verzameling affiene transformaties met de samenstelling als bewerking.

  \begin{proof}
    We bewijzen elk deel van het criterium van een deelgroep.
    \begin{itemize}
    \item Het neutraal element van $A(n,R)$: $e_{A(n,\mathbb{R})}:\ p \mapsto \mathbb{I} p + \vec{0}$ is een dilatatie.
    \item De samenstelling van twee dilataties is een dilatatie.\footnote{Zie stelling \ref{st:samenstelling-dilataties-dilatatie}.}
    \item De inverse van een dilatatie is een dilatatie.\\
      Kies een dilatatie $F_{*} = \lambda \mathbb{I} $. De inverse affiene transformatie $G$ van $F$ is ook een dilatatie:
      \[ G_{*} = \frac{1}{\lambda} \mathbb{I}\]
    \end{itemize}
  \end{proof}
\end{st}

\begin{st}
  Een affiene transformatie $F$ van $\mathbb{A}^{n}$ is een dilatatie als en slechts als $F$ ofwel een translatie is, ofwel een homothetie.

  \begin{proof}
    Bewijs van een equivalentie\\
    \begin{itemize}
    \item $\Rightarrow$\\
      Zij $F$ een dilatatie, dan is $F$ gegeven als volgt:
      \[ F:\ p \mapsto \lambda p + b\]
      Als $\lambda$ $1$ is, is $F$ een translatie.
      Als $\lambda$ niet gelijk is aan $1$, kies dan het punt $p_{0} = \frac{1}{1-\lambda}b$.
      $F$ is dan een homothetie: $H_{p_{0},\lambda}$
      \[\forall p\in \mathbb{A}^{n}:\ F(p) = \lambda p + b = \lambda p + (1 - \lambda) p_{0} = p_{0} + \lambda \overrightarrow{p_{0}p} = H_{p_{0},\lambda} \]
    \item $\Leftarrow$\\
      Zij $F$ een translatie $F:\ p \mapsto p + b$, dan is $F$ een dilatatie met $\lambda = 1$.
      Zij $F$ een homothetie $F = H_{p_{0},r}$, dan is $F$ een dilatatie met $\lambda = r$.
    \end{itemize}
  \end{proof}
\end{st}

\begin{st}
  De afbeelding van een affiene deelruimte $G$ van $\mathbb{A}^{n}$ onder een dilatatie is een deelruimte $H$ die parallel is met $G$.

  \begin{st}
    Zij $G = p + V$ een deelruimte van $\mathbb{A}^{n}$ en $F:\ p \mapsto \lambda p + b$ een dilatatie.
    De afbeelding $H$ van $G$ onder $F$ is ook een deelruimte van $\mathbb{A}^{n}$.\footnote{Zie stelling \ref{st:affiene-transformatie-deelruimte-invariant}.}
    Voor elk willekeurig punt $x$ van $g$ bestaat er een $v\in V$ zodat $x = p+v$ geldt.
    \[ F(x) = \lambda(p+v) + b = (b + \lambda p) + \lambda v \in G \]
    Elke afbeelding van een punt $x= p+v$ heeft dus een vrije vector $\lambda v$ die lineair afhankelijk is van $v$ en dus ook in $V$ zit.
    Omdat $F$ een bijectie is, moet $F(V) = V$ gelden en is $H$ dus parallel met $G$.
    \[ H \parallel G \]
  \end{st}
\end{st}

\begin{st}
  Een affiene transformatie die elke affiene deelruimte $G$ van $\mathbb{A}^{n}$ afbeeldt op een parallelle affiene deelruimte is een dilatatie.

\TODO{bewijs}
\end{st}

\begin{lem}
  Zij $F = t_{b}$ een translatie en $p$, $q$ twee verschillende punten op een rechte $L$ in $\mathbb{A}^{n}$ die niet in de richting van $b$ ligt.
  Noem nu $p' = F(p)$ en $q'= F(q)$ en noem $L'$ de rechte door $p'$ parallel met $L$.
  Noem bovendien $T$ de rechte door $p$ en $p'$ en $T'$ de rechte door $q$ parallel met $T$.

  Nu snijden $L'$ en $T'$ enkel in $q$.
  \[ L' \cap T' = \{q\} \]

\TODO{bewijs p 48}
\end{lem}

\begin{lem}
  Zij $F = H_{p_{0},r}$ een homothetie en $p$ en $q$ twee verschillende punten op een rechte $L$ in $\mathbb{A}^{n}$.
  Noem nu $p' = F(p)$ en $q'= F(q)$ en noem $L'$ de rechte door $p'$ parallel met $L$.
  Noem bovendien $T$ de rechte door $p_{0}$ en $p$, en $T'$ de rechte door $p_{0}$ en $q$.
  
  Nu snijden $L'$ en $T'$ enkel in $q$.
  \[ L' \cap T' = \{q\} \]

\TODO{bewijs}
\end{lem}

\end{document}
