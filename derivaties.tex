\documentclass[main.tex]{subfiles}
\begin{document}

\chapter{Derivaties}
\label{cha:derivaties}

\begin{de}
  Zij $\mathcal{F}(\mathbb{A}^{n})$ alle functies van $\mathbb{A}^{n}$ naar $\mathbb{R}$ waarvoor alle parti\"ele afgeleiden van willekeurige orde bestaan en overal continu zijn.
  \[ \{ f: \mathbb{A}^{n} \rightarrow \mathbb{R} \ |\ f \text{ is } C^{\infty} \} \]
  $\mathcal{F}(\mathbb{A}^{n})$ is bovendien een re\"ele vectorruimte.
\end{de}

\begin{de}
  Een \term{derivatie} in $p\in\mathbb{A}^{n}$ is een afbeelding $D_{p}$ waarop twee bewerkingen gedefinieerd zijn met bovendien de volgende twee eigenschappen.
  \[ D_{p}:\ \mathcal{F}(\mathbb{A}^{n}) \rightarrow \mathbb{R} \]
  \begin{itemize}
  \item $D_{p}$ is een lineaire afbeelding van $\mathcal{F}(\mathbb{A}^{n})$ naar $\mathbb{R}$:
    \[ \forall a,b \in \mathbb{R}, \forall f,g \in \mathcal{F}(\mathbb{A}^{n}):\ D_{p}(af + bg) = aD_{p}(f) + bD_{p}(g) \]
  \item Leibnitzregel: 
    \[ \forall f,g \in \mathcal{F}(\mathbb{A}^{n}):\ D_{p}(fg) = D_{p}(f)g(p) + f(f)D_{p}(g)\]
  \end{itemize}
\end{de}

\begin{ei}
  De verzameling $T_{p}\mathbb{A}^{n}$ van alle derivaties in een punt $p$ vormt een vectorruimte met de volgende bewerkingen.
  \begin{itemize}
  \item Optelling:
    \[ (D_{p} + D'_{p}):\ \mathcal{F}(\mathbb{A}^{n}) \rightarrow \mathbb{R}:\ f \mapsto (D_{p} + D'_{p})(f) = D_{p}(f) + D'_{p}(f)\]
  \item Scalaire vermenigvuldiging:
    \[ aD_{p}:\ \mathcal{F}(\mathbb{A}^{n}) \rightarrow \mathbb{R}:\ f \mapsto (aD_{p})(f) = aD_{p}(f)\]
  \end{itemize}
\end{ei}

\begin{de}
  Een \term{open bol} $B_{p,\epsilon}$ met middelpunt $p$ en straal $\epsilon$ is een verzameling punten als volgt:
  \[ B_{p,\epsilon} = \left\{ (x_{1},\dotsc ,x_{n})\in \mathbb{A}^{n}\ \left|\ \sum_{i=1}^{n}(x_{i}-p_{i})^{2} < \epsilon^{2} \right.\right\}\]
\end{de}

\begin{de}
  Een deelverzameling $U \subseteq \mathbb{A}^{n}$ noemen we een \term{omgeving} van $p$ als er een open bol met middelpunt $p$ bestaat die een deelverzameling is van $U$.
\end{de}

\begin{lem}
  Zij $D_{p}$ een derivatie in $p\in\mathbb{A}^{n}$ en $f\in\mathcal{F}(A^{n})$ een functie.
  Als $f$ constant is, dan is de derivatie van $f$ in $p$ nul.
  \[ (\exists c\in\mathbb{R},\ \forall q\in\mathbb{A}^{n}:\ f(q) = c) \quad\Rightarrow\quad D_{p}(f) = 0 \]

  \begin{proof}
    Kies een willekeurige $g\in \mathcal{F}(\mathbb{A}^{n})$ met $g(p) \neq 0$.
    Wegens de lineariteit van $D_{p}$ geldt volgende bewering:
    \[ D_{p}(fg) = D_{p}(f(p)g) = f(p)D_{p}(g) \]
    Omwille van de Leibnizregel geldt het volgende echter ook:
    \[ D_{p}(fg) = D_{p}(f)g(p) + f(p)D_{p}(g) \]
    Combineren we nu deze twee redeneringen, dan krijgen we het volgende:
    \[ f(p)D_{p}(g) = D_{p}(f)g(p) + f(p)D_{p}(g) \Rightarrow D_{p}(f)g(p) = 0 \]
    Omdat $g(p)$ niet nul is, moet $D_{p}(f)$ nul zijn.
  \end{proof}
\end{lem}

\begin{lem}
  Zij $D_{p}$ een derivatie in $p\in\mathbb{A}^{n}$ en $f\in\mathcal{F}(A^{n})$ een functie.
  Als $f$ nul is in een omgeving van $p$, dan is de derivatie van $f$ in $p$ nul.
  \[ (\exists \epsilon\in\mathbb{R}^{+}_{0},\ \forall q \in B(p,\epsilon):\ f(q) = 0) \quad\Rightarrow\quad D_{p}(f) = 0 \]

  \begin{proof}
    Per definitie bestaat er een open bol $B(p,\epsilon) \in U$. Binnen die open bol is $f$ nog steeds identiek nul.
    Beschouw nu de functie $g$ die buiten $B(p,\epsilon)$ identiek $1$ is, maar ook $0$ in $p$.
    $fg$ is dan gelijk aan $f$.
    Nu geldt er daarom ook het volgende:
    \[ D_{p}(f) = D_{p}(fg) = D_{p}(f)g(p) + f(p)D_{p}(g) = 0 \]
  \end{proof}
\end{lem}

\begin{lem}
  Zij $D_{p}$ een derivatie in $p\in\mathbb{A}^{n}$ en $f\in\mathcal{F}(A^{n})$ een functie.
  $D_{p}(f)$ hangt enkel af van de waarden van $f$ in een (willekeurige) omgeving van $p$.
  \[ \forall \epsilon\in\mathbb{R}^{+}_{0},\ \forall q \in B(p,\epsilon),\ \forall f_{1},f_{2}\in\mathcal{F}(\mathbb{A}^{n}):\ f_{1}(q) = f_{2}(q) \]

  \begin{proof}
    Stel dat twee functies $f_{1}$ en $f_{2}$ uit $\mathcal{F}(\mathbb{A}^{n})$ gelijk zijn in een omgeving $U$ van $p$, dan is $f_{1}-f_{2}$ nul op $U$.
    \[ D_{p}(f_{1}-f_{2}) = 0 \Rightarrow D_{p}(f_{1}) = D_{p}(f_{2}) \]
  \end{proof}
\end{lem}

\begin{lem}
  Zij $f \in \mathcal{F}(\mathbb{A}^{n})$ een functie en $p = (p_{1},\dotsb,p_{n}) \in \mathbb{A}^{n}$ een punt, dan bestaan er $n$ functies $f_{1},\dotsc,f_{n} \in \mathcal{F}(\mathbb{A}^{n})$ zodat het volgende geldt.
  \[
  \forall x = (x_{1},\dotsc,x_{n})\in\mathbb{A}^{n}:\ f(x) = f(p) + \sum_{i=i}^{n}f_{i}(x)(x_{i}-p_{i})
  \]
  \TODO{bewijs p 64}
\end{lem}

\begin{st}
  Zij $p\in\mathbb{A}^{n}$ een punt.
  De partiele afgeleiden in $p$ vormen een basis voor de ruimte der derivaties in $p$.
  \[ \left\{ \left(\frac{\partial}{\partial x_{1}}\right)_{p}, \left(\frac{\partial}{\partial x_{2}}\right)_{p}, \dotsc , \left(\frac{\partial}{\partial x_{n}}\right)_{p}\right\}  \]
  \TODO{bewijs p 64 - 65}
\end{st}

\begin{opm}
  Het verband tussen de definitie voor een rakende ruimte $T_{p}\mathbb{A}^{n}$ uit hoofdstuk \ref{cha:affiene-meetkunde} en de definitie voor een rakende ruimte als ruimte van derivaties in $p$ wordt gegeven door de volgende identificatie:
  \[ 
  (0,\dotsc,0,1,0,\dotsc,0)_{p}
  \leftrightarrow
  \left(\frac{\partial}{\partial x_{i}}\right)_{p}
  \]
  Hier staat de $1$ op de $i$-de plaats.
  In woorden identificeert elke raakvector dus met de richtingsafgeleide langs die vector.
  Zij immers $p\in\mathbb{A}^{n}$ een punt en $v_{p} = (v_{1},\dotsc,v_{n})_{p} \in T_{p}\mathbb{A}^{n}$ een raakvector.
  Beschouw dan de richtingsafgeleide van $f$ langs $v_{p}$:
  \[ v_{p}(f) = \left.\frac{d}{dt}f(p+tv)\right|_{t=0} = \sum_{i=1}^{n}v_{i}\frac{\partial f}{\partial x_{i}}(p)\]
\end{opm}

\section{Afgeleide afbeelding}
\label{sec:afgeleide-afbeelding}

\begin{de}
  We noemen een deelverzameling $U$ van $\mathbb{A}^{n}$ een \term{open deelverzameling} als er voor elke $p\in U$ een open bol $B(p,\epsilon)$ bestaat die volledig in $U$ ligt.
\end{de}

\begin{de}
  Zij $F$ een afbeelding van een deelverzameling $U$ van $\mathbb{A}^{n}$ naar $\mathbb{A}^{m}$ als volgt, dan noemen we de $F_{i}$ de \term{co\"ordinaatsfuncties} van $F$.
  \[ F: U\subseteq \mathbb{A}^{n} \rightarrow \mathbb{A}^{m}:\ p\mapsto F(p) = (F_{1}(p),\dotsc,F_{m}(p)) \]  
  We noemen $F$ differentieerbaar als al de $F_{i}$ $C^{\infty}$-continu zijn. 
\end{de}

\begin{de}
  Zij $F$ een differentieerbare afbeelding en $p \in U$ een punt.
  \[ F: U\subseteq \mathbb{A}^{n} \rightarrow \mathbb{A}^{m}:\ p\mapsto F(p) = (F_{1}(p),\dotsc,F_{m}(p)) \]
  De Jacobiaan van $F$ in $p$ is de matrix $\mathcal{J}$.
  \[
  \mathcal{J} = 
  \begin{pmatrix}
  \frac{\partial F_{1}}{\partial x_{1}}(p) & \hdots & \frac{\partial F_{1}}{\partial x_{n}}(p)\\
  \vdots & & \vdots\\
  \frac{\partial F_{m}}{\partial x_{1}}(p) & \hdots & \frac{\partial F_{m}}{\partial x_{n}}(p)\\
  \end{pmatrix}
  \]  
\end{de}

\begin{de}
  Zij $F$ een differentieerbare afbeelding en $p \in U$ een punt.
  \[ F:\ U\subseteq \mathbb{A}^{n} \rightarrow \mathbb{A}^{m}:\ p\mapsto F(p) = (F_{1}(p),\dotsc,F_{m}(p)) \]
  We defini\"eren de \term{afgeleide afbeelding} van $F$ in $p$ als de lineaire afbeelding $(F_{*})_{p}$:
  \[
  (F_{*})_{p}:\ T_{p}\mathbb{A}^{n}\rightarrow T_{F(p)}\mathbb{A}^{m}:\
  \begin{pmatrix}
    v_{1}\\v_{2}\\\vdots\\v_{n}
  \end{pmatrix}_{p}
  \mapsto
  \left(
  \mathcal{J}
  \begin{pmatrix}
    v_{1}\\v_{2}\\\vdots\\v_{n}
  \end{pmatrix}
  \right)_{p} 
  \]
\end{de}

\begin{lem}
  Zij $F$ een affiene transformatie.
  \[ F:\ \mathbb{A}^{n} \rightarrow \mathbb{A}^{n}: p\mapsto Ap + b \]
  De afgeleide afbeelding van $F$ in $p$ is gegeven door $(F_{*})_{p}$:
  \[ (F_{*})_{p}: T_{p}\mathbb{A}^{n} \rightarrow T_{F(p)}\mathbb{A}^{n}:\ v_{p} \mapsto (Av)_{F(p)} \]
  \TODO{bewijs p 68}
\end{lem}

\begin{st}
  Zij $F:\ U\subseteq \mathbb{A}^{n} \rightarrow \mathbb{A}^{m}$ een differentieerbare afbeelding en $p$ een elemen van $U$.
  Als $D_{p}$ een derivatie in $p$ is, dan is $(F_{*})_{p}(D_{p})$ de derivatie in $F(p)$.
  \[ \left( (F_{*})_{p}(D_{p})\right) (f) = D_{p} (f\circ F) \]
  \TODO{bewijs p 68}
\end{st}

\end{document}
