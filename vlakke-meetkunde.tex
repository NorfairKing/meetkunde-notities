\documentclass[main.tex]{subfiles}
\begin{document}

\chapter{Vlakke meetkunde}
\label{cha:vlakke-meetkunde}

\section{Ceva en Menelaos}
\label{sec:ceva-en-menelaos}

\begin{de}
  Een aantal rechten $L_{1},\dotsc,L_{k}$ noemen we \emph{concurrent} als de doorsnede van al de rechten niet leeg is.
  \[ L_{1} \cap \dotsb L_{k} \neq \emptyset \]
\end{de}

\begin{lem}
  \label{lem:rechten-concurrent-determinant}
  Zij $L_{1}$, $L_{2}$ en $L_{3}$ drie rechten in $\mathbb{A}^{2}$ met de volgende vergelijkingen:
  \[ L_{i} \leftrightarrow a_{i1}x_{1} + a_{i2}x_{2} = a_{i3} \]
  De rechten zijn evenwijdig of concurrent als en slechts als de volgende determinant nul is:
  \[
  \begin{vmatrix}
    a_{11} & a_{12} & a_{13}\\
    a_{21} & a_{22} & a_{23}\\
    a_{31} & a_{32} & a_{33}
  \end{vmatrix}
  = 0
  \]

  \begin{proof}
    De drie rechten $L_{1}$, $L_{2}$ en $L_{3}$ hebben een niet-lege doorsnede als en slechts als het volgende geldt:
    \clarify{waarom?}
    \[
    rang
    \begin{pmatrix}
    a_{11} & a_{12} & a_{13}\\
    a_{21} & a_{22} & a_{23}\\
    a_{31} & a_{32} & a_{33}
    \end{pmatrix}
    =
    rang
    \begin{pmatrix}
    a_{11} & a_{12}\\
    a_{21} & a_{22}\\
    a_{31} & a_{32}
    \end{pmatrix}
    \]
    Bovendien zijn de rechten evenwijdig als en slechts als de richtingsgetallen $(a_{11} a_{12})$ $(a_{21} a_{22})$ en $(a_{31}, a_{32})$ evenredig zijn.\footnote{Zie stelling \ref{st:parallelle-hypervlakken-gelijke-richtingsgetallen}.}
    \[
    rang
    \begin{pmatrix}
    a_{11} & a_{12}\\
    a_{21} & a_{22}\\
    a_{31} & a_{32}
    \end{pmatrix}
    = 1
    \]
    \begin{itemize}
    \item $\Rightarrow$\\
      Als de drie rechten $L_{1}$, $L_{2}$ en $L_{3}$ evenwijdig of concurrent zijn, dan is de volgende rang gelijk aan $1$ of aan $2$. 
      \[
      rang
      \begin{pmatrix}
        a_{11} & a_{12}\\
        a_{21} & a_{22}\\
        a_{31} & a_{32}
      \end{pmatrix}
      \]
      De determinant is dus $0$.
      \[
      \begin{vmatrix}
        a_{11} & a_{12} & a_{13}\\
        a_{21} & a_{22} & a_{23}\\
        a_{31} & a_{32} & a_{33}
      \end{vmatrix}
      = 0
      \]
    \item $\Leftarrow$\\
      Als de determinant $0$ is, is de rang van deze matrix kleiner dan $3$.
      \[
      rang
      \begin{pmatrix}
        a_{11} & a_{12}\\
        a_{21} & a_{22}\\
        a_{31} & a_{32}
      \end{pmatrix}
      < 3
      \]
      Er zijn dus twee mogelijkheden. Ofwel is de rang $2$ en zijn de rechten concurrent, ofwel is de rang $1$ en zijn de rechten evenwijdig.
    \end{itemize}
  \end{proof}
\end{lem}

\begin{st}
  Stelling van \term{Ceva}\\
  Zij $a$, $b$ en $c$ drie affien onafhankelijke punten in $\mathbb{A}^{2}$ en zij $a'$ een punt op de rechte $bc$, verschillend van $b$, $b'$ een punt op de rechte $ac$ verschillend van $c$ en $c'$ een punt op de rechte $ab$, verschillend van $a$.
  Noem nu $A = aa'$, $B = bb'$ en $C=cc'$.
  $A$, $B$ en $C$ zijn concurrent als en slechts als volgende gelijkheid geldt:
  \[ (a',b,c)(b',c,a)(c',a,b) = 1\]
  \begin{figure}[H]
    \centering
    \begin{tikzpicture}[scale=2] 
      \coordinate [label=below:$a$] (a) at (0,0);
      \coordinate [label=above:$b$] (b) at (1,2);
      \coordinate [label=below:$c$] (c) at (4,0);
      \draw (a) -- (b) -- (c) -- cycle;

      \coordinate [label=above:$a'$] (ap) at ($ (b)!.5!(c) $);      
      \coordinate [label=below:$b'$] (bp) at ($ (c)!.5!(a) $);    
      \coordinate [label=above:$c'$] (cp) at ($ (a)!.5!(b) $);    

      \draw (ap) -- (a);
      \draw (bp) -- (b);
      \draw (cp) -- (c);
    \end{tikzpicture}    
    \caption{Stelling van Ceva voor de transformatie}
    \label{fig:stelling-van-ceva-voor}
  \end{figure}

  \begin{proof}
    Er bestaat een affiene transformatie die $a$ op de oorsprong $(0,0)$ afbeeldt, $b$ op $(0,1)$ en $c$ op $(1,0)$.
    Omdat de ligging van de punten $a'$, $b'$, $c'$ ten opzichte van de rechten affien invariant is, alsook de deelverhoudingen, volstaat het om de stelling te bewijzen voor de getransformeerde punten.
    \clarify{waarom?}
    Dan bestaan er dus een $\lambda$, $\mu$ en $\nu$ uit $\mathbb{R}$ zodat de punten $a'$, $b'$ en $c'$ als volgt getransformeerd worden.
    \[
    \begin{array}{cl}
      a' &= (\lambda,1-\lambda)\\
      b' &= (\mu, 0)\\
      c' &= (0,\nu)\\
    \end{array}
    \]
    \begin{figure}[H]
      \centering
      \begin{tikzpicture}[scale=5] 
        \coordinate [label=below:$a$] (a) at (0,0);
        \coordinate [label=above:$b$] (b) at (0,1);
        \coordinate [label=below:$c$] (c) at (1,0);
        \draw (a) -- (b) -- (c) -- cycle;
        
        \coordinate [label=above:$a'$] (ap) at ($ (b)!.5!(c) $);      
        \coordinate [label=below:$b'$] (bp) at ($ (c)!.5!(a) $);    
        \coordinate [label=left:$c'$] (cp) at ($ (a)!.5!(b) $);    
        
        \draw (ap) -- (a);
        \draw (bp) -- (b);
        \draw (cp) -- (c);
      \end{tikzpicture}    
      \caption{Stelling van Ceva na de transformatie}
      \label{fig:stelling-van-ceva-na}
    \end{figure}
  \end{proof}
  Omdat we extra informatie hebben over de punten $a'$, $b'$ en $c'$, namelijk dat $a'$ niet gelijk is aan $b$, $b'$ niet gelijk is aan $c$ en $c'$ niet gelijk is aan $a$ weten we het volgende over $\lambda$, $\mu$ en $\nu$.
  \[
  \left\{
  \begin{array}{rcl}
    \lambda &\neq & 0\\ 
    \mu &\neq & 1\\
    \nu &\neq & 0
  \end{array}
  \right.
  \]
  De vergelijkingen van $A$, $B$ en $C$ zijn dan de volgende:
  \[
  \begin{array}{rclcl}
  A &\leftrightarrow &
  \begin{vmatrix}
    x_{1} & x_{2} & 1\\
    0 & 0 & 1\\
    \lambda & 1-\lambda & 1
  \end{vmatrix}
  = 0
  &\Leftrightarrow&
  (1-\lambda)x_{1} - \lambda x_{2} = 0
\\
  B &\leftrightarrow &
  \begin{vmatrix}
    x_{1} & x_{2} & 1\\
    0 & 1 & 1\\
    \mu & 0 & 1
  \end{vmatrix}
  = 0
  &\Leftrightarrow&
  x_{1} - \mu x_{2} = \mu
\\
  C &\leftrightarrow &
  \begin{vmatrix}
    x_{1} & x_{2} & 1\\
    1 & 0 & 1\\
    0 & \nu & 1
  \end{vmatrix}
  = 0
  &\Leftrightarrow&
  \nu x_{1} - x_{2} = \nu
  \end{array}
  \]
  $A$, $B$ en $C$ zijn nu concurrent als en slechts als de volgende determinant $0$ is.\footnote{Zie stelling \ref{lem:rechten-concurrent-determinant}.}
  \[
  \begin{vmatrix}
    (1-\lambda) & -\lambda & 0 \\
    1 & \mu & \mu \\
    \nu & 1 & \nu
  \end{vmatrix}
  = 0
  \quad\Leftrightarrow\quad
  (1 - \lambda)\mu(\nu-1) + \lambda(\mu-1)\nu = 0
  \]
  We bekijken nu de deelverhoudingen:
  \[
  \left\{
  \begin{array}{rlll}
  (a',b,c) &= \dfrac{\overrightarrow{a'c}}{\overrightarrow{a'b}} &= \dfrac{1-\lambda,\lambda-1}{\lambda,\lambda} &= \dfrac{\lambda-1}{\lambda}\\
  (b',c,a) &= \dfrac{\overrightarrow{b'a}}{\overrightarrow{b'c}} &= \dfrac{-\mu,0}{1-\mu,0} &= \dfrac{\mu}{\mu-1}\\
  (c',a,b) &= \dfrac{\overrightarrow{c'b}}{\overrightarrow{c'a}} &= \dfrac{0,1-\nu}{0,\nu} &= \dfrac{\nu -1}{\nu}\\
  \end{array}
  \right.
  \]
  Deze impliceren dezelfde voorwaarde:
  \[
  \dfrac{\lambda-1}{\lambda}\dfrac{\mu}{\mu-1}\dfrac{\nu -1}{\nu} = -1
  \Leftrightarrow
  (1 - \lambda)\mu(\nu-1) + \lambda(\mu-1)\nu = 0
  \]
  Bijgevolg is de stelling bewezen.
\clarify{waarom?}
\end{st}

\begin{opm}
  De voorwaarde dat $a' \neq b$, $b' \neq c$ en $c' \neq a$ gelden zorgen ervoor dat de deelverhoudingen goed gedefinieerd zijn.
  Stel bijvoobeeld dat $a'=b$ geldt (zoals in figuur \ref{fig:stelling-van-ceva-alternatief}), dan is $\lambda$ nul en $(a',b,c) = \frac{\lambda-1}{\lambda}$ bijgevolg niet gedefinieerd.
  Als de rechten toch evenwijdig of concurrent zijn, dan moet \'e\'en van de andere twee deelverhoudingen nul zijn.
  \clarify{waarom?}
  \begin{figure}[H]
    \centering
    \begin{tikzpicture}[scale=2] 
      \coordinate [label=below:$a$] (a) at (0,0);
      \coordinate [label=above:$b$] (b) at (1,2);
      \coordinate [label=below:$c$] (c) at (4,0);
      \draw (a) -- (b) -- (c) -- cycle;
      
      \coordinate [label=below:$a'$] (ap) at ($ (b) $);      
      \coordinate [label=above:$b'$] (bp) at ($ (a) $);    
      \coordinate [label=left:$c'$] (cp) at ($ (a)!.5!(b) $);    
      
      \draw (ap) -- (a);
      \draw (bp) -- (b);
      \draw (cp) -- (c);
    \end{tikzpicture}    
    \caption{Stelling van Ceva zonder de voorwaarden: $a' \neq b$, $b' \neq c$ en $c' \neq a$}
    \label{fig:stelling-van-ceva-alternatief}
  \end{figure}
\end{opm}

\begin{st}
  Stelling van \term{Menelaos}\\
  Zij $a$, $b$ en $c$ drie affien onafhankelijke punten in $\mathbb{A}^{2}$ en zij $a'$ een punt op de rechte $bc$, verschillend van $b$, $b'$ een punt op de rechte $ac$, verschillend van $b$ en $c'$ een punt op de rechte $ab$, verschillend van $a$.
  $a$, $b$ en $c$ zijn collineair als en slechts als het volgende geldt over de deelverhoudingen:
  \[ (a',b,c)(b',c,a)(c',a,b) = 1 \]
  
  \begin{figure}[H]
    \centering
    \begin{tikzpicture}[scale=1.5] 
      \coordinate [label=below:$a$] (a) at (0,0);
      \coordinate [label=above:$b$] (b) at (2,2);
      \coordinate [label=below:$c$] (c) at (4,0);
      \draw (a) -- (b) -- (c) -- cycle;

      \coordinate [label=above:$a'$] (ap) at ($ (b)!0.4!(c) $);      
      \coordinate [label=below:$b'$] (bp) at ($ (a)!2!(c) $);    
      \coordinate [label=above:$c'$] (cp) at ($ (a)!.75!(b) $);    

      \draw [dashed] (c) -- (bp);

      \draw (cp) -- (bp);
    \end{tikzpicture}    
    \caption{Stelling van Menelaos voor de transformatie}
    \label{fig:stelling-van-menelaos-voor}
  \end{figure}
  
  \begin{figure}[H]
    \centering
    \begin{tikzpicture}[scale=4.5] 
      \coordinate [label=below:$a$] (a) at (0,0);
      \coordinate [label=above:$b$] (b) at (0,1);
      \coordinate [label=below:$c$] (c) at (1,0);
      \draw (a) -- (b) -- (c) -- cycle;

      \coordinate [label=above:$a'$] (ap) at ($ (b)!0.4!(c) $);      
      \coordinate [label=below:$b'$] (bp) at ($ (a)!2!(c) $);    
      \coordinate [label=left:$c'$] (cp) at ($ (a)!.75!(b) $);    

      \draw [dashed] (c) -- (bp);

      \draw (cp) -- (bp);
    \end{tikzpicture}    
    \caption{Stelling van Menelaos na de transformatie}
    \label{fig:stelling-van-menelaos-na}
  \end{figure}

\TODO{bewijs 54} 
\end{st}


\section{Pappus en Desargues}
\label{sec:pappus-en-desargues}

\begin{st}
  Stelling van \term{Pappus}\\
  Zij $L$ en $L'$ twee verschillende rechten in $\mathbb{A}^{2}$.
  Zij $x$, $y$ en $z$ drie punten op $L$ en $x'$, $y'$ en $z'$ drie punten op $L'$ die allemaal niet tot de doorsnede van $L$ en $L'$ behoren.
  \[
    ((xy' \parallel x'y) \wedge (yz' \parallel y'z)) \Rightarrow xz' \parallel x'z
  \]
  
\TODO{bewijs 56} 
\end{st}

\begin{st}
  Stelling van \term{Desargues}\\
  Zij $a$, $b$ en $c$ drie affien onafhankelijke punten en $a'$, $b'$ en $c'$ drie affien onafhankelijke punten, beide in $\mathbb{A}^{2}$.
  \[  ((ab \parallel a'b') \wedge (ac \parallel a'c') \wedge (bc \parallel b'c')) \Rightarrow aa', bb' \text{ en } cc' \text{ zijn concurrent of parallel }\]
  
\TODO{bewijs 57} 
\end{st}


\end{document}
