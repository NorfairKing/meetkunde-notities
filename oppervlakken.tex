\documentclass[main.tex]{subfiles}
\begin{document}

\chapter{Oppervlakken in $\mathbb{E}^{3}$}
\label{cha:oppervl-e3}

\section{Reguliere parametrisaties en patches}
\label{sec:regul-param-en}

\begin{de}
  Een afbeelding $x: U \subseteq \mathbb{R}^{2} \rightarrow \mathbb{E}^{3}:\ (u,v) \mapsto (x_{1}(u,v), x_{2}(u,v), x_{3}(u,v))$ ven een open deel van $\mathbb{R}^{2}$ naar $\mathbb{E}^{3}$ die oneindig vaak differentieerbaar is, wordt \term{regulier} genoemd als het volgende geldt:
  \[
  \forall u,v \in U:\ 
  rang
  \begin{pmatrix}
    \frac{\partial x_{1}}{\partial u}(u,v) & \frac{\partial x_{2}}{\partial u}(u,v) & \frac{\partial x_{3}}{\partial u}(u,v)\\
    \frac{\partial x_{1}}{\partial v}(u,v) & \frac{\partial x_{2}}{\partial v}(u,v) & \frac{\partial x_{3}}{\partial v}(u,v)
  \end{pmatrix}
  = 2
  \]
  De $x$ noemen we een \term{reguliere parametrisatie}.
\end{de}

\begin{de}
  Een \term{patch} is een injectieve reguliere parametrisatie.
\end{de}

\extra{voorbeelden}
\section{Omwentelingsoppervlakken en regeloppervlakken}
\label{sec:omwent-en-regel}

\extra{voorbeelden}
\section{Vectorvelden en de shape-operator}
\label{sec:vectorvelden-en-de}

\begin{de}
   Zij $x: U \subseteq \mathbb{R}^{2} \rightarrow \mathbb{E}^{3}:\ (u,v) \mapsto x(u,v)$ een patch.
   Een \term{vectorveld} langs $x$ is een afbeelding $Y$ als volgt:
   \[ Y: U \rightarrow T\mathbb{E}^{3}:\ (u,v) \mapsto Y(u,v) \in T_{x(u,v)}\mathbb{E}^{3} \]
\end{de}

\begin{de}
  Een vectorveld $Y$ is \term{rakend} als het volgende geldt:
  \[ \forall (u,v) \in U:\ Y(u,v) \in T_{x(u,v)}x(U) \]
\end{de}

\begin{de}
  Een vectorveld $Y$ is \term{normaal} als het volgende geldt:
  \[ \forall (u,v) \in U:\ Y(u,v) \in T_{x(u,v)}^{\bot}x(U) \]
\end{de}

\begin{de}
  Het \term{eenheidsnormaal vectorveld} geassocieerd aan een patch $x$ definieri\"eren we als $\xi$:
  \[ \xi:\ U \rightarrow T\mathbb{E}^{3}:\ (u,v) \mapsto \xi(u,v) = \frac{x_{u}(u,v) \times x_{v}(u,v)}{\Vert x_{u}(u,v) \times x_{v}(u,v) \Vert} \in T_{x(u,v)}^{\bot}x(U) \]
\end{de}

\begin{de}
  Zij $x: U \subseteq \mathbb{R}^{2} \rightarrow \mathbb{E}^{3}:\ (u,v) \mapsto x(u,v)$ een patch, $\xi$ het eenheidsnormaal vectorveld geassocieerd aan $x$ en $p=x(u_{0},v_{0})$ een punt van $x(U)$.
  We defini\"eren de \term{shape-operator} in $p$ als de lineaire afbeelding $S_{p}$:
  \[
  S_{p}:\ T_{p}x(U) \rightarrow T_{p}x(U):\ 
  \left\{
    \begin{array}{rl}
      S_{p}(x_{u}(u_{0},v_{0})) &= -\xi_{u}(u_{0},v_{0})\\
      S_{p}(x_{v}(u_{0},v_{0})) &= -\xi_{v}(u_{0},v_{0})\\
    \end{array}
  \right.
  \]
\end{de}

\begin{lem}
  De shape-operator is symmetrisch.
\TODO{bewijs p 196}
\end{lem}

\begin{de}
  De orthonormale basis $\{e_{1},e_{2}\}$ die we bekomen door de shape-operator te diagonaliseren noemen we de \term{hoofdrichtingen} in $p$.
  \[ Se_{1}=k_{1}e_{1} \text{ en } Se_{2}= k_{2}e_{2} \]
  De $k_{i}$ noemen we de \term{hoofdkrommingen} in $p$.
\end{de}

\begin{de}
  Zij  $x: U \subseteq \mathbb{R}^{2} \rightarrow \mathbb{E}^{3}:\ (u,v) \mapsto x(u,v)$ een patch, $p$ een punt van $x(U)$ en $S$ de shape-operator in $p$.
  Voor een eenheidsvector $w\in T_{p}x(U)$ defini\"eren we de \term{normale kromming} van $x(U)$ in $p$ in de richting van $w$ als $k(w)$:
  \[ k(w) = Sw \cdot w \]
\end{de}

\begin{lem}
  De normale krommingen van $x(U)$ in $p$ zijn volledig bepaal door de hoofdkrommingen.
  \TODO{meer uitwerken en bewijs p 197}
\end{lem}

\section{Meetkundige interpretiatie van de normale kromming}
\label{sec:meetk-interpr-van}

\begin{de}
  Zij $x: U \subseteq \mathbb{R}^{2} \rightarrow \mathbb{E}^{3}:\ (u,v) \mapsto x(u,v)$ een patch, $p$ een punt van $x(U)$ en $w$ een eenheidsvector uit $T_{p}x(U)$.
  Defini\"eer het vlak $\pi$, door $p$ met richting opgespannen door $w$ en de eenheidsnormaal $\xi(u_{0},v_{0})$ in $p$:
  \[ \pi = p + span\{ w, \xi(u_{0},v_{0})\} \]
  We noemen $\pi$ een \term{normaal vlak} op $x(U)$ in $p$.
\end{de}

\begin{de}
  De doorsnede van $\pi$ en $x(U)$ noemen we de \term{normale doorsnede} geassocieerd aan $w$.
\end{de}

\begin{lem}
  Zij $x: U \subseteq \mathbb{R}^{2} \rightarrow \mathbb{E}^{3}:\ (u,v) \mapsto x(u,v)$ een patch, $\alpha:\ I \subseteq \mathbb{R} \rightarrow \mathbb{E}^{3}$ een kromme waarvan het beeld volledig in $x(U)$ ligt.
  Er bestaan dan differentieerbare functie $a,b: I \rightarrow \mathbb{R}$ als volgt:
  \[ \forall t\in I:\ \alpha(t) = x(a(t),b(t))\]
\TODO{bewijs p 199}
\end{lem}

\begin{st}
  Zij $x: U \subseteq \mathbb{R}^{2} \rightarrow \mathbb{E}^{3}$ een patch, $p(u_{0},v_{0})\in x(U)$ een punt en $\xi(u_{0},v_{0})$ de eenheidsnormaal geassocieerd aan $x$ in $p$
  Zij verder $w\in T_{p}x(U)$ een eenheidsvvector en $\pi$ een vlak als volgt:
  \[ \pi = p + span\{ w, \xi(u_{0},v_{0})\} \]
  Noem $\kappa$ de kromming van de normale doorsnede $x\cap x(U)$ in het punt $p$ en $k(w)$ de normale kromming van $w$.
  \[ k(w) = \epsilon \kappa \]
  Met $\epsilon = \pm 1$
  $\epsilon$ is $1$ als de normale doorsnede in $p$ naar $\xi(u_{0},v_{0})$ toe draait en $-1$ als de normale doornede in $p$ van $\xi(u_{0},v_{0})$ weg draait.
\TODO{bewijs p 201}
\end{st}

\section{Gausskromming en gemiddelde kromming}
\label{sec:gausskr-en-gemidd}

\begin{de}
  Zij $x: U \subseteq \mathbb{R}^{2} \rightarrow \mathbb{E}^{3}$ een patch, $p$ een punt van $x(U)$ en $k_{1}$ en $k_{2}$ de hoofdkrommingen van $x(U)$ in $p$.
  De \term{Gausskromming} van $x(U)$ in $p$ defini\"eren we als $K(p)$...
  \[ K(p) = k_{1}k_{2} \]
  ... en de \term{gemiddelde kromming} als $H(p)$.
  \[ H(p) = \frac{k_{1}+k_{2}}{2} \]
\end{de}



\end{document}
