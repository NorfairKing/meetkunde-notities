\documentclass[main.tex]{subfiles}
\begin{document}

\chapter{Euclidische transformaties}
\label{cha:euclidische-transformaties}

\section{Rotaties in $\mathbb{E}^{2}$}
\label{sec:roties-in-e2}

\begin{lem}
  Voor elke $A\in SO(2)$ met $\det(A)=1$ bestaat er een $\theta\in \mathbb{R}$ zodat $A$ er als volgt uit ziet:
  \[
  A =
  \begin{pmatrix}
    \cos(\theta) & -\sin(\theta)\\
    \sin(\theta) & \cos(\theta)
  \end{pmatrix}
  \]

  \begin{proof}
    Zij $A$ een matrix in $SO(2)$:
    \[
    A =
    \begin{pmatrix}
      a_{11} & a_{12}\\
      a_{21} & a_{22}
    \end{pmatrix}
    \]
    $A^{T}$ en $A^{-1}$ zien er dan als volgt uit:
    \[
    A^{T} =
    \begin{pmatrix}
      a_{11} & a_{21}\\
      a_{12} & a_{22}
    \end{pmatrix}
    ,\quad
    A^{-1} = 
    \begin{pmatrix}
      a_{22} & -a_{12}\\
      -a_{21} & a_{11}
    \end{pmatrix}
    \]
    Omdat $A$ orthogonaal is moet $A^{T}$ gelijk zijn aan $A^{-1}$.
    $a_{11}$ en $a_{22}$, en $a_{21}$ en $-a_{12}$ moeten dus gelijk zijn.
    \[ 
    A = 
    \begin{pmatrix}
      a_{11} & a_{12}\\
      -a_{12} & a_{11}
    \end{pmatrix}
    \]
    Omdat de determinint van $A$ $1$ is moet ook $a_{11}^{2} + a_{12}^{2}=1$ gelden.
    $a_{11}$ en $A_{21}$ moeten dus $\cos\theta$ en $\sin\theta$ zijn voor een bepaalde $\theta$.
  \end{proof}
\end{lem}

\begin{opm}
  Voor elke $A\in O(2)$ met $\det(A) =-1$ bestaat er dus ook een $\theta\in \mathbb{R}$ zodat $A$ er als volgt uit ziet:
  \[
  A =
  \begin{pmatrix}
    \cos(\theta) & \sin(\theta)\\
    \sin(\theta) & -\cos(\theta)
  \end{pmatrix}
  \]
\end{opm}

\begin{de}
  Een \term{rotatie} van $\mathbb{E}^{2}$ is een afbeelding van de vorm $Rot$:
  \[
  Rot:\ \mathbb{E}^{2} \rightarrow \mathbb{E}^{2}:\ 
  p \mapsto Rot(p)
  = x_{0} + A(\overrightarrow{x_{0}p})
  = x_{0} +
  \begin{pmatrix}
    \cos(\theta) & -\sin(\theta)\\
    \sin(\theta) & \cos(\theta)
  \end{pmatrix}
  \overrightarrow{x_{0}p}
  \]
  Hierin is $x_{0}$ een element van $\mathbb{E}^{2}$ en $A$ een element van $SO(2)$ verschillend van de eenheidsmatrix.
  We noemen $x_{0}$ het \term{centrum} van de rotatie en $\theta$ de \term{rotatiehoek}.
\end{de}

\begin{ei}
  Een rotatie is een ori\"entatiebewarende isometrie.
\extra{bewijs}
\end{ei}

\begin{de}
  We definieren voor een vector $(x,y)$ van $\mathbb{E}^{2}$ de \term{poolco\"ordinaten} $(r,\theta)$ als volgt:
  \[ (x,y) = (r\cos\theta, r\sin\theta) \]
\end{de}

\begin{st}
  Zij $p$ een punt in $\mathbb{E}^{2}$ met poolcoordinaten $(r,v), r\in \mathbb{R}^{+}, v\in \mathbb{R}$ ten opzichte van het centrum van een $A$ een rotatie met rotatiehoek $\theta$.
  \[
  A = 
  \begin{pmatrix}
    \cos(\theta) & -\sin(\theta)\\
    \sin(\theta) & \cos(\theta)
  \end{pmatrix}
  \quad\text{ en }\quad
  \overrightarrow{x_{0}p} = 
  \begin{pmatrix}
    r\cos(v)\\ r\sin(v)\\
  \end{pmatrix}
  \]
  \[ Rot(p) = x_{0} +
  \begin{pmatrix}
    r\cos(v+\theta)\\ r\sin(v+\theta)\\
  \end{pmatrix}
  \]
\extra{bewijs}
\end{st}
\section{Rotaties in $\mathbb{E}^{3}$}
\label{sec:roties-in-e3}

\begin{de}
  Een \term{rotatie} van $\mathbb{E}^{3}$ is een afbeelding van de vorm $Rot$.
  \[ Rot:\ \mathbb{E}^{3} \rightarrow \mathbb{E}^{3}:\ p \mapsto Rot(p) = x_{0} + A(\overrightarrow{x_{0}p}) \]
  Hierin is $x_{0}$ een element van $\mathbb{E}^{3}$ en $A$ een element van $SO(3)$ verschillend van de eenheidsmatrix.
  We noemen $x_{0}$ het \term{centrum} van de rotatie en $\theta$ de \term{rotatiehoek}.
\end{de}


\begin{st}
  De punten die op zichzelf worden afgebeeld vormen een affiene deelruimte van $\mathbb{E}^{3}$ door $x_{0}$ in de richting van $Ker(A-\mathbb{I}_{3})$.
\extra{bewijs}
\end{st}

\begin{st}
  Een vast punt van een rotatie is een eigenvector van de rotatiematrix.
\extra{bewijs}
\end{st}

\begin{de}
  We noemen $L = x_{0} + Ker(A-\mathbb{I}_{3})$ de \term{as} van de rotatie.
\end{de}

\section{Spiegelingen}
\label{sec:spiegelingen}

\begin{de}
  Zij $S$ een affiene deelruimte van $\mathbb{E}^{n}$.
  Voor een willekeurig punt $x\in \mathbb{E}^{n}$ defini\"eren we $T_{x}$ als de unieke deelruimte door $x$, orthogonaal complementair mat $S$.
  Bovendien definieren wa $\pi_{S}(x)$ als het snijpunt van $S$ en $T_{x}$.
\end{de}
 
\begin{de}
  De afbeelding $\pi_{S}:\ \mathbb{E}^{n} \rightarrow S:\ x \mapsto \pi_{S}(x)$ noemen we een \term{orthogonale projectie} op $S$.
\end{de}

\begin{de}
  De \term{spiegeling} $R_{S}$ in een affiene deelruimte $S$ van $\mathbb{E}^{n}$ is een afbeelding als volgt:
  \[ R_{S}:\ \mathbb{E}^{n} \rightarrow \mathbb{E}^{n}:\ x \mapsto R_{S}(x) = x + 2\overrightarrow{x\pi_{S}(x)} \]
\end{de}

\begin{st}
  Een spiegeling is een isometrie.
\TODO{bewijs p 95}
\end{st}

\begin{st}
  Voor elke spiegeling $R_{S}$ geldt $R_{S} \circ R_{S} = Id$.
\extra{bewijs}
\end{st}

\begin{st}
  Voor elke spiegeling $R_{S}$ geldt $\forall s \in S:\ R_{S}(x) = x$.
\extra{bewijs}
\end{st}

\begin{lem}
  Zij $S=p+V$ een affiene deelruimte van $\mathbb{E}^{n}$ en $W$ het orthogonaal complement van $V$.
  \[ \forall v\in V:\ (R_{S})_{*}(v) = v \]
\extra{bewijs}
\end{lem}

\begin{lem}
  Zij $S=p+V$ een affiene deelruimte van $\mathbb{E}^{n}$ en $W$ het orthogonaal complement van $V$.
  \[ \forall w\in W:\ (R_{S})_{*}(w) = w \]
\extra{bewijs}
\end{lem}

\begin{de}
  Een \term{schuifspiegeling} is de samenstelling van een spiegeling in een as $S$ en een translatie in de richting van een vector parallel met $S$.
\end{de}

\section{Structuur van isometrie\"en}
\label{sec:struct-van-isom}

\begin{de}
  Zij $F$ een isometrie van $\mathbb{E}^{n}$, dan noemen wa $x$ een \term{vast punt} van $F$ als $F(x)=x$ geldt.
\end{de}

\begin{de}
  De verzameling vaste punten van een isometrie noteren we als $V(F)$.
  \[ V(F) = \{ x \in \mathbb{E}^{n} \ |\ F(x) = x \} \]
\end{de}

\begin{st}
  Als $F$ een isometrie van $\mathbb{E}^{n}$ is en $V(F)$ niet leeg, dan is $V(F)$ een affiene deelruimte in de richting van $Ker(F_{*}-\mathbb{I}_{n})$.
\TODO{bewijs p 97}
\end{st}

\begin{st}
  Zij $F$ een isometrie van $\mathbb{E}^{n}$.
  Dan bestaan er juist \'e\'en isometrie $G$ en juist \'e\'en translatie $t_{b}$ van $\mathbb{E}^{n}$ als volgt:
  \begin{itemize}
  \item $F = t_{b} \circ G$ $(F_{*} = F_{*})$
  \item $V(G) \neq \emptyset$
  \item $G_{*}b = b$ ($b$ ligt in de richting van $V(G)$.)
  \item $t_{b}\circ G = G \circ t_{b}$
  \end{itemize}
\TODO{bewijs p 97}
\end{st}

\begin{st}
  Elke isometrie $F\in Iso(2,\mathbb{R}$ van het Euclidisch vlak is een translatie, een rotatie of een schuifspiegeling met als as een rechte.
  \TODO{bewijs p 99}
\end{st}

\begin{de}
  De samenstelling van een ratatie rond een as $L$ met een translatie in de richting van $L$ een \term{schroefbeweging}.
\end{de}

\begin{de}
  De samenstelling van een rotatie rond een as $L$ en een spiegelijk in een vlak $H$, loodrecht op $L$ een \term{draaispiegeling}.
\end{de}

\begin{st}
  Elke isometrie $F\in Iso(3,\mathbb{R})$ van de driedimensionale Euclidische ruimte is een translatie, een schroefbeweging, een schuifspiegeling tegenover een vlak of een draaispiegeling.
\TODO{bewijs p 102} 
\end{st}

\begin{st}
  \[ R_{H} \circ K = K \circ R_{H} \]
  \clarify{meer uitleg}
\extra{bewijs}
\end{st}




\end{document}
