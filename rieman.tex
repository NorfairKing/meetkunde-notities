\documentclass[main.tex]{subfiles}
\begin{document}

\chapter{Riemannse en pseudo-Riemannse meetkunde}
\label{cha:rieman}

\section{De Lorentz-Minkowski ruimte}
\label{sec:de-lorentz-minkowski}

\begin{de}
  Zij $p$ een punt van $\mathbb{A}^{n}$ en $v,w\in T_{p}\mathbb{A}^{n}$ raakvectoren aan $p$.
  We defini\"eren het \term{Lorentz scalair product} als volgt:
  \[ v \cdot w = -v_{1}w_{1} + v_{2}w_{2} + \dotsb + v_{n}w_{n} \]
\end{de}

\begin{de}
  Een affiene ruimte $\mathbb{A}^{n}$, uitgerust met het Lorentz scalair product noemt men de \term{Lorentz-Minkowski ruimte}.
\end{de}

\begin{de}
  Een vector $v\in T_{p}\mathbb{A}^{n}$ is een \term{lichtvector} als $v\cdot v$ kleiner is dan $0$.
\end{de}

\begin{de}
  Een vector $v\in T_{p}\mathbb{A}^{n}$ is een \term{ruimtevector} als $v\cdot v$ groter is dan $0$.
\end{de}

\begin{de}
  Een vector $v\in T_{p}\mathbb{A}^{n}$ is een \term{nulvector} of \term{lichtvector} als $v\cdot v$ gelijk is aan $0$.
\end{de}

\section{Riemannse meetkunde}
\label{sec:riemannse-meetkunde}

\begin{de}
  Zij $U$ een open deel van $\mathbb{A}^{n}$ en $f: U \rightarrow \mathbb{R}^{+}_{0}$ een positieve differentieerbare functie.
  We definieren een scalair product voor alle $p\in U$ en $v_{p},w_{p} \in T_{p}\mathbb{A}^{n}$ als volgt:
  \[ v_{p}\cdot w_{p} = f(p)\left( \sum_{i=1}^{n}v_{i}w_{i} \right) \]
\end{de}



\end{document}
