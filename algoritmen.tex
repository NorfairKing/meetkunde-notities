\documentclass[main.tex]{subfiles}
\begin{document}

\chapter{Algoritmen}
\label{cha:algoritmen}

\section{Onderlinge ligging van twee affiene deelruimten bepalen}
\subsection{Abstract}
\subsubsection{Vraag}

\begin{center}
  Gegeven twee affiene deelruimten $p+V$ en $q+W$.
  Bepaal de onderlinge ligging van $p+V$ en $q+W$.
\end{center}

\subsubsection{Antwoord}
Herneem eerst het deel over affiene deelruimten.\secref{sec:affiene-deelruimten}
Ga in volgorde deze `vragenlijst' af.
\begin{itemize}
\item $V = W \rightarrow p+V \parallel q+W$
\item $V \subseteq W \rightarrow p+V \triangleleft q+W$
\item $S \cap T \neq \emptyset \rightarrow $ snijdend
\item $S \cap T = \emptyset \rightarrow $ kruisend
\end{itemize}

Meestal worden de punten helemaal gegeven, maar van de lineaire deelruimten enkel een basis of een verzameling opspannede vectoren $v$ en $w$.

\begin{itemize}
\item Dun $v$ en $w$ uit tot basissen $v'$ en $w'$ van $V$ en $W$.
  Noem $v'$ en $w'$ trouwens zo dat $v'$ de kleinste verzameling is van de twee.
\item Als $v'$ en $w'$ niet hetzelfde aantal vectoren bevat, dan kunnen $V$ en $W$ al niet gelijk zijn.
\item Als de vectoren in $v'$ lineair afhankelijk zijn van de vectoren in $w'$, dan zijn $V$ en $W$ gelijk of is $V$ een deel van $W$.
\end{itemize}

\subsection{Voorbeeld}

\subsubsection{Vraag}
\begin{center}
  Gegeven de punten $p=(1,1,1)$ en $q=(2,-1,0)$ en de lineaire deelruimten $V=<(1,0,1),(1,2,3)>$ en $W=<(1,4,5)>$. Bepaal de onderlinge ligging van $V$ en $W$.
\end{center}

\subsubsection{Antwoord}
\begin{itemize}
\item $v=\{(1,0,1),(1,2,3)\}$ is al een vrije verzameling, alsook $w=\{(1,4,5)\}$.
\item $v$ en $w$ bevatten niet hetzelfde aantal vectoren, dus $V$ en $W$ zijn zeker niet gelijk.
\item We bekijken of $v\cup w$ nog steeds een vrije verzameling is.
  In feite lossen we dus het volgende stelsel op:
  \[
  a(1,0,1) + b(1,2,3) + c(1,4,5) = 0
  \]
  Als de enige oplossing van dit stelsel $(a,b,c) = (0,0,0)$ is, dan is $W$ geen deel van $V$.
  Omdat we met precies drie vectoren te maken hebben van dimensie $3$ kunnen we dit echter eenvoudiger berekenen met een determinant:
  \[
  \begin{vmatrix}
    1 & 0 & 1\\
    1 & 2 & 3\\
    1 & 4 & 5
  \end{vmatrix}
  \]
  Deze determinant is $0$, dus $v\cup w$ is geen vrij verzameling. $W$ is dus een deel(ruimte) van $V$.
  Dit betekent dat $q+W$ zwak parallel is met $p+V$.
  \[ q+W \triangleleft p+V \]
\end{itemize}

\newpage
\section{Parametervergelijkingen van een affiene deelruimte bepalen}
\subsection{Abstract}
\subsubsection{Vraag}
\begin{center}
  Zij $S$ de affiene deelruimte van $\mathbb{A}^{n}$ door het punt $p$ in de richting van een lineaire deelruimte $V$ van $\mathbb{R}^{n}$ met als basis $\beta=\{v_{1},v_{2},\dotsc,v_{k}\}$.
  Bepaal de parametervergelijkingen van $S=p+V$.
\end{center}
\subsubsection{Antwoord}
Elk punt $x\in \mathbb{A}^{n}$ in $S$ voldoet aan de volgende vector vergelijking:
\[ x\in S \Leftrightarrow \exists \lambda_{1},\lambda_{2},\dotsc ,\lambda_{k}:\ x =  p  + \sum_{i=1}^{k}\lambda_{i}v_{i} \]
Deze vectorvergelijking komt overeen met het volgende stelsel:
\[
x \in S \Leftrightarrow \exists \lambda_{1},\lambda_{2},\dotsc ,\lambda_{k}:\
\left\{
  \begin{array}{c}
    x_{1} = p_{1} + \lambda_{1}v_{11} + \lambda_{2}v_{12} + \dotsb + \lambda_{k}v_{1k}\\
    x_{2} = p_{2} + \lambda_{1}v_{21} + \lambda_{2}v_{22} + \dotsb + \lambda_{k}v_{2k}\\
    \vdots \\
    x_{n} = p_{n} + \lambda_{1}v_{n1} + \lambda_{2}v_{n2} + \dotsb + \lambda_{k}v_{nk}\\
  \end{array}
\right.
\]

\subsection{Voorbeeld}
\subsubsection{Vraag}
\begin{center}
  Zij $S$ de affiene deelruimte van $\mathbb{A}^{4}$ door het punt $p= (1,2,3,4)$ in de richting van $V = <(0,1,-1,0), (4,6,3,5)>$.
  Bepaal de parametervergelijkingen van $S$.
\end{center}
\subsubsection{Antwoord}
\[
x \in \Leftrightarrow \exists \lambda,\mu:\ x = (1,2,3,4) + \lambda (0,1,-1,0) + \mu (4,6,3,5)
\]
\[
x \in S \Leftrightarrow \exists \lambda, \mu:\
\left\{
  \begin{array}{c}
    x_{1} = 1 + 0\lambda + 4\mu\\
    x_{2} = 2 + 1\lambda + 6\mu\\
    x_{3} = 3 - 1\lambda + 3\mu\\
    x_{4} = 4 + 0\lambda + 5\mu\\
  \end{array}
\right.
\]

\newpage
\section{Carthesische vergelijkingen van een affiene deelruimte bepalen}
\subsection{Abstract}
\subsubsection{Vraag}
\begin{center}
  Zij $S$ de affiene deelruimte van $\mathbb{A}^{n}$ door het punt $p$ in de richting van een lineaire deelruimte $V$ van $\mathbb{R}^{n}$ met als basis $\beta=\{v_{1},v_{2},\dotsc,v_{k}\}$.
  Bepaal de Carthesische van $S=p+V$.
\end{center}

\subsubsection{Antwoord}
De vergelijkingen voor $S$ halen we uit de volgende:
\[
x \in S \Leftrightarrow
rang
\begin{pmatrix}
  x_{1}-p_{1} & x_{2}-p_{2} & \hdots & x_{n}-p_{n} \\
  v_{11} & v_{12} & \hdots & v_{1n} \\
  v_{21} & v_{22} & \hdots & v_{2n} \\
  \vdots & \vdots & \ddots & \vdots \\ 
  v_{k1} & v_{k2} & \hdots & v_{kn}
\end{pmatrix}
= dim(V)
\]
Wanneer we de determinant van deze matrix ontwikkelen naar de eerste rij moet elke determinant nul zijn, en zo bekomen we de carthesische vergelijkingen.
(Bekijk vooral het voorbeeld, dat zal veel verduidelijken)

\subsection{Voorbeeld}
\subsubsection{Vraag}
\begin{center}
  Bepaal in $\mathbb{A}_{4}$ voor de affiene deelruimte door $p=(1,1,5,7)$ in de richting van $V=<(1,4,1,6),(6,4,2,8)>$ de carthesische vergelijkingen.
\end{center}

\subsubsection{Antwoord}
\[
x \in S \Leftrightarrow
rang
\begin{pmatrix}
  x_{1}-1 & x_{2}-1 & x_{3}-5 & x_{4}-7\\
  1 & 4 & 1 & 6\\
  6 & 4 & 2 & 8
\end{pmatrix}
= 2
\]
Dit komt neer op de volgende vergelijkingen.
\[
x \in S \Leftrightarrow
\left\{
  \begin{array}{c}
    \begin{vmatrix}
      x_{2}-1 & x_{3}-5 & x_{4}-7\\
      4 & 1 & 6\\
      4 & 2 & 8
    \end{vmatrix}
    = 0\\
    \begin{vmatrix}
      x_{1}-1  & x_{3}-5 & x_{4}-7\\
      1 & 1 & 6\\
      6 & 2 & 8
    \end{vmatrix}
    =0\\
    \begin{vmatrix}
      x_{1}-1 & x_{2}-1 & x_{4}-7\\
      1 & 4 & 6\\
      6 & 4 & 8
    \end{vmatrix}
    = 0\\
    \begin{vmatrix}
      x_{1}-1 & x_{2}-1 & x_{3}-5\\
      1 & 4 & 1 \\
      6 & 4 & 2 
    \end{vmatrix}
    = 0\\
  \end{array}
\right.
\]
We rekenen alleen de eerste twee uit omdat de dimensie van de gezochte affiene deelruimte maximum $2$ kan zijn.
De andere twee vergelijkingen zullen dus lineair afhankelijk zijn van de eerste twee.
Of uitgeschreven:
\[
x \in S \Leftrightarrow
\left\{
\begin{array}{cccccc}
          &-4x_{2} &-8x_{3}  &+ 4x_{4} &+16 &= 0\\
  -4x_{1} &        &+28x_{3} &-4x_{4} &-106 &= 0\\
(8x_{1} &+28x_{2} &        &-20x_{4} &+172 &=0)\\
(4x_{1}  &+4x_{2} &- 20x_{3} &        &+92 &=0)
\end{array}
\right.
\]
De bekomen vergelijkingen zijn nu lineair onafhankelijk, maar we kunnen ze nog vereenvoudigen.
\[
x \in S \Leftrightarrow
\left\{
\begin{array}{cccccc}
          &-x_{2} &-2x_{3}  &+ x_{4} &+4 &= 0\\
  -2x_{1} &        &+14x_{3} &-2x_{4} &-53 &= 0\\
\end{array}
\right.
\]

\newpage
\section{Som van twee affiene deelruimten bepalen vanuit Parametervergelijkingen}
\subsection{Abstract}
\subsubsection{Vraag}
\begin{center}
  Zij $p+V$ en $p+W$ twee affiene deelruimten van $\mathbb{A}^{n}$ met parametervergelijkingen als volgt, bepaal de som ervan.
  \[ x\in p+V \Leftrightarrow \exists \lambda_{1},\lambda_{2},\dotsc,\lambda_{a}:\ x = p + \sum_{i=1}^{a}\lambda_{i}v_{i} \]
  \[ x\in q+W \Leftrightarrow \exists \lambda_{1},\lambda_{2},\dotsc,\lambda_{b}:\ x = p + \sum_{i=1}^{b}\lambda_{i}w_{i} \]
\end{center}

\subsubsection{Antwoord}
\[ x \in p+(V+W)  \Leftrightarrow \lambda_{1},\lambda_{2},\dotsc,\lambda_{a}\lambda'_{1},\lambda'_{2},\dotsc,\lambda'_{b}:\ x = p + \sum_{i=1}^{a}\lambda_{i}v_{i}+ \sum_{i=1}^{b}\lambda'_{i}w_{i}\]

\subsection{Voorbeeld}
\subsubsection{Vraag}
\begin{center}
  Zij $p=(1,2,3)$ een punt van $\mathbb{A}^{3}$ en $V=<(5,4,3),(-2,5,1)>$ en $W=<(1,2,3),(-1,0,1)>$ twee lineaire deelruimten van $\mathbb{R}^{3}$
  Bepaal $(p+V) + (p+W)$.
\end{center}

\subsubsection{Antwoord}
\[
x\in p+(V+W) \Leftrightarrow \exists \lambda, \gamma, \mu, \nu:\ 
x = (1,2,3) + \lambda (5,4,3) + \gamma (-2,5,1) + \mu (1,2,3) + \nu (-1,0,1)
\]


\question{kunnen we met parametervergelijkingen ook een doorsnede berekenen?}

\newpage
\section{Doorsnede van twee affiene deelruimten bepalen vanuit Carthesische vergelijkingen}
\subsection{Abstract}
\subsubsection{Vraag}
\begin{center}
  Zij $R$ en $S$ twee affiene deelruimten van $\mathbb{A}^{n}$ met carthesische vergelijkingen als volgt, bepaal de onderlinge ligging.
  \[
  x \in R \Leftrightarrow
  \left\{
    \begin{array}{c}
      a_{11}x_{1} + \dotsb + a_{1n}x_{n} = b_{1}\\
      \vdots\\
      a_{k1}x_{1} + \dotsb + a_{kn}x_{n} = b_{k}\\
    \end{array}
  \right.
  ,\quad
  x \in S \Leftrightarrow
  \left\{
    \begin{array}{c}
      c_{11}x_{1} + \dotsb + c_{1n}x_{n} = d_{1}\\
      \vdots\\
      c_{k1}x_{1} + \dotsb + c_{kn}x_{n} = d_{k}\\
    \end{array}
  \right.
  \]
\end{center}
\subsubsection{Antwoord}
We zoeken de doorsnede van $R$ en $S$ door de vergelijkingen samen te nemen en er ze uit te dunnen tot een vrij stel vergelijkingen.

\subsection{Voorbeeld}
\subsubsection{Vraag}
\begin{center}
  Zij $R$ en $S$ de affiene deelruimtes van $\mathbb{A}^{3}$ als volgt, bepaal de onderlinge ligging.
  \[
  x \in R \Leftrightarrow
  \left\{
  x_{1}+x_{2}=1
  \right.
  ,\quad
  x\in S \Leftrightarrow
  \left\{
    \begin{array}{c}
      x_{1}-x_{2} = 2\\
      x_{1} + 3x_{3} = 0\\
    \end{array}
  \right.
  \]
\end{center}

\subsubsection{Antwoord}
We bepalen eerst de doorsnede van $R$ en $S$ door de vergelijkingen samen te nemen.
\[
  x\in R \cap S \Leftrightarrow
  \left\{
    \begin{array}{c}
      x_{1}+x_{2}=1\\
      x_{1}-x_{2} = 2\\
      x_{1} + 3x_{3} = 0\\
    \end{array}
  \right.
\]
\[
\leftrightarrow
  \left\{
    \begin{array}{c}
      x_{1}=\frac{3}{2}\\
      x_{2}=-\frac{1}{2}\\
      \\
    \end{array}
  \right.
\]
De doorsnede van $R$ en $S$ is dus de rechte daar $(\frac{3}{2},-\frac{1}{2},0)$, evenwijdig met de $x_{3}$ as.

\newpage
\section{Onderlinge ligging van twee affiene deelruimten bepalen vanuit zowel carthesische als parametervergelijkingen}
\subsection{Abstract}
\subsubsection{Vraag}
\begin{center}
  Zij $R$ en $S=p+V$ twee affiene deelruimten van $\mathbb{A}^{n}$ met carthesische vergelijkingen en parametervergelijkingen als volgt, bepaal de onderlinge ligging.
  \[
  x \in R \Leftrightarrow
  \left\{
    \begin{array}{c}
      a_{11}x_{1} + \dotsb + a_{1n}x_{n} = b_{1}\\
      \vdots\\
      a_{k1}x_{1} + \dotsb + a_{kn}x_{n} = b_{k}\\
    \end{array}
  \right.
  ,\quad
  x\in S \Leftrightarrow
  x= p+ \sum_{i=1}^{k}a_{i}v_{i}
  \]
\end{center}

\subsubsection{Antwoord}
We vullen de parametervergelijkingen in in de carthesische vergelijkingen om de doorsnede van $R$ en $S$ te berekenen.
\extra{en dan?}

\subsection{Voorbeeld}
\subsubsection{Vraag}
\begin{center}
  Zij $R$ en $S=p+V$ twee affiene deelruimten van $\mathbb{A}^{n}$ met Carthesische vergelijkingen en parametervergelijkingen als volgt.
\end{center}
\[
x\in R \Leftrightarrow
\left\{
  \begin{array}{c}
    x_{1}+x_{2}+x_{3}+x_{4}=1
  \end{array}
\right.
,\quad
x\in S \Leftrightarrow
\begin{pmatrix}x_{1}\\x_{2}\\x_{3}\\x_{4}\end{pmatrix}
=
\begin{pmatrix}1\\1\\1\\1\\\end{pmatrix}
+\lambda
\begin{pmatrix}0\\1\\-1\\0\\\end{pmatrix}
\]

\subsubsection{Antwoord}
We vullen de parametervergelijkingen in in de Carthesische vergelijkingen.
\[ 1+(1+\lambda)+(1+\lambda) + 1 = 1 \]
Deze vergelijking is vals, dus de doorsnede van $R$ en $S$ is leeg.
\extra{en nu?}

\newpage
\section{Vergelijkingen opstellen van een rechte}
\subsection{Abstract}
\subsubsection{Vraag}
\begin{center}
  Zij $p$ en $q$ twee punten in $\mathbb{A}^{n}$, bepaal de rechte door $p$ en $q$.
\end{center}
\subsubsection{Antwoord}
Bereken eerst $\overrightarrow{pq}= q-p$, u zal het nodig hebben!
\begin{itemize}
\item Parametervergelijking:
  \[
  x\in pq \Leftrightarrow \exists \lambda
  x = p + \lambda \overrightarrow{pq} = p + \lambda (q-p)
  \]
\item Carthesische vergelijkingen:
  \[
  x\in pq \Leftrightarrow
  rang
  \begin{pmatrix}
    x_{1}-p_{1} & x_{2}-p_{2} & \hdots & x_{n}-p_{n}\\
    \overrightarrow{pq}_{1} & \overrightarrow{pq}_{2} & \hdots & \overrightarrow{pq}_{n}\\
  \end{pmatrix}
  =
  rang
  \begin{pmatrix}
    x_{1}-p_{1} & x_{2}-p_{2} & \hdots & x_{n}-p_{n}\\
    q_{1}-p_{1} & q_{2}-p_{2} & \hdots & q_{n}-p_{n}\\
  \end{pmatrix}
  = 1
  \]
  Elke $2\times 2$ deelmatrix hiervan moet dus determinant $0$ hebben.
\item Barycentrische vergelijkingen:
  \[
  x\in pq \Leftrightarrow \exists \lambda
  x = \lambda p_{1} + (1-\lambda)p_{2}
  \]
\end{itemize}

\subsection{Voorbeeld}
\subsubsection{Vraag}
\begin{center}
  Zij $p=(8,6,-1,1)$ en $q=(-1,0,4,2)$ twee punten in $\mathbb{A}^{4}$, bepaal de rechte door $p$ en $q$.
\end{center}

\subsubsection{Antwoord}
\[ q-p= (-9,-6,5,1) \]

\begin{itemize}
\item Parametervergelijkingen:
  \[
  x\in pq \Leftrightarrow \exists \lambda:\ 
  x = (8,6,-1,1) + \lambda (-9,-6,5,1)
  \]
\item Carthesische vergelijkingen:
  \[
  x \in pq \Leftrightarrow
  rang
  \begin{pmatrix}
    x_{1}-8 & x_{2}-6 & x_{3}+1 & x_{4}-1\\
    -9 & -6 & 5 & 1\\
  \end{pmatrix}
  =1
  \]
  \[
  x \in pq \Leftrightarrow
  \left\{
  \begin{array}{c}
    \begin{vmatrix}
      x_{1}-8 & x_{2}-6\\
      -9 & -6\\
    \end{vmatrix}
    =0\\
    \begin{vmatrix}
      x_{1}-8 & x_{3}+1\\
      -9 & 5\\
    \end{vmatrix}
    =0\\
    \begin{vmatrix}
      x_{1}-8 & x_{4}-1\\
      -9 & 1\\
    \end{vmatrix}
    =0\\
  \end{array}
  \right.
  \]
  \[ 
  x \in pq \Leftrightarrow
  \left\{
    \begin{array}{ccccccc}
      -6x_{1} &+9x_{2} &        &       & -6 &= 0\\
      5x_{1}  &       &+9x_{3}  &       &-31 &= 0\\
      x_{1}   &       &        &+9x_{4} &-17 &= 0\\
    \end{array}
  \right.
  \]
\item Barycentrische vergelijkingen:
  \[
  x\in pq \Leftrightarrow \exists \lambda:\ 
  x = \lambda (8,6,-1,1) + (1- \lambda)(-1,0,4,2)
  \]
\end{itemize}





\newpage
\section{Parametervergelijkingen omzetten naar Carthesische vergelijkingen}
\subsection{Abstract}
\subsubsection{Vraag}
\subsubsection{Antwoord}

\subsection{Voorbeeld}
\subsubsection{Vraag}
\subsubsection{Antwoord}

\section{Carthesische vergelijkingen omzetten naar parametervergelijkingen}
\subsection{Abstract}
\subsubsection{Vraag}
\subsubsection{Antwoord}

\subsection{Voorbeeld}
\subsubsection{Vraag}
\subsubsection{Antwoord}

\section{Carthesische vergelijkingen omzetten naar parametervergelijkingen}
\subsection{Abstract}
\subsubsection{Vraag}
\subsubsection{Antwoord}

\subsection{Voorbeeld}
\subsubsection{Vraag}
\subsubsection{Antwoord}



\end{document}


%%% Local Variables: 
%%% mode: latex
%%% TeX-master: t
%%% End: 
