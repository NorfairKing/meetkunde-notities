\documentclass[main.tex]{subfiles}
\begin{document}

\chapter{Algoritmen}
\label{cha:algoritmen}

\section{Onderlinge ligging van twee affiene deelruimten bepalen}
\subsection*{Abstract}
\subsubsection*{Vraag}

\begin{center}
  Gegeven twee affiene deelruimten $p+V$ en $q+W$.
  Bepaal de onderlinge ligging van $p+V$ en $q+W$.
\end{center}

\subsubsection*{Antwoord}
Herneem eerst het deel over affiene deelruimten.\secref{sec:affiene-deelruimten}
Ga in volgorde deze `vragenlijst' af.
\begin{itemize}
\item $V = W \rightarrow p+V \parallel q+W$
\item $V \subseteq W \rightarrow p+V \triangleleft q+W$
\item $S \cap T \neq \emptyset \rightarrow $ snijdend
\item $S \cap T = \emptyset \rightarrow $ kruisend
\end{itemize}

Meestal worden de punten helemaal gegeven, maar van de lineaire deelruimten enkel een basis of een verzameling opspannede vectoren $v$ en $w$.

\begin{itemize}
\item Dun $v$ en $w$ uit tot basissen $v'$ en $w'$ van $V$ en $W$.
  Noem $v'$ en $w'$ trouwens zo dat $v'$ de kleinste verzameling is van de twee.
\item Als $v'$ en $w'$ niet hetzelfde aantal vectoren bevat, dan kunnen $V$ en $W$ al niet gelijk zijn.
\item Als de vectoren in $v'$ lineair afhankelijk zijn van de vectoren in $w'$, dan zijn $V$ en $W$ gelijk of is $V$ een deel van $W$.
\end{itemize}

\subsection*{Voorbeeld}

\subsubsection*{Vraag}
\begin{center}
  Gegeven de punten $p=(1,1,1)$ en $q=(2,-1,0)$ en de lineaire deelruimten $V=<(1,0,1),(1,2,3)>$ en $W=<(1,4,5)>$. Bepaal de onderlinge ligging van $V$ en $W$.
\end{center}

\subsubsection*{Antwoord}
\begin{itemize}
\item $v=\{(1,0,1),(1,2,3)\}$ is al een vrije verzameling, alsook $w=\{(1,4,5)\}$.
\item $v$ en $w$ bevatten niet hetzelfde aantal vectoren, dus $V$ en $W$ zijn zeker niet gelijk.
\item We bekijken of $v\cup w$ nog steeds een vrije verzameling is.
  In feite lossen we dus het volgende stelsel op:
  \[
  a(1,0,1) + b(1,2,3) + c(1,4,5) = 0
  \]
  Als de enige oplossing van dit stelsel $(a,b,c) = (0,0,0)$ is, dan is $W$ geen deel van $V$.
  Omdat we met precies drie vectoren te maken hebben van dimensie $3$ kunnen we dit echter eenvoudiger berekenen met een determinant:
  \[
  \begin{vmatrix}
    1 & 0 & 1\\
    1 & 2 & 3\\
    1 & 4 & 5
  \end{vmatrix}
  \]
  Deze determinant is $0$, dus $v\cup w$ is geen vrij verzameling. $W$ is dus een deel(ruimte) van $V$.
  Dit betekent dat $q+W$ zwak parallel is met $p+V$.
  \[ q+W \triangleleft p+V \]
\end{itemize}

\newpage
\section{Parametervergelijkingen van een affiene deelruimte bepalen}
\subsection*{Abstract}
\subsubsection*{Vraag}
\begin{center}
  Zij $S$ de affiene deelruimte van $\mathbb{A}^{n}$ door het punt $p$ in de richting van een lineaire deelruimte $V$ van $\mathbb{R}^{n}$ met als basis $\beta=\{v_{1},v_{2},\dotsc,v_{k}\}$.
  Bepaal de parametervergelijkingen van $S=p+V$.
\end{center}
\subsubsection*{Antwoord}
Elk punt $x\in \mathbb{A}^{n}$ in $S$ voldoet aan de volgende vector vergelijking:
\[ x\in S \Leftrightarrow \exists \lambda_{1},\lambda_{2},\dotsc ,\lambda_{k}:\ x =  p  + \sum_{i=1}^{k}\lambda_{i}v_{i} \]
Deze vectorvergelijking komt overeen met het volgende stelsel:
\[
x \in S \Leftrightarrow \exists \lambda_{1},\lambda_{2},\dotsc ,\lambda_{k}:\
\left\{
  \begin{array}{c}
    x_{1} = p_{1} + \lambda_{1}v_{11} + \lambda_{2}v_{12} + \dotsb + \lambda_{k}v_{1k}\\
    x_{2} = p_{2} + \lambda_{1}v_{21} + \lambda_{2}v_{22} + \dotsb + \lambda_{k}v_{2k}\\
    \vdots \\
    x_{n} = p_{n} + \lambda_{1}v_{n1} + \lambda_{2}v_{n2} + \dotsb + \lambda_{k}v_{nk}\\
  \end{array}
\right.
\]

\subsection*{Voorbeeld}
\subsubsection*{Vraag}
\begin{center}
  Zij $S$ de affiene deelruimte van $\mathbb{A}^{4}$ door het punt $p= (1,2,3,4)$ in de richting van $V = <(0,1,-1,0), (4,6,3,5)>$.
  Bepaal de parametervergelijkingen van $S$.
\end{center}
\subsubsection*{Antwoord}
\[
x \in \Leftrightarrow \exists \lambda,\mu:\ x = (1,2,3,4) + \lambda (0,1,-1,0) + \mu (4,6,3,5)
\]
\[
x \in S \Leftrightarrow \exists \lambda, \mu:\
\left\{
  \begin{array}{c}
    x_{1} = 1 + 0\lambda + 4\mu\\
    x_{2} = 2 + 1\lambda + 6\mu\\
    x_{3} = 3 - 1\lambda + 3\mu\\
    x_{4} = 4 + 0\lambda + 5\mu\\
  \end{array}
\right.
\]

\newpage
\section{Carthesische vergelijkingen van een affiene deelruimte bepalen}
\subsection*{Abstract}
\subsubsection*{Vraag}
\begin{center}
  Zij $S$ de affiene deelruimte van $\mathbb{A}^{n}$ door het punt $p$ in de richting van een lineaire deelruimte $V$ van $\mathbb{R}^{n}$ met als basis $\beta=\{v_{1},v_{2},\dotsc,v_{k}\}$.
  Bepaal de Carthesische van $S=p+V$.
\end{center}

\subsubsection*{Antwoord}
De vergelijkingen voor $S$ halen we uit de volgende:
\[
x \in S \Leftrightarrow
rang
\begin{pmatrix}
  x_{1}-p_{1} & x_{2}-p_{2} & \hdots & x_{n}-p_{n} \\
  v_{11} & v_{12} & \hdots & v_{1n} \\
  v_{21} & v_{22} & \hdots & v_{2n} \\
  \vdots & \vdots & \ddots & \vdots \\ 
  v_{k1} & v_{k2} & \hdots & v_{kn}
\end{pmatrix}
= dim(V)
\]
Wanneer we de determinant van deze matrix ontwikkelen naar de eerste rij moet elke determinant nul zijn, en zo bekomen we de carthesische vergelijkingen.
(Bekijk vooral het voorbeeld, dat zal veel verduidelijken)

\subsection*{Voorbeeld}
\subsubsection*{Vraag}
\begin{center}
  Bepaal in $\mathbb{A}_{4}$ voor de affiene deelruimte door $p=(1,1,5,7)$ in de richting van $V=<(1,4,1,6),(6,4,2,8)>$ de carthesische vergelijkingen.
\end{center}

\subsubsection*{Antwoord}
\[
x \in S \Leftrightarrow
rang
\begin{pmatrix}
  x_{1}-1 & x_{2}-1 & x_{3}-5 & x_{4}-7\\
  1 & 4 & 1 & 6\\
  6 & 4 & 2 & 8
\end{pmatrix}
= 2
\]
Dit komt neer op de volgende vergelijkingen.
\[
x \in S \Leftrightarrow
\left\{
  \begin{array}{c}
    \begin{vmatrix}
      x_{2}-1 & x_{3}-5 & x_{4}-7\\
      4 & 1 & 6\\
      4 & 2 & 8
    \end{vmatrix}
    = 0\\
    \begin{vmatrix}
      x_{1}-1  & x_{3}-5 & x_{4}-7\\
      1 & 1 & 6\\
      6 & 2 & 8
    \end{vmatrix}
    =0\\
    \begin{vmatrix}
      x_{1}-1 & x_{2}-1 & x_{4}-7\\
      1 & 4 & 6\\
      6 & 4 & 8
    \end{vmatrix}
    = 0\\
    \begin{vmatrix}
      x_{1}-1 & x_{2}-1 & x_{3}-5\\
      1 & 4 & 1 \\
      6 & 4 & 2 
    \end{vmatrix}
    = 0\\
  \end{array}
\right.
\]
We rekenen alleen de eerste twee uit omdat de dimensie van de gezochte affiene deelruimte maximum $2$ kan zijn.
De andere twee vergelijkingen zullen dus lineair afhankelijk zijn van de eerste twee.
Of uitgeschreven:
\[
x \in S \Leftrightarrow
\left\{
\begin{array}{cccccc}
          &-4x_{2} &-8x_{3}  &+ 4x_{4} &+16 &= 0\\
  -4x_{1} &        &+28x_{3} &-4x_{4} &-106 &= 0\\
(8x_{1} &+28x_{2} &        &-20x_{4} &+172 &=0)\\
(4x_{1}  &+4x_{2} &- 20x_{3} &        &+92 &=0)
\end{array}
\right.
\]
De bekomen vergelijkingen zijn nu lineair onafhankelijk, maar we kunnen ze nog vereenvoudigen.
\[
x \in S \Leftrightarrow
\left\{
\begin{array}{cccccc}
          &-x_{2} &-2x_{3}  &+ x_{4} &+4 &= 0\\
  -2x_{1} &        &+14x_{3} &-2x_{4} &-53 &= 0\\
\end{array}
\right.
\]

\newpage
\section{Som van twee affiene deelruimten bepalen vanuit Parametervergelijkingen}
\subsection*{Abstract}
\subsubsection*{Vraag}
\begin{center}
  Zij $p+V$ en $p+W$ twee affiene deelruimten van $\mathbb{A}^{n}$ met parametervergelijkingen als volgt, bepaal de som ervan.
  \[ x\in p+V \Leftrightarrow \exists \lambda_{1},\lambda_{2},\dotsc,\lambda_{a}:\ x = p + \sum_{i=1}^{a}\lambda_{i}v_{i} \]
  \[ x\in q+W \Leftrightarrow \exists \lambda_{1},\lambda_{2},\dotsc,\lambda_{b}:\ x = p + \sum_{i=1}^{b}\lambda_{i}w_{i} \]
\end{center}

\subsubsection*{Antwoord}
\[ x \in p+(V+W)  \Leftrightarrow \lambda_{1},\lambda_{2},\dotsc,\lambda_{a}\lambda'_{1},\lambda'_{2},\dotsc,\lambda'_{b}:\ x = p + \sum_{i=1}^{a}\lambda_{i}v_{i}+ \sum_{i=1}^{b}\lambda'_{i}w_{i}\]

\subsection*{Voorbeeld}
\subsubsection*{Vraag}
\begin{center}
  Zij $p=(1,2,3)$ een punt van $\mathbb{A}^{3}$ en $V=<(5,4,3),(-2,5,1)>$ en $W=<(1,2,3),(-1,0,1)>$ twee lineaire deelruimten van $\mathbb{R}^{3}$
  Bepaal $(p+V) + (p+W)$.
\end{center}

\subsubsection*{Antwoord}
\[
x\in p+(V+W) \Leftrightarrow \exists \lambda, \gamma, \mu, \nu:\ 
x = (1,2,3) + \lambda (5,4,3) + \gamma (-2,5,1) + \mu (1,2,3) + \nu (-1,0,1)
\]

\newpage
\section{Doorsnede van twee affiene deelruimten bepalen vanuit parametervergelijkingen}
\label{sec:doorsnede-van-affiene-deelruimten-bepalen}
\subsection*{Abstract}
\subsubsection*{Vraag}
\begin{center}
  Zij $p+V$ en $p+W$ twee affiene deelruimten van $\mathbb{A}^{n}$ met parametervergelijkingen als volgt, bepaal de doorsnede ervan.
  \[ x\in p+V \Leftrightarrow \exists \lambda_{1},\lambda_{2},\dotsc,\lambda_{a}:\ x = p + \sum_{i=1}^{a}\lambda_{i}v_{i} \]
  \[ x\in q+W \Leftrightarrow \exists \lambda_{1},\lambda_{2},\dotsc,\lambda_{b}:\ x = p + \sum_{i=1}^{b}\lambda_{i}w_{i} \]
\end{center}

\subsubsection*{Antwoord}
We zoeken de doorsnede van de twee 'stelsels' in de opgave.
Met andere woorden zoeken we $\lambda_{i}$ die aan het volgende stelsel voldoen.
\[ \sum_{i=1}^{a}\lambda_{i}v_{i} = \sum_{i=1}^{b}\lambda_{i}w_{i} \]
Van daaruit berekenen we dan de $x$ die beschreven worden door die $\lambda_{i}$.

\subsection*{Voorbeeld}
\subsubsection*{Vraag}
\begin{center}
  Zij $p+V$ en $q+W$ twee affiene deelruimten van $\mathbb{A}^{3}$ 
  Bepaal de doorsnede van $p+V$ en $q+W$:
  \[ x\in p+V \Leftrightarrow \exists \lambda, \mu:\ x = (1,2,3) + \lambda (2,3,0) + \mu (-1,-1,0) \]
  \[ x\in q+W \Leftrightarrow \exists \nu:\ x = (3,3,3) + \nu (0,-1,0) \]
\end{center}

\subsubsection*{Antwoord}
\[
x\in p+V\cap q+V \Leftrightarrow \exists \lambda, \mu, \nu:\ (1,2,3) + \lambda (2,3,0) + \mu (-1,-1,0) = (3,3,3) + \nu (0,-1,0)
\]
\[
x\in p+V\cap q+V \Leftrightarrow \exists \lambda, \mu, \nu:\ 
\lambda (2,3,0) + \mu (-1,-1,0) + \nu (0,1,0) = (2,1,0)
\]
\[
\begin{array}{rcc}
  \left(
    \begin{array}{ccc|c}
      2 & -1 & 0 & 2\\
      3 & -1 & 1 & 1\\
      0 & 0 & 0 & 0\\
    \end{array}
  \right)
  &\overset{R2 \mapsto R2-\frac{3}{2}R1}{\longleftrightarrow}&
  \left(
    \begin{array}{ccc|c}
      2 & -1 & 0 & 2\\
      0 & \frac{1}{2} & 1 & -2\\
      0 & 0 & 0 & 0\\
    \end{array}
  \right)
  \\
  &\overset{R2 \mapsto 2\cdot R2}{\longleftrightarrow}&
  \left(
    \begin{array}{ccc|c}
      2 & -1 & 0 & 2\\
      0 & 1 & 2 & -4\\
      0 & 0 & 0 & 0\\
    \end{array}
  \right)\\
  &\overset{R1 \mapsto \frac{1}{2}(R1+R2)}{\longleftrightarrow}&
  \left(
    \begin{array}{ccc|c}
      1 & 0 & -1 & -1\\
      0 & 1 & 2 & -4\\
      0 & 0 & 0 & 0\\
    \end{array}
  \right)\\
\end{array}
\]
De oplossingsverzameling van dit stelsel is de volgende:
\[
\{ (\nu-1,-2\nu-4,\nu) \mid \nu \in \mathbb{R} \}
\]
Dit betekent dat dit de doorsnede van $p+V$ en $q+W$ de volgende is: (heel $q+W$.)
\[ x\in \exists \nu:\ x = (3,3,3) + \nu (0,-1,0) \]

\newpage
\section{Doorsnede van twee affiene deelruimten bepalen vanuit Carthesische vergelijkingen}
\subsection*{Abstract}
\subsubsection*{Vraag}
\begin{center}
  Zij $R$ en $S$ twee affiene deelruimten van $\mathbb{A}^{n}$ met carthesische vergelijkingen als volgt, bepaal de onderlinge ligging.
  \[
  x \in R \Leftrightarrow
  \left\{
    \begin{array}{c}
      a_{11}x_{1} + \dotsb + a_{1n}x_{n} = b_{1}\\
      \vdots\\
      a_{k1}x_{1} + \dotsb + a_{kn}x_{n} = b_{k}\\
    \end{array}
  \right.
  ,\quad
  x \in S \Leftrightarrow
  \left\{
    \begin{array}{c}
      c_{11}x_{1} + \dotsb + c_{1n}x_{n} = d_{1}\\
      \vdots\\
      c_{k1}x_{1} + \dotsb + c_{kn}x_{n} = d_{k}\\
    \end{array}
  \right.
  \]
\end{center}
\subsubsection*{Antwoord}
We zoeken de doorsnede van $R$ en $S$ door de vergelijkingen samen te nemen en er ze uit te dunnen tot een vrij stel vergelijkingen.

\subsection*{Voorbeeld}
\subsubsection*{Vraag}
\begin{center}
  Zij $R$ en $S$ de affiene deelruimtes van $\mathbb{A}^{3}$ als volgt, bepaal de onderlinge ligging.
  \[
  x \in R \Leftrightarrow
  \left\{
  x_{1}+x_{2}=1
  \right.
  ,\quad
  x\in S \Leftrightarrow
  \left\{
    \begin{array}{c}
      x_{1}-x_{2} = 2\\
      x_{1} + 3x_{3} = 0\\
    \end{array}
  \right.
  \]
\end{center}

\subsubsection*{Antwoord}
We bepalen eerst de doorsnede van $R$ en $S$ door de vergelijkingen samen te nemen.
\[
  x\in R \cap S \Leftrightarrow
  \left\{
    \begin{array}{c}
      x_{1}+x_{2}=1\\
      x_{1}-x_{2} = 2\\
      x_{1} + 3x_{3} = 0\\
    \end{array}
  \right.
\]
\[
\leftrightarrow
  \left\{
    \begin{array}{c}
      x_{1}=\frac{3}{2}\\
      x_{2}=-\frac{1}{2}\\
      \\
    \end{array}
  \right.
\]
De doorsnede van $R$ en $S$ is dus de rechte daar $(\frac{3}{2},-\frac{1}{2},0)$, evenwijdig met de $x_{3}$ as.

\newpage
\section{Onderlinge ligging van twee affiene deelruimten bepalen vanuit zowel carthesische als parametervergelijkingen}
\subsection*{Abstract}
\subsubsection*{Vraag}
\begin{center}
  Zij $R$ en $S=p+V$ twee affiene deelruimten van $\mathbb{A}^{n}$ met carthesische vergelijkingen en parametervergelijkingen als volgt, bepaal de onderlinge ligging.
  \[
  x \in R \Leftrightarrow
  \left\{
    \begin{array}{c}
      a_{11}x_{1} + \dotsb + a_{1n}x_{n} = b_{1}\\
      \vdots\\
      a_{k1}x_{1} + \dotsb + a_{kn}x_{n} = b_{k}\\
    \end{array}
  \right.
  ,\quad
  x\in S \Leftrightarrow
  x= p+ \sum_{i=1}^{k}a_{i}v_{i}
  \]
\end{center}

\subsubsection*{Antwoord}
We vullen de parametervergelijkingen in in de carthesische vergelijkingen om de doorsnede van $R$ en $S$ te berekenen.
\extra{en dan?}

\subsection*{Voorbeeld}
\subsubsection*{Vraag}
\begin{center}
  Zij $R$ en $S=p+V$ twee affiene deelruimten van $\mathbb{A}^{n}$ met Carthesische vergelijkingen en parametervergelijkingen als volgt.
\end{center}
\[
x\in R \Leftrightarrow
\left\{
  \begin{array}{c}
    x_{1}+x_{2}+x_{3}+x_{4}=1
  \end{array}
\right.
,\quad
x\in S \Leftrightarrow
\begin{pmatrix}x_{1}\\x_{2}\\x_{3}\\x_{4}\end{pmatrix}
=
\begin{pmatrix}1\\1\\1\\1\\\end{pmatrix}
+\lambda
\begin{pmatrix}0\\1\\-1\\0\\\end{pmatrix}
\]

\subsubsection*{Antwoord}
We vullen de parametervergelijkingen in in de Carthesische vergelijkingen.
\[ 1+(1+\lambda)+(1+\lambda) + 1 = 1 \]
Deze vergelijking is vals, dus de doorsnede van $R$ en $S$ is leeg.
\extra{en nu?}

\newpage
\section{Vergelijkingen opstellen van een rechte}
\subsection*{Abstract}
\subsubsection*{Vraag}
\begin{center}
  Zij $p$ en $q$ twee punten in $\mathbb{A}^{n}$, bepaal de rechte door $p$ en $q$.
\end{center}
\subsubsection*{Antwoord}
Bereken eerst $\overrightarrow{pq}= q-p$, u zal het nodig hebben!
\begin{itemize}
\item Parametervergelijking:
  \[
  x\in pq \Leftrightarrow \exists \lambda
  x = p + \lambda \overrightarrow{pq} = p + \lambda (q-p)
  \]
\item Carthesische vergelijkingen:
  \[
  x\in pq \Leftrightarrow
  rang
  \begin{pmatrix}
    x_{1}-p_{1} & x_{2}-p_{2} & \hdots & x_{n}-p_{n}\\
    \overrightarrow{pq}_{1} & \overrightarrow{pq}_{2} & \hdots & \overrightarrow{pq}_{n}\\
  \end{pmatrix}
  =
  rang
  \begin{pmatrix}
    x_{1}-p_{1} & x_{2}-p_{2} & \hdots & x_{n}-p_{n}\\
    q_{1}-p_{1} & q_{2}-p_{2} & \hdots & q_{n}-p_{n}\\
  \end{pmatrix}
  = 1
  \]
  Elke $2\times 2$ deelmatrix hiervan moet dus determinant $0$ hebben.
\item Barycentrische vergelijkingen:
  \[
  x\in pq \Leftrightarrow \exists \lambda
  x = \lambda p_{1} + (1-\lambda)p_{2}
  \]
\end{itemize}

\subsection*{Voorbeeld}
\subsubsection*{Vraag}
\begin{center}
  Zij $p=(8,6,-1,1)$ en $q=(-1,0,4,2)$ twee punten in $\mathbb{A}^{4}$, bepaal de rechte door $p$ en $q$.
\end{center}

\subsubsection*{Antwoord}
\[ q-p= (-9,-6,5,1) \]

\begin{itemize}
\item Parametervergelijkingen:
  \[
  x\in pq \Leftrightarrow \exists \lambda:\ 
  x = (8,6,-1,1) + \lambda (-9,-6,5,1)
  \]
\item Carthesische vergelijkingen:
  \[
  x \in pq \Leftrightarrow
  rang
  \begin{pmatrix}
    x_{1}-8 & x_{2}-6 & x_{3}+1 & x_{4}-1\\
    -9 & -6 & 5 & 1\\
  \end{pmatrix}
  =1
  \]
  \[
  x \in pq \Leftrightarrow
  \left\{
  \begin{array}{c}
    \begin{vmatrix}
      x_{1}-8 & x_{2}-6\\
      -9 & -6\\
    \end{vmatrix}
    =0\\
    \begin{vmatrix}
      x_{1}-8 & x_{3}+1\\
      -9 & 5\\
    \end{vmatrix}
    =0\\
    \begin{vmatrix}
      x_{1}-8 & x_{4}-1\\
      -9 & 1\\
    \end{vmatrix}
    =0\\
  \end{array}
  \right.
  \]
  \[ 
  x \in pq \Leftrightarrow
  \left\{
    \begin{array}{ccccccc}
      -6x_{1} &+9x_{2} &        &       & -6 &= 0\\
      5x_{1}  &       &+9x_{3}  &       &-31 &= 0\\
      x_{1}   &       &        &+9x_{4} &-17 &= 0\\
    \end{array}
  \right.
  \]
\item Barycentrische vergelijkingen:
  \[
  x\in pq \Leftrightarrow \exists \lambda:\ 
  x = \lambda (8,6,-1,1) + (1- \lambda)(-1,0,4,2)
  \]
\end{itemize}

\newpage
\section{Parametervergelijkingen omzetten naar Carthesische vergelijkingen}
\subsection*{Abstract}
\subsubsection*{Vraag}
\subsubsection*{Antwoord}

\subsection*{Voorbeeld}
\subsubsection*{Vraag}
\subsubsection*{Antwoord}
\extra{hoe?}

\newpage
\section{Carthesische vergelijkingen omzetten naar parametervergelijkingen}
\subsection*{Abstract}
\subsubsection*{Vraag}
\begin{center}
  Zij $S$ een euclidische deelruimte van $\mathbb{E}^{n}$ met de volgende Carthesische vergelijkingen.
  Bepaal de parametervergelijkingen.
\end{center}
\[
  x \in R \Leftrightarrow
  \left\{
    \begin{array}{cc}
      a_{11}x_{1} + \dotsb + a_{1n}x_{n} &= b_{1}\\
      \vdots\\
      a_{k1}x_{1} + \dotsb + a_{kn}x_{n} &= b_{k}\\
    \end{array}
  \right.
\]

\subsubsection*{Antwoord}
Los het stelsel op naar de $x_{i}$.
De oplossingsverzameling zal parameters bevatten.
We vinden zo de parametervergelijkingen.

\subsection*{Voorbeeld}
\subsubsection*{Vraag}
\begin{center}
  Zij $S$ een euclidische deelruimte van $\mathbb{E}^{5}$ met de
  volgende Carthesische vergelijkingen.  Bepaal de
  parametervergelijkingen.
  \[
  x \in R \Leftrightarrow
  \left\{
    \begin{array}{cc}
      x_{1}-x_{2}-x_{3}+x_{4}-2x_{5}&=4\\
      x_{1} + x_{2} -x_{3} + x_{5} &=5
    \end{array}
  \right.
  \]
\end{center}

\subsubsection*{Antwoord}
\[
  \leftrightarrow
  \left(
    \begin{array}{ccccc|c}
      1 & -1 & -1 & 1 & -2 & 4\\
      1 &  1 & -1 & 0 &  1 & 5
    \end{array}
  \right)
  \leftrightarrow
  \left(
    \begin{array}{ccccc|c}
      1 & 0 & -1 &  \frac{1}{2} & -\frac{1}{2} & \frac{9}{2}\\
      0 & 1 & 0  & -\frac{1}{2} &  \frac{3}{2} & \frac{1}{2}
    \end{array}
  \right)
\]
De oplossingsverzameling $S$ hiervan is de volgende:
\[
S = 
\left\{
  \left( \frac{9}{2}, \frac{1}{2}, 0, 0 ,0 \right) + \lambda \left( 1,0,1,0,0 \right) + \mu \left( -\frac{1}{2}, \frac{1}{2},0,1,0\right) + \nu \left( \frac{1}{2},-\frac{3}{2},0,0,1\right) 
  \mid \lambda,\mu,\nu \in \mathbb{R}
\right\}
\]



\newpage
\section{Het orthogonaal complement bepalen}
\label{sec:orthogonaal-complement-bepalen}
\subsection*{Abstract}
\subsubsection*{Vraag}
\begin{center}
  Gegeven een lineaire deelruimte $V$ van $\mathbb{R}^{n}$, bepaal het orthogonaal complement $V^{\bot}$.
\end{center}

\subsubsection*{Antwoord}
Voor elke vector $v_{i}$ in een basis voor $V$ stellen we een vergelijking op voor een vector $x$ in $V^{\bot}$:
\[ \forall i: v_{i}\cdot x = 0 \]
We lossen dit stelsel op om de parametervergelijking te vinden van $V^{\bot}$.

\subsection*{Voorbeeld}
\subsubsection*{Vraag}
\begin{center}
  Zij $V$ een lineaire deelruimte van $\mathbb{R}^{3}$ als volgt, bepaal het orthogonaal complement.
  \[ V = \{ \lambda(1,7,-5) \mid \lambda \in \mathbb{R} \} \]
\end{center}
\subsubsection*{Antwoord}
Voor elke vector in de basis van $V$ stellen we een vergelijking op: (in dit geval dus maar \'e\'en)
Voor elke vector $w=(x,y,z)$ van $V^{\bot}$ geldt het volgende:
\[ \left\{ (x,y,z) \cdot (1,7,-5) = 0\right. \Leftrightarrow \left\{ x+7y-5z=0 \right. \]
We lossen dit stelsel op:
\[ V^{\bot} = \{ (-7\lambda+5\mu,\lambda,\mu) \mid \lambda,\mu \in \mathbb{R} \} \]

\newpage
\section{De loodrechte euclidische deelruimte door een punt bepalen}
\label{sec:loodrechte-op-euclidische-deelruimte}
\subsection*{Abstract}
\subsubsection*{Vraag}
\begin{center}
  Zij $S=p+V$ een euclidische deelruimte van $\mathbb{E}^{n}$, bepaal de euclidische deelruimte $R$, loodrecht op $S$, door $q$.
\end{center}
\subsubsection*{Antwoord}
We bepalen eerst het orthogonaal complement $V^{\bot}$ van $V$.\secref{sec:orthogonaal-complement-bepalen}
$R$ is dan $q+V^{\bot}$.

\subsection*{Voorbeeld}
\subsubsection*{Vraag}
\begin{center}
  Zij $S=(1,4,5) + \lambda(3,4,2) + \mu (7,6,4)$ een euclidische deelruimte van $\mathbb{E}^{3}$.
  Bepaal de euclidisch deelruimte loodrecht op $S$, door $q=(5,6,4)$.
\end{center}
\subsubsection*{Antwoord}
\begin{itemize}
\item We bepalen eerst de richting $V$ van $S$:
  \[ V = \{ \lambda(3,4,2) + \mu (7,6,4) \mid \lambda,\mu \in
  \mathbb{R} \} \]
\item Vervolgens bepalen we $V^{\bot}$:
  Voor elke vector $w=(x,y,z)\in V^{\bot}$ geldt het volgende:
  \[
  \left\{
    \begin{array}{cc}
      (x,y,z) \cdot (3,4,2)&= 0\\
      (x,y,z) \cdot (7,6,4)&= 0
    \end{array}
  \right.
  \Leftrightarrow
  \left\{
    \begin{array}{cc}
      3x+4y+2z&= 0\\
      7x+6y+4z&= 0
    \end{array}
  \right.
  \longleftrightarrow
  \begin{pmatrix}
    1 & 0 & \dfrac{2}{5}\\
    0 & 1 & \dfrac{1}{5}
  \end{pmatrix}
  \]
  \[
  V^{\bot} = \{ \lambda(-2, - 1, 5) \mid \lambda \in \mathbb{R} \}
  \]
\item We kunnen $R$ nu eenvoudigweg opschrijven:
  \[ R = q+V^{\bot} \leftrightarrow x = (5,6,4) + \lambda (-2, - 1, 5) \]
\end{itemize}

\newpage
\section{De gemeenschappelijke loodlijn van twee rechten bepalen}
\label{sec:gemeenschappelijke-loodlijn-bepalen}
\subsection*{Abstract}
\subsubsection*{Vraag}
\begin{center}
  Gegeven twee kruisende rechten $L_{1}=p_{1}+\lambda v_{1}$ en $L_{2}=p_{2}+\mu v_{2}$ in $\mathbb{E}^{3}$, bepaal de gemeenschappelijke loodlijn die beide rechten snijdt.
\end{center}

\subsubsection*{Antwoord}
\begin{itemize}
\item
  De gemeenschappelijke loodlijn van twee rechten staat loodrecht op beide rechten, dus elke vector $x$ uit de richting van de gemeenschappelijke loodlijn staat loodrecht op de richting van $L_{1}$ en $L_{2}$.
  \[
  \left\{
    \begin{array}{cc}
      x \cdot v_{1} &= 0\\
      x \cdot v_{2} &= 0
    \end{array}
  \right.
  \]
  Wanneer we dit stelsel oplossen bekomen we een vergelijking voor de richting $<v_{3}>$ van de gemeenschappelijke loodlijn.
\item We lossen tenslotte vervolgens de volgend stelsel op om het snijpunt $p$ van $L_{1}$ met de gezochte rechte te vinden.
  \[ p_{2} + a v_{2} + b v_{3} = p_{1} + c v_{1} \]
\item De gemeenschappelijke rechte is nu $p+<v_{3}>$.
\end{itemize}



\subsection*{Voorbeeld}
\subsubsection*{Vraag}
\begin{center}
  Gegeven twee rechten $L_{1}= (2,0,1) + \lambda(1,2,0)$ en $L_{2}= (-1,0,4) + \mu(2,0,1)$, bepaal de gemeenschappelijke loodlijn die beide rechten snijdt.
\end{center}

\subsubsection*{Antwoord}
\begin{itemize}
\item Elke vector $w=(x,y,z)$ in de richting van de gemeenschappelijke loodlijn staat loodrecht op de richting van zowel $L_{1}$ als $L_{2}$.
  \[
  \left\{
    \begin{array}{cc}
      (x,y,z) \cdot (1,2,0) = 0\\
      (x,y,z) \cdot (2,0,1) = 0\\
    \end{array}
  \right. 
  \longleftrightarrow
  \left\{
    \begin{array}{cc}
      x+2y &= 0\\
      2x+z &= 0\\
    \end{array}
  \right. 
  \longleftrightarrow
  \begin{pmatrix}
    1 & 0 & \frac{1}{2}\\
    0 & 1 & -\frac{1}{4}
  \end{pmatrix}
  \]
  De richting van de gemeenschappelijke loodlijn is dus $V$:
  \[ V = \{ \lambda(-2,1,4)  \mid \lambda \in \mathbb{R} \} \]
\item 
  \[ (-1,0,4) + a (2,0,1) + b (-2,1,4) = (2,0,1) + c(1,2,0) \]
  \[
  \longleftrightarrow
  \begin{pmatrix}
    2 & -2 & -1 & 3\\
    0 & 1 & -2 & 0\\
    1 & 4 & 0 & -3
  \end{pmatrix}
  \longleftrightarrow
  \begin{pmatrix}
    1 & 0 & 0 & \frac{249}{157}\\
    0 & 1 & 0 & \frac{180}{157}\\
    0 & 0 & 0 & \frac{90}{157}
  \end{pmatrix}
  \]
  Het snijpunt $p$ van de gemeenschappelijke loodlijn met $L_{1}$ is dus $\left(\frac{249}{157},\frac{180}{157},\frac{90}{157}\right)$.
\item De gemeenschappelijke loodlijn is dan $\leftrightarrow \left(\frac{249}{157},\frac{180}{157},\frac{90}{157}\right) + \lambda (-2,1,4)$.
\end{itemize}



\newpage
\section{De afstand bepalen tussen een punt en een rechte}
\subsection*{Abstract}
\subsubsection*{Vraag}
\begin{center}
  Zij $L=p+V$ een rechte en $q$ een punt uit $\mathbb{E}^{n}$.
  Bepaal de afstand tussen $q$ en $L$.
\end{center}

\subsubsection*{Antwoord}
\begin{itemize}
\item Bepaal de euclidische deelruimte $D$, loodrecht op $L$ door $q$.\secref{sec:loodrechte-op-euclidische-deelruimte}
\item Bepaal het snijpunt $s$ van $D$ met $L$.\secref{sec:doorsnede-van-affiene-deelruimten-bepalen}
\item Bepaal de afstand tussen $s$ en $q$.
\end{itemize}

\subsection*{Voorbeeld}
\subsubsection*{Vraag}
\begin{center}
  Zij $L=p+V$ met de volgende parametervergelijking en $q=(1,1,1)$ een
  punt in $\mathbb{E}^{3}$.
  \[ L \leftrightarrow x = (1,1,0) + \lambda (1,7,-5) \]
  Bepaal de afstand tussen $q$ en $L$.
\end{center}

\subsubsection*{Antwoord}
\begin{itemize}
\item 
  \[
  D = (1,1,1) + \lambda(-7,1,0) + \mu(5,0,1)
  \]
\item 
  \[
  (1,1,0) + a(1,7,-5) = (1,1,1) + b(-7,1,0) + c(5,0,1)
  \]
  \[
  \longleftrightarrow
  \begin{pmatrix}
    -7 & 5  & -1 & 0\\
    1  & 0  & -7 & 0\\
    0  & 1  &  5 & -1
  \end{pmatrix}
  \longleftrightarrow
  \begin{pmatrix}
    1  & 0  &  0 & -\frac{1}{2}\\
    0  & 1  &  0 & -\frac{5}{7}\\
    0  & 0  &  1 & -\frac{1}{14}
  \end{pmatrix}
  \]
  We vinden dat $(a,b,c)$ gelijk is aan $\left(-\frac{1}{2}, -\frac{5}{7}, -\frac{1}{14} \right)$
  $s$ ziet er dus als volgt uit:
  \[ s = (1,1,0) - \frac{1}{2}(1,7,-5)= (\frac{1}{2},-\frac{5}{2},-\frac{5}{2}) \]
\item
  \[ d(s,q) = \Vert q-s \Vert = \sqrt{(q-s)\cdot (q-s)} = \sqrt{\left(\frac{1}{2},\frac{7}{2},\frac{7}{2}\right)} = \sqrt{\frac{99}{4}} = \frac{3\sqrt{11}}{2}\]
\end{itemize}

\newpage
\section{De afstand bepalen tussen twee rechten}
\subsection*{Abstract}
\subsubsection*{Vraag}
\begin{center}
  Zij $L_{1}$ en $L_{2}$ twee rechten in $\mathbb{E}^{n}$.
  Bepaal de afstand tussen $L_{1}$ en $L_{2}$.
\end{center}
\subsubsection*{Antwoord}
\begin{itemize}
\item Bepaal de gemeenschappelijke loodlijn $L$ van $L_{1}$ en $L_{2}$\secref{sec:gemeenschappelijke-loodlijn-bepalen}
\item Zoek het snijpunt $p$ van $L$ en $L_{1}$.
\item Zoek het snijpunt $q$ van $L$ en $L_{2}$
\item Bepaal de afstand tussen $p$ en $q$.
\end{itemize}

\subsection*{Voorbeeld}
\subsubsection*{Vraag}
\begin{center}
  Gegeven zijn volgende kruisende rechten in $\mathbb{E}^{3}$, bepaal de afstand tussen $L_{1}$ en $L_{2}$.
  \[
  L_{1} \leftrightarrow
  \left\{
    \begin{array}{cc}
      x+2y+3z&=3\\
      2x-y-z&=1\\
    \end{array}
  \right.
  \qquad
  L_{2} \leftrightarrow
  \left\{
    \begin{array}{cc}
      x-y-z&=-1\\
      4x+2y-z&=2\\
    \end{array}
  \right.
  \]
\end{center}

\subsubsection*{Antwoord}
\begin{itemize}
\item We bepalen de parametervergelijkingen van $L_{1}$ en $L_{2}$.
  \[ 
  L_{1} \leftrightarrow
  \begin{pmatrix}
    1 & 2 & 3 & 3\\
    2 & -1 & -1 & 1
  \end{pmatrix}
  \longleftrightarrow
  \begin{pmatrix}
    1 & 0 & \frac{1}{5} & 1\\
    0 & 1 & \frac{7}{5} & 1
  \end{pmatrix}
  \qquad
  L_{2} \leftrightarrow
  \begin{pmatrix}
    1 & -1 & -1 & -1\\
    4 & 2 & -1 & 2
  \end{pmatrix}
  \longleftrightarrow
  \begin{pmatrix}
    1 & 0 & -\frac{1}{2} & 0\\
    0 & 1 & \frac{1}{2} & 1
  \end{pmatrix}
  \]
  \[
  L_{1} \leftrightarrow x = (1,1,0) + \lambda (1,7,-5)
  \qquad
  L_{2} \leftrightarrow x = (0,1,0) + \mu (1,-1,2)
  \]
\item We bepalen de gemeenschappelijk loodlijn $L$ van $L_{1}$ en $L_{2}$.
  Elke vector (x,y,z) in de richting van $L$ voldoet aan volgend stelsel.
  \[
  \left\{
    \begin{array}{cc}
      (1,7,-5)(x,y,z) &= 0\\
      (1,-1,2)(x,y,z) &= 0
    \end{array}
  \right.
  \longleftrightarrow
  \left\{
    \begin{array}{cc}
      x+7y-5z &= 0\\
      x-y+2z  &= 0
    \end{array}
  \right.
  \]
  We vinden dat de richting van $L$ er als volgt uitziet
  \[
  \{ \lambda(-9,7,8) \mid \lambda \in \mathbb{R} \}
  \]
\item We zoeken een punt $p$ op $L_{1}$ en $L$, en een punt $q$ op $L_{2}$ en $L$.
  Met \'e\'en stelsel vinden we $p$, $q$ en $d(p,q)$:
  \[ \left((0,1,0) + \mu (1,-1,2)\right) - \left( (1,1,0) + \lambda (1,7,-5)\right) = \nu (-9,7,8) \]
  \[ \lambda (-1,-7,5) + \mu (1,-1,2) + \nu (9,-7,-8) = (1,0,0) \]
  De oplossing van dit stelsel is als volgt:
  \[
  \left\{ \left( -\frac{11}{97}, \frac{91}{194}, \frac{9}{194} \right) \right\}
  \]
  We vinden $p$ en $q$ als volgt:
  \[ p = \frac{1}{97}(86,20,55), \quad q= \frac{1}{194}(91,0,188)\]
\item De afstand tussen $p$ en $q$ kunnen we nu rechtstreeks uit $\nu$ en $(-9,7,8)$ halen, of zelf berekenen met $p$ en $q$:
  \[ q-p = \left(-\frac{81}{194}, -\frac{20}{97}, \frac{39}{97}\right)\]
  \[ d(p,q) = \sqrt{(q-p)^{2}} \approx 0.6 \]
  \[ d(p,q) = \nu \Vert (-9,7,8) \Vert= \frac{9}{194}\sqrt{194} \approx 0.6 \]
\end{itemize}


\newpage
\section{De afstand bepalen tussen een punt en een hypervlak }
\subsection*{Abstract}
\subsubsection*{Vraag}
\begin{center}
  Zij $V$ een hypervlak en $p$ een punt uit $\mathbb{E}^{n}$.
  Bepaal de afstand $d(P,h)$ tussen $V$ en $p$.
  \[ V \leftrightarrow   \sum_{i=1}^{n}a_{i}x_{i} + b = 0 \]
\end{center}
\subsubsection*{Antwoord}
\[d(P,h) = \frac{b+ \sum_{i=1}^{n}a_{i}p_{i}}{\sqrt{\sum_{i=1}^{n}a_{1}^{2}}}\]

\subsection*{Voorbeeld}
\subsubsection*{Vraag}
\begin{center}
  Zij $H$ een hypervlak met de volgende carthesische vergelijking, wat is de afstand van $H$ tot $q=(5,4,2,5,9) \in \mathbb{E}^{5}$.
  \[ \left\{ x_{1}-x_{2}-x_{3}+x_{4}-2x_{5} = 4  \right. \]
\end{center}
\subsubsection*{Antwoord}
\[ d(q,H) = \frac{-4 + 5-4-2+5-18}{\sqrt{1^{2}+1^{2}+1^{2}+1^{2}+2^{2}}} = \frac{-18}{2\sqrt{2}} = \frac{-9\sqrt{2}}{2}\]











\end{document}


%%% Local Variables: 
%%% mode: latex
%%% TeX-master: t
%%% End: 
