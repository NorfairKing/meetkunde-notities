\documentclass[main.tex]{subfiles}
\begin{document}

\chapter{Algoritmen}
\label{cha:algoritmen}

\section{Onderlinge ligging van twee affiene deelruimten bepalen}
\subsection{Abstract}
\subsubsection{Vraag}

\begin{center}
  Gegeven twee affiene deelruimten $p+V$ en $q+W$.
  Bepaal de onderlinge ligging van $p+V$ en $q+W$.
\end{center}

\subsubsection{Antwoord}
Herneem eerst het deel over affiene deelruimten.\secref{sec:affiene-deelruimten}
Ga in volgorde deze `vragenlijst' af.
\begin{itemize}
\item $V = W \rightarrow p+V \parallel q+W$
\item $V \subseteq W \rightarrow p+V \triangleleft q+W$
\item $S \cap T \neq \emptyset \rightarrow $ snijdend
\item $S \cap T = \emptyset \rightarrow $ kruisend
\end{itemize}

Meestal worden de punten helemaal gegeven, maar van de lineaire deelruimten enkel een basis of een verzameling opspannede vectoren $v$ en $w$.

\begin{itemize}
\item Dun $v$ en $w$ uit tot basissen $v'$ en $w'$ van $V$ en $W$.
  Noem $v'$ en $w'$ trouwens zo dat $v'$ de kleinste verzameling is van de twee.
\item Als $v'$ en $w'$ niet hetzelfde aantal vectoren bevat, dan kunnen $V$ en $W$ al niet gelijk zijn.
\item Als de vectoren in $v'$ lineair afhankelijk zijn van de vectoren in $w'$, dan zijn $V$ en $W$ gelijk of is $V$ een deel van $W$.
\end{itemize}

\subsection{Voorbeeld}

\subsubsection{Vraag}
\begin{center}
  Gegeven de punten $p=(1,1,1)$ en $q=(2,-1,0)$ en de lineaire deelruimten $V=<(1,0,1),(1,2,3)>$ en $W=<(1,4,5)>$. Bepaal de onderlinge ligging van $V$ en $W$.
\end{center}

\subsubsection{Antwoord}
\begin{itemize}
\item $v=\{(1,0,1),(1,2,3)\}$ is al een vrije verzameling, alsook $w=\{(1,4,5)\}$.
\item $v$ en $w$ bevatten niet hetzelfde aantal vectoren, dus $V$ en $W$ zijn zeker niet gelijk.
\item We bekijken of $v\cup w$ nog steeds een vrije verzameling is.
  In feite lossen we dus het volgende stelsel op:
  \[
  a(1,0,1) + b(1,2,3) + c(1,4,5) = 0
  \]
  Als de enige oplossing van dit stelsel $(a,b,c) = (0,0,0)$ is, dan is $W$ geen deel van $V$.
  Omdat we met precies drie vectoren te maken hebben van dimensie $3$ kunnen we dit echter eenvoudiger berekenen met een determinant:
  \[
  \begin{vmatrix}
    1 & 0 & 1\\
    1 & 2 & 3\\
    1 & 4 & 5
  \end{vmatrix}
  \]
  Deze determinant is $0$, dus $v\cup w$ is geen vrij verzameling. $W$ is dus een deel(ruimte) van $V$.
  Dit betekent dat $q+W$ zwak parallel is met $p+V$.
  \[ q+W \triangleleft p+V \]
\end{itemize}



\end{document}


%%% Local Variables: 
%%% mode: latex
%%% TeX-master: t
%%% End: 
