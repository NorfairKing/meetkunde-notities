\documentclass[main.tex]{subfiles}
\begin{document}

\chapter{Oefenzittingen}
\label{cha:oefenzittingen}

\section{Oefenzitting 1}
\label{sec:oz1}

\subsection*{Oefening 1}
$T_{p}\mathbb{A}^{n}$ is inderdaad een vectorruimte.\footnote{Zie stelling \ref{st:rakende-ruimte-is-vectorruimte}.}
$\phi$ is inderdaad een isomorfisme van re\"ele vectorruimten.\footnote{Zie stelling \ref{st:phi-isomorphisme}.}

\subsection*{Oefening 2}

\begin{itemize}
\item Nee, $V$ en $W$ zijn ongelijk want $(1,2)$ is lineair onafhankelijk van $(0,-2)$.
\item $q + V$.
\item $p + W$
\end{itemize}

\subsection*{Oefening 3}
\begin{enumerate}
\item $p+V$ en $q+W$
  \[
  \begin{vmatrix}
    1 & 1 & 1\\
    0 & 2 & 4\\
    1 & 3 & 5
  \end{vmatrix}
  = 0
  \Rightarrow W \subsetneq V 
  \]
  $q+W$ is dus zwak parallel met $p+V$.
  \[ p+W \triangleleft q+V \]
\item $p+V$ en $q+R$
  \[
  \begin{vmatrix}
    2 & 9 & 1\\
    2 & 2 & 4\\
    4 & 2 & 5
  \end{vmatrix}
  = 0
  \Rightarrow W \subsetneq R 
  \]
  $p+W$ is dus zwak parallel met $q+R$.
  \[ p+W \triangleleft q+R \]
  
\item $q+W$ en $q+R$
  \[
  \begin{vmatrix}
    1 & 0 & 1\\
    1 & 2 & 3\\
    2 & 2 & 4
  \end{vmatrix}
  = 0
  \text{ en }
  \begin{vmatrix}
    1 & 0 & 1\\
    1 & 2 & 3\\
    0 & 2 & 2
  \end{vmatrix}
  = 0
  \Rightarrow V = R
  \]
  $q+W$ en $q+R$ zijn dus parallel.
  \[ q+W \parallel q+R \]
\end{enumerate}

\TODO{oefening 4}

\subsection*{Oefening 5}
\[ S + T = p + (V + W) = p + <\{(1,0,0),(0,1,0)\}>\]

\subsection*{Oefening 6}
Zie stelling \ref{st:affiene-deelruimten-niet-lege-doorsnede-gelijk}

\subsection*{Oefening 7}
We maken een gevalsonderscheid.
\begin{itemize}
\item $(s,t)$ met $s$ of $t$ groter of gelijk aan $n$ kan geen oplossing zijn want dan zou $S$ of $T$ geen deelruimte zijn van $\mathbb{A}^{n}$.
\item $(n-1,n-1)$ kan wel:
  Kies een $n-1$ dimensionale affiene deelruimte $S = p + V$ van $\mathbb{A}^{n}$.
  Er zit nu minstens \'e\'en punt $q\in \mathbb{A}^{n}$ niet in $S$.
  Beschouw nu de $n-1$ dimensionale affiene deelruimte $T = q + V$.
  $S$ en $T$ zijn parallel en niet gelijk, dus disjunct.\footnote{Zie stelling \ref{st:parallelle-deelruimten-gelijk-of-disjunct}.}
\item $(s,t)$ met $s,t < n-1$ kan ook.
  Beschouw immers een $s$ en $t$ dimensionale deelruimnten van $S$ en $T$ uit het bovenstaande puntje.
  Deze deelruimten zijn nog steeds disjunct.
\end{itemize}
Antwoord:
\[ \{(s,t)\in \mathbb{N}\times\mathbb{N}\ |\ s,t<n\}\]

\subsection*{Oefening 8}
$S$ en $T$ zijn kruisend.
\begin{enumerate}
\item Stel $S = s+V$ en $T = t+W$.
  Beschouw nu $R_{1} = s + (V+W)$
  $R_{1}$ is nu zwak parallel met $T$ want $W \subsetneq (V+W)$.
  $R_{1}$ is bovendien uniek, want stel dat er twee verschillende vlakken $R_{1}$ en $R_{1}'$ zijn die $S$ bevatten en zwak parallel zijn met $T$, dan is $W$ een deelverzameling van zowel de richting van $R_{1}$ als de richting van $R_{1}'$. Bovendien zou $s$ een element zijn van zowel $R_{1}$ als $R_{1}'$, en zouden bijgevolg $R_{1}$ en $R_{1}'$ gelijk zijn.\footnote{Zie stelling \ref{st:zwak-parallelle-deelruimten-deel-of-disjunct}.}
\item Volledig analoog.
\item $R_{1} = s + (V+W)$ en $R_{1} = t + (V+W)$, hebben dezelfde richting en dimensie, en zijn bijgevolg parallel.
\end{enumerate}

\TODO{oefening 9}
 
\subsection*{Oefening 10}
Tegenvoorbeeld:
Kies $S = <e_{1},e_{2}>$ en $T = <e_{3},e_{4}>$ deelruimten van $\mathbb{A}^{4}$.


\section{Oefenzitting 2}
\label{sec:oefenzitting-2}

\subsection{Oefening 1}
Bepaal de parametervergelijkingen en carthesische vergelijkingen van de rechte door $p$ en $q$:
\begin{enumerate}
\item 
\TODO{oefening 1.1}
\item
\TODO{oefening 1.2}
\item
\TODO{oefening 1.3}
\item $p = (2,-1,7)$ en $q=(6,4,-3)$ in $\mathbb{A}^{3}$\\
  \[ \overrightarrow{pq} = (6,4,-3) - (2,-1,7) = (4,5,-10) \]
  Parametervergelijkingen:
  \[ L \leftrightarrow x \in p + <\overrightarrow{pq}> \]
  \[ L \leftrightarrow x = (2,-1,7) + \lambda(4,5,-10) \]
  Carthesische vergelijkingen:
  \[
  L \leftrightarrow
  rang
  \begin{pmatrix}
    (x_{1}-2) & (x_{2}+1) & (x_{3}-7)\\
    4 & 5 & -10
  \end{pmatrix}
  = 1 = dim(rechte)
  \]
  \[
  \Leftrightarrow 
  \begin{vmatrix}
    (x_{1}-2) & (x_{2}+1)\\
    4 & 5
  \end{vmatrix}
  = 0
  \wedge
  \begin{vmatrix}
    (x_{2}+1) & (x_{3}-7)\\
    5 & -10
  \end{vmatrix}
  = 0
  \]
  \[ v_{1}(x_{i}-p_{i}) = v_{i}(x_{1}-p_{1}) \]
  $\rightarrow$
  \[
  \left\{
    \begin{array}{ccc}
      4(x_{2}+1) &=& 5(x_{1}-2)\\
      4(x_{3}-7) &=& -10(x_{1}-2)
    \end{array}
  \right.
  \leftrightarrow
  \left\{
    \begin{array}{ccccc}
      5x_{1}&-4x_{2}&&=&14\\
      5x_{1}&&+2x_{3}&=&24
    \end{array}
  \right.
  \]
\item
\TODO{oefening 1.5}
\end{enumerate}

\subsection{Oefening 2}
Bepaal parametervergelijkingen en carthesische vergelijkingen van het vlak door $p$, $q$ en $r$ in de volgende gevallen.
\begin{enumerate}
\item $p=(0,1,1)$, $q=(1,-1,1)$ en $r=(3,-2,4)$ in $\mathbb{A}^{3}$\\
  \[ \overrightarrow{pq} = (1,-1,1) - (0,1,1) = (1,-2,0) \]
  \[ \overrightarrow{pr} = (3,-2,4) - (0,1,1) = (3,-3,3) \]
  Parametervergelijkingen:
  \[ H \leftrightarrow x = (0,1,1) + \lambda (1,-2,0) + \mu (3,-3,3) \]
  Carthesische vergelijkingen:
  \[
  H \leftrightarrow 
  rang 
  \begin{pmatrix}
    x_{1} & (x_{2}-1) & (x_{3}-1)\\
    1 & -2 & 0\\
    1&-1&1
  \end{pmatrix}
  = 2 = dim(vlak)
  \]
  \[
  \Leftrightarrow 
  \begin{vmatrix}
    x_{1} & (x_{2}-1) & (x_{3}-1)\\
    1 & -2 & 0\\
    1&-1&1
  \end{vmatrix}
  = 0
  \]
  \[ H \leftrightarrow 2x_{1}+ x_{2}-x_{3} = 0 \]
\TODO{oefening 2.1}
\item $p= (1,0,6,1)$, $q=(2,-1,3,7)$ en $r=(0,0,2,1)$ in $\mathbb{A}^{4}$\\
  \[ \overrightarrow{pq} = (2,-1,3,7) - (1,0,6,1) = (1,-1,-3,6) \]
  \[ \overrightarrow{pr} = (0,0,2,1) - (1,0,6,1) = (1,0,-4,0) \]
\TODO{oefening 2.2}
  
\end{enumerate}

\subsection{Oefening 3}
Bepaal parametervergelijkingen en carthesische vergelijkingen van de affiene deelruimten $H$ bepaald door de volgende punten.
\begin{itemize}
\item $p_{0} = (1,0,0,0)$, $p_{1} = (0,1,1,2)$, $p_{2} = (0,0,0,0)$ en $p_{3} = (1,2,3,4)$ in $\mathbb{A}^{4}$\\
  \[ \overrightarrow{p_{1}p_{2}} = (0,1,1,2) - (1,0,0,0) = (-1,1,1,2) \]
  \[ \overrightarrow{p_{1}p_{3}} = (0,0,0,0) - (1,0,0,0) = (-1,0,0,0) \]
  \[ \overrightarrow{p_{1}p_{4}} = (1,2,3,4) - (1,0,0,0) = (0,2,3,4) \]
  Parametervergelijkingen:
  \[ H \leftrightarrow x = (1,0,0,0) + \lambda(-1,1,1,2) + \mu(-1,0,0,0) + \nu(0,2,3,4) \]
  Carthesische vergelijkingen:
  \[
  H \leftrightarrow
  rang
  \begin{pmatrix}
    x_{1} & x_{2} & x_{3} & x_{4}\\
    1 & 2 & 3 & 4\\
    1 & 0 & 0 & 0\\
    0 & 1 & 1 & 2\\
  \end{pmatrix}
  = 3
  \]
  \[
  \Leftrightarrow
  \begin{vmatrix}
    x_{1} & x_{2} & x_{3} & x_{4}\\
    1 & 2 & 3 & 4\\
    1 & 0 & 0 & 0\\
    0 & 1 & 1 & 2\\
  \end{vmatrix}
  = 
  \begin{vmatrix}
    x_{2} & x_{3} & x_{4}\\
    2 & 3 & 4\\
    1 & 1 & 2\\
  \end{vmatrix}
  = 0
  \]
  \[
  H \leftrightarrow 2x_{2} - x_{4} = 0
  \]
\item 
  \TODO{ oefening 3.2 }
\end{itemize}

\subsection{Oefening 4}
  \TODO{ oefening 4 }
\subsection{Oefening 5}
Gegeven een is volgend stelsel vergelijkingen.
\[
\left\{
  \begin{array}{cc}
    1 + x_{1} + x_{2} - x_{3} - 6x_{4} &= 0\\
    1 - x_{1} + 2x_{2} - x_{3} - 2x_{4} &= 0\\
    1 - 3x_{1} + 3x_{2} - x_{3} + 2x_{4} &= 0
  \end{array}
\right.
\]
\begin{itemize}
\item Wat is de dimensie van de bijhorende affiene deelruimte $H$ van $\mathbb{A}^{4}$?
\item Geef de Parametervergelijkingen voor deze affiene deelruimte.
\end{itemize}

\[ 
A = 
\begin{pmatrix}
    1 & 1 & -1 & -6 & -1\\
    -1 & 2 & -1 & -2 & -1\\
    -3 & 3 & -1 & 2 & -1
\end{pmatrix}
\rightarrow
\begin{pmatrix}
  1 & 0 & -\frac{1}{3} & -\frac{10}{3} & -\frac{1}{3}\\
  0 & 1 & -\frac{2}{3} & -\frac{8}{3}  & -\frac{2}{3}\\
  0 & 0 & 0            & 0             & 0
\end{pmatrix}
\] 
\begin{itemize}
\item $dim(H) = n - rang(A) = 4 - 2 = 2$.
\item Stelsel oplossen:
  \[
  \left\{
  \begin{array}{cl}
    x_{1} &= -\frac{1}{3} \lambda + -\frac{10}{3} \mu -\frac{1}{3}\\
    x_{2} &= -\frac{2}{3} \lambda + -\frac{8}{3}  \mu -\frac{2}{3}\\
    x_{3} &= \lambda\\
    x_{4} &= \mu
  \end{array}
  \right.
  \]
  \[ H \leftrightarrow x = \left(-\frac{1}{3},-\frac{2}{3},0,0-\right) + \lambda\left(-\frac{1}{3},-\frac{2}{3},1,0\right) + \mu\left(-\frac{10}{3},-\frac{8}{3},0,1\right) \]
\end{itemize}

\subsection{Oefening 6}
Gegeven is de rechte $L$ in $\mathbb{A}^{3}$.
\[ 
\left\{
  \begin{array}{cc}
    x + y - z + 7 = 0\\
    2x - y + z + 8 = 0
  \end{array}
\right.
\]
\begin{enumerate}
\item Geef de richting van $L$.\\
  Het volgende homogeen stelsel geeft ons de richting van $L$ in zijn oplossing:
\[ 
\left\{
  \begin{array}{cc}
    x + y - z = 0\\
    2x - y + z = 0
  \end{array}
\right.
\]
\[
\begin{pmatrix}
  1 & 1 & -1 & 0\\
  2 & -1 & 1 & 0
\end{pmatrix}
\rightarrow
\begin{pmatrix}
  1 & 0 & 0  & 0\\
  0 & 1 & -1 & 0
\end{pmatrix}
\]
De richting van $L$ is dus $V$:
\[
V = \{ (0,-\lambda,\lambda)\ |\ \lambda \in \mathbb{R}^{3} \} 
\]
\item Bepaal de rechte $L'$ door het punt $p = (-1,2,0)$ parallel met $L$.\\
\[ L' = p + V = (-1,2,0) + \lambda (0,-1,1) \]
OF bepaal $a$ en $b$ in volgend stelsel voor de carthesische vergelijkingen door $p$ in te vullen.
\[ 
\left\{
  \begin{array}{cc}
    x + y - z + a = 0\\
    2x - y + z + b = 0
  \end{array}
\right.
\rightarrow 
\left\{
  \begin{array}{cc}
    a = -1\\
    b = 4
  \end{array}
\right.
\]
Dan krijgen we volgende carthesische vergelijkingen voor $L$:
\[ 
\left\{
  \begin{array}{cc}
    x + y - z  -1 = 0\\
    2x - y + z + 4 = 0
  \end{array}
\right.
\]
\end{enumerate}

\subsection{Oefening 7}
Gegeven zijn de rechten $L$ en $L'$ in $\mathbb{A}^{3}$ met als vergelijkingen:
\[
L = 
\left\{
\begin{array}{cl}
x + 2y + 3z &= 3\\
2x -y -z &= 1    
\end{array}
\right.
\quad\text{ en }\quad
L' = 
\left\{
\begin{array}{cl}
x - y - z &= -4\\
4x +2y -z &= 2
\end{array}
\right.
\]
\begin{itemize}
\item Toon aan dat deze rechten kruisend zijn.\\
We berekenen de doorsnede van $L$ en $L'$ door de vergelijkingen samen te nemen.
\[
\left\{
\begin{array}{cl}
x + 2y + 3z &= 3\\
2x -y -z &= 1\\
x - y - z &= -4\\
4x +2y -z &= 2
\end{array}
\right.
\]
\[
\rightarrow
\begin{pmatrix}
  1 & 2 & 3 & 3\\
  0 &-5 &-7 &-5\\
  0 & 0 & 1 & -20\\
  0 & 0 &-5 & 4
\end{pmatrix}
\]
We zien dat dit stelsel strijdig is, en de doorsnede bijgevolg leeg is.
\[ L \leftrightarrow x = (1,1,0) + \lambda(1,7,-5) \]
\[ L' \leftrightarrow x = (-1,3,0) + \lambda(1,-1,2) \]
We zien dat de rechten een verschillende righting hebben, dus ze zijn kruisend.

\item Bepaal de rechte met richting $(0,1,-1)$ die deze rechten snijdt.\\
Zij $H_{1}$ het vlak dat $L$ bevat en ook de richting $(0,1,-1)$ heeft.
\[ H_{1} \leftrightarrow x = (1,1,0) + \lambda(1,7,-5) + \mu (0,1,-1) \]
We zoeken dan het snijpunt $p$ van $H_{1}$ en $L_{2}$, dat is een punt op $L_{3}$.
\[ (-1,3,0) + \nu(1,-1,2) = (1,1,0) + \lambda(1,7,-5) + \mu (0,1,-1) \]
\[
\rightarrow
\begin{pmatrix}
  1 & 0 &-1 &-2\\
  7 & 1 & 1 & 2\\
  -5 &-1 &-2 & 0
\end{pmatrix}
\leftrightarrow
\begin{pmatrix}
  1 & 0 &-1 &-2\\
  7 & 1 & 1 & 3\\
  5 &-1 &-2 & 0
\end{pmatrix}
\]
De oplossing van dit stelsel is $(\lambda,\mu,\nu) = (4,-32,6)$ dus het snijpunt vinden we als volgt:
\[ (-1,3,0) + 6(1,-1,2) = (5,-3,12) \]
We kennen dan de richting van $L_{3}$ en een aangrijpingspunt van $L_{3}$.
\[ L_{3} = (5,-3,12) + \lambda (0,1,-1) \]
\end{itemize}

\TODO{Oefening 8}
\TODO{Oefening 9}
\TODO{Oefening 10}
\TODO{Oefening 11}
\TODO{Oefening 12}

\end{document}