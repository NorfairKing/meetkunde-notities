\documentclass[main.tex]{subfiles}
\begin{document}

\chapter{Geodeten}
\label{cha:geodeten}

\begin{de}
  Een kromme noemt men een \term{pregeodeet} als ze het volgende infimum realiseert tussen twee punten.
  \[
  d(p,q) = \inf 
  \left\{
    \int_{a}^{b}\Vert \alpha'(t)\ dt \ |\ \alpha(a) = p, \alpha(b) q
  \right\}
  \]
\end{de}

\begin{de}
  Een \term{geodeet} is een pregeodeet die we kunnen herparametriseren tot een kromme met constante snelheid.
\end{de}

\section{Krommingseigenschappen}
\label{sec:kromm}



\end{document}
